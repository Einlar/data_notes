%&latex
%
\documentclass[../template.tex]{subfiles}
\usepackage{graphicx}

\begin{document}

\chapter{Ising Model}
The \textbf{Ising Model} is a paradigmatic model of Statistical Mechanics, originally created to explain the phenomenon of \textbf{paramagnetism}. It is exactly solvable in $d=1$ or $d=2$, but no exact solutions exist for higher dimension.   

Let us start with the grand canonical formulation of a fluid:
\begin{align*}
    H_{N}(\mathbb{Q}, \mathbb{P}) = \sum_{i=1}^N \frac{P_i^2}{2m} + V_N(\mathbb{Q}) 
\end{align*}
The grand canonical partition function is given by:
\begin{align*}
    e^{\beta PV} = \sum_N \frac{1}{N!} \int \prod_i \frac{\dd[3]{\bm{q_i} \dd[3]{\bm{p_i}}}}{h^3} e^{-\beta [H_N(\mathbb{Q}, \mathbb{P}) - \mu N]} 
\end{align*}
The integral:
\begin{align*}
    \int \frac{\dd[3N]{\mathbb{P}}}{h^{3N}}  \exp\left(-\sum_{i=1}^N \frac{P_i^2}{2m} \beta \right) = \left(\frac{2 \pi m}{\beta h} \right)^{\frac{3N}{2}} = \lambda^{-3N} \qquad \left[\frac{\dd[3]{\bm{p}}}{h^3} \right] = \frac{1}{\si{\m^3}} 
\end{align*}
The quantity $\lambda$ is the thermal wavelength:
\begin{align*}
    \lambda = \left(\frac{2\pi \hbar^2}{m k_B T} \right)^{\frac{1}{2}} 
\end{align*}
and:
\begin{align*}
    z \equiv \frac{e^{\beta \mu}}{\lambda^3} 
\end{align*}
Notice that in the previous chapter we were definining $z = e^{\beta \mu}$.

\begin{align}
    e^{\beta}
\end{align}

Space is discretized as a cubic lattice (or, more generally, a $d$-dimensional lattice) of $V$
We know consider the lattice gas approximation. Each site in the lattice is described by an integer $d$-dimensional vector

...pag.3...

Each cell can be occupied by at most one particle. If there is, $\sigma_x = 1$, otherwise $\sigma_x = -1$. The lattice spacing $a$ becomes the unit distance in the model. 

\begin{align*}
    V_N(\mathbb{Q}) = \sum_{i< j} v(q_{ij}) 
\end{align*}
When two particles are occupying nearest neighbour sites, they are interacting, and sit in the \textit{minimum} of the interaction potential. Let $V(a) = -\epsilon_0$. Then:
\begin{align*}
    V_N(\mathbb{Q}) = -\epsilon_0 \sum_{\langle x,y \rangle} \frac{1+\sigma_x}{2} \frac{1+\sigma_y}{2}  
\end{align*}
where $\langle x,y \rangle$ represents the sum over neighbouring elements, i.e. on pairs $(x,y)$ such that $|\bm{x}-\bm{y}| = a$. Note that if both $\sigma_x$ and $\sigma_y$ are $1$, then the product will be $1$. Otherwise, the result will be $0$.

In a finite system, sites at the boundaries have fewer neighbours than sites in the bulk, and so \q{interact less}, leading to \textbf{open} boundary conditions. Boundaries can be removed by applying \textbf{periodic} boundary conditions, i.e. connecting the different boundaries so that the particle at one side \textit{interacts} with the particles in the opposite side (as if the $d=1$ system were a \textit{ring}, or the $d=2$ a \textit{torus}). This leads to a system that is completely translationally invariant.   %topology
%TO DO: add pictures

...





\end{document}

