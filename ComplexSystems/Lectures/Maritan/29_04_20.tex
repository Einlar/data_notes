%&latex
%
\providecommand{\main}{../..}
\documentclass[../../main.tex]{subfiles}

\begin{document}

\section{Critical Behaviours and Scaling Laws}
\lesson{22}{29/04/20}

In the last section, we were able to finally describe a \textbf{phase-transition} , by analysing the Ising Model in the mean field approximation in $d > 1$. Mathematically, we observed how the \textbf{spontaneous magnetization} $M(K, h)$,\marginpar{Power laws} when $K$ is chosen in the proximity of the \textit{critical parameter} $K_c$ needed for the phase-transition, is described by a \textbf{power law} (\ref{eqn:mean-field-MS}).

\medskip

This happens to be a very general kind of behaviour, proper of \textbf{not only} \textbf{mean field} models.
Scaling laws such as (\ref{eqn:mean-field-MS}) were originally formulated from empirical evidence, and then given a theoretical foundation in the 1960s by Widom, Kadanoff and Kenneth Wilson, leading to the field of \textbf{renormalization group theory}. In this framework, all critical phenomena can be treated on equal ground, and general results can be mathematically proven. 

\medskip

The importance of scaling laws, and especially the values of their \textit{critical exponents} (such as $\beta$ for the IM) resides in their \textbf{universality}, i.e. in the fact that they are \textit{largely independent} on the \q{model's details}. In other words, the very same scaling law can describe two systems that - from the outside - seem completely different - but that share some fundamental characteristic (e.g. symmetry). 

\medskip

So, let's continue using the Ising Model in the mean field approximation as a \textit{concrete} example, and let's focus on deriving and understanding scaling laws.

\subsection{Spontaneous magnetization}

We start with (re)deriving the power law for the \textbf{spontaneous magnetization}. Recall the expression for the variational free energy (\ref{eqn:FV-uniform}):
\begin{align}
    \beta \frac{F_V(m,K,h)}{N} = - K d m^2 + \frac{1+m}{2} \ln \frac{1+m}{2} + \frac{1-m}{2} \ln \frac{1-m}{2} - hm   
\end{align}
$F_V$ is closest to the \textit{true} \q{unapproximated} free energy $F$ when it's minimum:
\begin{align}\label{eqn:minimizing-eq}
    \pdv{m} F_V(m, K, h) \overset{!}{=} 0 \underset{(\ref{eqn:uniform-variational-eq})}{\Rightarrow}  m(h,K) = \tanh(2d K m + h)
\end{align} 
Let's solve (\ref{eqn:minimizing-eq}) for $h$ and expand\footnote{For notational simplicity, we do not denote with $M$ the value of $m$ that solves (\ref{eqn:minimizing-eq}), as was instead done in the previous section.} for $m \sim 0$:
\begin{align}\nonumber
    h &= -2dKm + \tanh^{-1}m  \underset{(\ref{eqn:inv_tanh})}{=} -2dKm + \frac{1}{2} \ln \frac{1+m}{1-m} \underset{(\ref{eqn:tanh})}{=} -2dKm + \frac{1}{2}[\ln(1+m) - \ln(1-m)] =\\ \nonumber
    &= -2dKm +\frac{1}{2}\left[m-\cancel{\frac{m^2}{2}}+\frac{m^3}{3}-\bcancel{\frac{m^4}{4}}+\frac{m^5}{5}+\dots  - \left(-m -\cancel{\frac{m^2}{2}} -\frac{m^3}{3} -\bcancel{\frac{m^4}{4} }-\frac{m^5}{5} + \dots    \right) \right] =\\
    &= m(1-2dK) + \frac{m^3}{3} + \frac{m^5}{5} + \dots  \label{eqn:h-eq} 
\end{align}
When $\textcolor{Blue}{h=0}$\marginpar{1. $h=0$} and $K \geq K_c = 1/2d$, then\footnote{Consider fig. \ref{fig:phase-diagram-uniform}, pag. \pageref{fig:phase-diagram-uniform}. When $h=0$, we are \q{moving} along a horizontal line, encountering the singularity at $K = K_c$.} $1-2dK < 0$. We already know that the solution $m=0$ is a local \textit{maximum} of $F_V$, and the minima are given by the other \textit{two} solutions. Dividing by $m$ and rearranging leads to:
\begin{align*}
    (2dK-1) = \frac{m^2}{3} + \frac{m^4}{5} + \dots \Rightarrow  m^2 = 3(2dK - 1) - \frac{m^4}{5} + \dots
\end{align*}
If $(2dK - 1)$ is of order $O(m^2)$, then $m^4$ is of order $O[(2dK-1)^2]$, and so:
\begin{align*}
    m^2 = 3(2dK-1) + O[(2dK-1)^2]
\end{align*}
And substituting $K_c = 1/2d$:
\begin{align*}
    m^2 = 3 \frac{K-K_c}{K_c} + O\left(\left[\frac{K-K_c}{K_c} \right]^2\right) 
\end{align*}
Then, using $K=J/k_B T$ and $K_c = J/k_B T_c$ leads to the equivalent relation in terms of temperatures:
\begin{align*}
    m^2 = 3 \frac{T_c - T}{T_c} + O\left(\left[\frac{T_c - T}{T_c} \right]^2\right)  = 3 |t| + O(t^2) \qquad t \equiv \frac{T-T_c}{T_c} 
\end{align*} 
Taking the square root leads to the power law for the magnetization:
\begin{align}\label{eqn:m-power-law-h0}
    m = \sqrt{3} |t|^\beta \theta(-t) \qquad \beta = \frac{1}{2} 
\end{align}
Here, the Heaviside function $\theta(-t)$ ensures that $m=0$ for $T > T_c$.

\medskip

Let's now consider (\ref{eqn:h-eq}) in the case $\textcolor{Blue}{h \neq 0}$ \marginpar{2. $h\neq 0$}. To \q{see} the phase-transition, we fix\footnote{Referring to fig. \ref{fig:phase-diagram-uniform}, \pageref{fig:phase-diagram-uniform}, we are \q{moving} along a vertical line with $K = K_c$} any $K \geq K_c$, for example (and for simplicity) $K=K_c$. 
%TODO Do the calculation

Then (\ref{eqn:h-eq}) becomes:
\begin{align}\label{eqn:m-power-law-h}
    h = \frac{m^3}{3} + \frac{m^5}{5} + \dots \underset{m\sim 0}{=} \frac{m}{3} |m|^{\delta - 1} \qquad \delta=3
\end{align}
After collecting a $m$, all the powers are even, and so we can insert a modulus. The exponent of the leading order is then $\delta=3$.

\subsection{Susceptibility}
Near criticality, also the susceptibility $\chi$, measuring \q{how much} the system reacts to a change in the external field, obeys a power law.

\medskip

Recall that the susceptibility $\chi$ is defined as:
\begin{align*}
    \chi^{-1} \underset{(\ref{eqn:susceptibility})}{=} \pdv{h}{m}   
\end{align*}
Using the expression (\ref{eqn:h-eq}) for $h$ and expanding around $m = 0$ we get:
\begin{align*}
    \chi^{-1} = \pdv{h}{m} \underset{(\ref{eqn:h-eq})}{=}  -2dK + \frac{1}{1-m^2}  = 1 - 2dK + m^2 + O(m^4)
\end{align*}
And using (\ref{eqn:m-power-law-h0}) to compute $m^2$ we arrive to:
\begin{align*}
    \chi^{-1} \Big|_{h=0} \underset{(\ref{eqn:m-power-law-h0})}{=} \begin{cases}
        1 - 2 d K = \frac{K_c-K}{K_c} = \frac{T-T_c}{T} = t+ O(t^2) & T > T_c \> (m=0)\\  
        1 - 2 dK + m^2 = \frac{K_c-K}{K_c} + 3\frac{K-K_c}{K_c} = 2|t| + O(t^2) & T<T_c 
    \end{cases}
\end{align*}



\end{document}
