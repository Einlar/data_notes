%&latex
%
\documentclass[../Theoretical.tex]{subfiles}

\begin{document}

\section{Introduction}
\lesson{1}{30/9/2019}
Physics values experiments and setup, but also mathematical frameworks capable of grouping many different phenomena in a unifying explanation.\\
An example is given by quantities that remain \textit{constant} - that is \textit{invariant}.\\

In these lessons, we will start with an \textbf{introduction to QFT}, divided in the subsequent sections:
\begin{enumerate}
\item Basics of Group Theory\\
\textit{Costa-Giunti \q{Group Theory in particle physics} (or any other book)}
\item Constructions of QFT (for non-interacting particles/fields)\\
\textit{Check syllabus online for book recommendations + Greiner \q{Introduction to Quantum Field Theory} and his book about special relativity}
\item Interacting fields
\end{enumerate}

\chapter{Basics of Group Theory}
Symmetries are related to transformations, and the main framework to understand them is given by \textit{group theory}.

\begin{dfn}
\label{dfn:group}
A \textbf{group} $\{G, \cdot\}$ is a set (i.e. list of objects) with a \textbf{product} operation $\cdot\colon G \times G \to G$, $g_1 \cdot g_2 \mapsto g_3 = g_1 \cdot g_2$ with the following properties:
\begin{enumerate}
\item \textbf{Closure}: the product only leads to elements in $G$, that is if $g_1, g_2 \in G$, then also $g_1 \cdot g_2 \in G$
\item \textbf{Neutral element} (Identity). $\forall g \in G, \exists! e$ such that $e\cdot g= g\cdot e=g$, $e\in G$.
\item \textbf{Inverse element} $\forall g \in G, \exists g^{-1}$ such that $g\cdot g^{-1} = e = g^{-1}\cdot g, g^{-1}\in G$
\end{enumerate}
\end{dfn}

For example, the set of \textit{rotations} in a 2D plane, with the \textit{composition} operation, is a group: in fact the composition of two rotations is also a rotation, the rotation of $0$ is the neutral element, and for every rotation of angle $\theta$, the rotation $2\pi-\theta$ is its inverse element.

\begin{dfn}
The \textbf{commutator} of two variables is defined as:
\begin{align*}
[g_1, g_2] = g_1 \cdot g_2 - g_2 \cdot g_1
\end{align*}
Note that it is not necessary to define a \textit{minus} operation on $G$, as the commutator can be expressed in terms of only the product. For example, the commutator is $0$ if and only if $g_1 \cdot g_2 \cdot g_1^{-1} \cdot g_{2}^{-1} = e$.
\end{dfn}

\begin{dfn}
A group $\{G, \cdot\}$ is said to be \textbf{abelian} if it is commutative, that is if the following is true:
\begin{align*}
g_1 \cdot g_2 = g_2 \cdot g_1 \Leftrightarrow [g_1, g_2] = 0 \quad \forall g_1, g_2 \in G
\end{align*}
\end{dfn}

\begin{dfn}
A \textbf{subgroup} $H \leq G$ of a group $G$ is a subset of $G$ ($H \subseteq G$) for which all the properties of a group still hold, that is $\forall h_i \in H$, all the requirements of def. \ref{dfn:group} are true.
\end{dfn}

\begin{dfn}
The \textbf{center} of a group $G$ is every element that commutes with all other elements of $G$:
\begin{align*}
C(G) = \{g_i \in G | [g_i, h] = 0, \forall h \in G\}
\end{align*}
\end{dfn}

\textbf{Remark}. Note that $C(G) \leq G$, that is $C(G)$ is a subgroup of $G$. Also:
\begin{itemize}
\item $C(G)$ is abelian
\item If $G$ is a abelian, then $C(G)=G$
\end{itemize}

Some \textbf{examples} of groups are:
\begin{itemize}
\item The set of \textbf{integer numbers}: $\{\bb{Z},+\}$, which is abelian and discrete. Here $e=0$, and $m^{-1}=-m\quad \forall m \in \bb{Z}$
\item \textbf{Real numbers} $\{\bb{R},+\}$, abelian, $e=0$, $x^{-1}=-x\quad \forall x\in \bb{R}$
\item $\{\bb{R},\cdot\}$ is \textbf{not} a group, because it is not possible to define the inverse for every element. In fact $0^{-1}$ does not exist.
\item $\{\bb{R}\setminus \{0\},\cdot\}$ is a group
\item \textbf{Symmetric group} $\bb{S}_3$, set of \textit{permutations} of $3$ elements. This group can be pictured as three points in a plane, connected by clockwise arrows. Then its elements are:
\begin{align*}
\bb{S}_3 = \left\{\underbrace{123,231,312}_{\mathrm{even}},\underbrace{213,132,321}_{\mathrm{odd}}\right\}
\end{align*}
The elements that can be constructed \q{following the direction of the arrows} are said to be \textit{even} (123, 231, 312), and the others are \textit{odd} (213, 132, 321).\\
Another possibility is to define two transformations:
\begin{align*}
\rho = \{1\to 2, 2\to 3, 3\to 1\}; \quad \sigma=\{1\to 2,2\to 1, 3\to 3\}
\end{align*}
Then:
\begin{align*}
\bb{S}_3 = \{e,\rho, \rho^2, \sigma, \rho\sigma, \sigma\rho\}
\end{align*}
\end{itemize}
For example, imagine to apply $\rho$ to $e=123$, then $231$ is obtained, as before.


\begin{dfn}
\textbf{Linear group} $\op{GL}(N, \bb{C})$
\begin{align*}
\op{GL}(N,\bb{C}) = \{N\times N \text{ matrices, with complex elements, } \op{det}\neq 0\}\\
\op{GL}(N, \bb{R}) \subseteq \op{GL}(N,\bb{C}) 
\end{align*}
\end{dfn}

\begin{dfn}
\textbf{Special Linear Group} $\op{SL(N,\bb{C})}$: same requirements as $\op{GL}(N,\bb{C})$, with the added $\op{det} = 1$.
\begin{align*}
\op{SL}(N,\bb{C})\subseteq \op{GL}(N,\bb{C})
\end{align*}
\end{dfn}

\begin{dfn}
\textbf{Unitary group}
\begin{align*}
\op{U}(N,\bb{C}) = \{U\in \op{GL}(N,\bb{C}) | U^\dag U = UU^\dag = \bb{I}_{N\times N}
\end{align*}
\begin{align*}
\op{SU(N,\bb{C})} \subseteq \op{U}(N,\bb{C}), \quad \op{det}=1
\end{align*}
\end{dfn}

$U$ preserves the \textbf{complex scalar product}:
\begin{align*}
\langle z, w\rangle_{\bb{C}} = z^\dag w = z_1^* w_1 + \dots + z_N^* w_N \in \bb{C}, \quad z,w\in \bb{V}
\end{align*}
where $\bb{V}$ is a vector space.

\begin{dfn}
\textbf{Orthogonal Group} $\op{O}(N,\bb{R})$, defined as:
\begin{align*}
\op{O}(N,\bb{R}) = \{O \in \op{GL}(N, \bb{R}), O^T O = OO^T =\ \bb{I}_{N\times N}\} \subseteq \op{U}(N,\bb{C})
\end{align*}
As before, the \textit{special} orthogonal group has the same definition, but with $\op{det} = 1$. Then $\op{SO}(N,\bb{R}) \subseteq \op{SU}(N,\bb{C})$
\end{dfn}
$O$ preserves the \textbf{real scalar product}.

\begin{dfn}
\textbf{Lorentz group} (homogeneous)
%\begin{itemize}
\item $\op{O}(1,3)$: group of invertible matrices $\Lambda \in \op{GL}(4, \bb{R})$ which preserve the metric $g$:
\begin{align*}
\Lambda^T g \Lambda =\ g, \quad g=\begin{pmatrix}
1 & 0 & 0 & 0\\
0 & -1 & 0 & 0\\
0 & 0 & -1 & 0\\
0 & 0 & 0 & -1
\end{pmatrix}
\end{align*}
This is different from $\op{O}(4)$, because the \textit{time dimension} (1) is treated differently from the \textit{space dimensions} (3). Nonetheless, many properties of $\op{O}(4)$ are true for $\op{O}(1,3)$ - but the differences have an important physical meaning.
%\end{itemize}
\end{dfn}

\begin{dfn}
\textbf{Poincaré group} $\mathcal{P}$ (or $\mathcal{P}(1,3)$), \textit{non-homonegeous} lorentz group.\\
$\mathcal{P}$ is defined as the group of isometries in Minkowski space, that is of transformations:
\begin{align*}
p^\mu \mapsto \underbrace{\Lambda^\mu}_{\in \op{O}(1,3)} \lor p^\nu + \underbrace{b^\mu}_{\in\bb{C}^4}
\end{align*}
\end{dfn}

\begin{dfn}
\textbf{Euclidean Group} $E_N$, defined as the group of transformations:
\begin{align*}
\bb{R}^N \ni x \mapsto \underbrace{A}_{\in \op{O}(N,\bb{R})}x + \underbrace{b}_{\in \bb{R}^N}
\end{align*}
\end{dfn}
For example for $N=2$ we get $E_2$:
\begin{align*}
\begin{pmatrix}x\\y\end{pmatrix} \mapsto \begin{pmatrix}x'\\y'\end{pmatrix} =
\underbrace{\begin{pmatrix}\cos\theta & \sin\theta\\-\sin\theta & \cos\theta\end{pmatrix}}_{R(\theta)\in \op{SO}(2)}\begin{pmatrix}x\\ y\end{pmatrix} + \begin{pmatrix}b_1 \\ b_2\end{pmatrix}
\end{align*}
This affine transformation can be seen as a specific kind of matrix multiplication in a higher-dimensional space:
\begin{align*}
\begin{pmatrix}x\\y\\1\end{pmatrix}\mapsto \begin{pmatrix}x'\\y'\\1\end{pmatrix} = \begin{pmatrix}\bm{R(\theta)} & & b_1\\ & & b_2\\ 0 & 0 & 1\end{pmatrix}\begin{pmatrix}x\\y\\1\end{pmatrix}
\end{align*} %center and add separator lines
Let's denote an element of $E_2$ as $E_2(\vec{b}, \theta$), that is a translation of vector $\vec{b}$ along with a rotation of angle $\theta$. Then, using the notation just introduced, the composition of two rotations is given by:
\begin{align*}
E_2(\vec{b}, \theta_1) \cdot E_2(\vec{c}, \theta_2) = \begin{pmatrix}
\bm{R(\theta_1)+R(\theta_2)} & R(\theta_1) \cdot \begin{pmatrix}c_1\\c_2\end{pmatrix} + \begin{pmatrix}b_1\\b_2\end{pmatrix}\\
0\dots 0 & 1
\end{pmatrix} = E_2(R(\theta_1)\vec{c}+\vec{b}, \theta_1+\theta_2)
\end{align*}

\begin{dfn}
The \textbf{normal} (or \textbf{invariant}) subgroup $N \trianglelefteq G$,  where $N \subseteq G$, is the set of all elements of $N$ that can be transformed by elements of $G$ in other elements of $N$. That is, the following \q{remapping} relation holds:
\begin{align*}
\forall g \in G, h \in N\quad g\cdot h\cdot g^{-1} \in N
\end{align*}
With an abuse of notation, one can say that $gNg^{-1} = N$, that is $g$ \q{transforms} the set $N$ in itself.
\end{dfn}

\begin{itemize}
\item The \textbf{simple group} is one of the trivial normal group of a certain $G$, that is a group made of only the neutral element, or all the elements (identity) of $G$:
\begin{align*}
N \trianglelefteq G \Rightarrow N = \{e\} \text{ or } N=G
\end{align*}
\item A \textbf{semi-simple group} is a \textit{non-abelian} normal group, $\forall N \trianglelefteq G$, $N$ are non-abelian.
\end{itemize}

\begin{dfn}
Given two groups $G$, $G'$, one can define their \textbf{direct  product} as the group defined by:
\begin{align*}
G,G' \mapsto G\otimes G' = \{(\underbrace{g_1}_{\in\ G}, \underbrace{g'}_{\in G'}) \text{ with the operation } (g_1, g_1') \cdot (g_2, g_2') =\ (g_1\cdot g_2, g_1'\cdot g_2')\}
\end{align*}
\end{dfn}

Let's consider two subgroups $H_1, H_2 \leq G$.\\ Their direct product is the same as the larger group $G$, that is $G=H_1\otimes H_2$, if:
\begin{enumerate}
\item The two subgroups are \q{almost disjoint}, i.e. their interesection is only the neutral element:$H_1 \cap H_2 = e$
\item All elements of $H_1$ commute with all elements of $H_2$: $h_1 \in H_1, h_2 \in H_2$, $[h_1, h_2] = 0$
\item $G \ni g = h_1 \cdot h_2$
\end{enumerate}

\begin{dfn}
A semi-direct group $G=H_1 \osubseteq H_2$ if the 1st and 3rd properties as the previous definition are true, and then:
\begin{align*}
H_1 \trianglelefteq G \text{ is invariant}
\end{align*}
\end{dfn}
%sistemare qui alla fine

\end{document}


