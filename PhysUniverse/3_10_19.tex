%&latex
%
\documentclass[../PhysUniverse.tex]{subfiles}
\begin{document}

\section{Introduction}
\lesson{1}{3/10/2019}
The \textbf{Copernican Principle} states that we do not occupy a preferred position in the universe. In other words, the Earth is not special.\\

This is generalized by the \textbf{cosmological principle}:
\begin{center}
Every comoving observer sees the Universe around her/him at a \textbf{fixed time} as being \textbf{homogeneous} and \textbf{isotropic}.
\end{center}
\begin{itemize}
\item This is really true if we forget all about details, that is if we observe the universe at its largest scales
\item \textbf{Comoving}. The Cosmic Radiowave Background is a signal in the infrared, at temperature of about $\SI{3}{\kelvin}$ that permeates the universe. In particular, there is a \textit{dipole effect} - that is the CMB, as observed by Earth, appears hotter in a certain direction, and colder in the opposite one, by about $\SI{1}{\milli\kelvin}$. This can be explained by the Doppler effect, with a huge velocity of $\sim \SI{630}{\kilo\m\per\s}$. This cannot be accounted by the revolution motion of Earth around the Sun, or the Sun around the center of the galaxy - but it's generated by the attraction between \textit{clusters of galaxies}. The presence of this dipole component means that Earth is not a \textit{comoving observer}, for which the CMB would appear with anisotropies in the order of $\si{\micro\kelvin}$.\\
Note that the existance of preffered reference frames is not allowed in special relativity, but it is in \textit{general relativity}.
\item For \textbf{fixed time} we define an instant in the proper time of a comoving observer.
\item \textbf{Isotropy} means that the Universe \q{looks the same in every direction} (invariant to rotations). On the other hand, \textbf{homogeneity} is the phylosophical assumption - not verifiable - that the same results can be inferred in every other position in the universe (invariance to translations). This is true if Earth is really a typical planet - which holds only in approximation, as the existence of life requires a specific subset of all the conditions possible in the universe.
\end{itemize}

In special relativity, we use the Minkowski metric:
\begin{align*}
ds^2 = c^2 dt^2 -dx^2 - dy^2 -dz^2
\end{align*}
Which is a particular metric:
\begin{align*}
ds^2 = g_{ab} dx^a dx^b
\end{align*}
with the choice of:
\begin{align*}
g_{ab} =\eta_{ab} = \op{diag}([1,-1,-1,-1])
\end{align*}
Let $dl^2 = dx^2 + dy^2 + dz^2$, so that we can write $ds^2 = c^2dt^2 -dl^2$. Then, we can transform it to spherical coordinates:
\begin{align*}
dl^2 = d\rho^2 + \rho^2 d\Omega^2 \qquad d\Omega^2 = d\theta^2 + \sin^2\theta d\varphi^2; \quad \rho \in \bb{R}_+, \theta \in [0,\pi), \varphi \in [0,2\pi)
\end{align*}
This is one of the building blocks of the Roberston-Walker metric, which is the metric \textit{as seen} by a comoving observer - useful in cosmology:
\begin{align*}
ds^2 = c^2 dt^2 - a^2(t) \left[\frac{dr^2}{1-kr^2}+r^2d\Omega^2\right]
\end{align*}
where $a(t)$ is the \textit{scale factor} (it has units of length,$[a(t)] = \mathscr{L}$), a function of the proper time in a comoving frame of reference, and $r=\rho/a(t)$ is an adimensional variable.\\
Depending on the value of $k \in [-1,1]$ (\textbf{curvature} parameter), this metric can describe different \q{kinds} of universes.\\

Minkowski is a maximally symmetric spacetime, with $10$ degrees of freedom. It is not the only spacetime with this property in $d=2$: the same property holds for de-Sitter and anti-de-Sitter spacetimes.\\

In our universe, however, we have less simmetries, as the universe evolves with time (it has not time invariance), and also the Lorentz boost simmetries are broken. This means that only $6$ simmetries are possible. In these conditions, there is only one possible metric that preserves all of them, and that is the Robertson-Walker metric previously introduced.\\

%consider plane, sphere, hyperboloid => analogy with R-W
If $k=0$ in $d=2$ (for simplicity):
\begin{align*}
dl^2 = a^2(dr^2 + r^2 d\theta^2)
\end{align*}
With $k=+1$ we obtain a sphere:
\begin{align*}
dl^2 = a^2(d\theta^2 + \sin^2\theta d\varphi^2)
\end{align*}
where $a=R$, the sphere radius.\\
Substituting $r\equiv \sin\theta$ (and $dr =\cos\theta \,d\theta$) we arrive in fact at:
\begin{align*}
dl^2 = a^2 \left(\frac{dr^2}{1-r^2} + r^2 d\varphi^2\right)
\end{align*}

With $k=-1$ the only change is $\sin\to\sinh$:
\begin{align*}
dl^2 = a^2(d\theta^2 + \sinh^2 \theta d\varphi^2)
\end{align*}
In analogy as before, letting $r\equiv \sinh\theta$ leads to:
\begin{align*}
dl^2 = a^2 \left(\frac{dr^2}{1+r^2}+r^2d\varphi^2\right)
\end{align*}

Starting from R-W, by setting $k$ to different values we can arrive to the metric of a plane ($k=0$), a sphere ($k=+1$) or a hyperboloid ($k=-1$):
\begin{align*}
ds^2 &= cdt^2 -a^2(t) \left(\frac{dr^2}{1-kr^2} + r^2 d\varphi^2\right) =\\
&= c^2 dt^2 - a^2 \begin{cases}
d\chi^2 + \sin^2 \chi d\Omega^2 & k=+1, \Omega = \sin \chi\\
d\chi^2 + \chi^2 d\Omega^2 & k=0, r=\chi\\
d\chi^2 + \sinh^2 \chi d\Omega^2 & k=-1,\Omega = \sinh\chi
\end{cases}
\end{align*}
\begin{align*}
ds^2 &= c^2 dt^2 - a^2(t) \left(1+ \frac{k|\chi|^2}{4}\right)^{-2}(dx^2+dy^2+dz^2)\\
\end{align*} [?]\\

It is possible to simplify the R-W metric by a substitution, passing to \textbf{conformal time} $dt \equiv a(\eta)\eta$, $a(\eta)\equiv a(t(\eta))$, $\int d\eta = \int dt/a(t)$
[...]\\


\subsection{Friedman equations}
The evolution of a universe for which the cosmological principle holds is described by the Friedman equations, where a time derivative (denoted with a dot - like in $\dot{a}$) is interpreted with respect to the proper time of a comoving observer.\\

\begin{align*}
\dot{a}^2 &= \frac{8\pi G}{3}\rho a^2 - kc^2\\
\ddot{a} &= -\frac{4\pi G}{3}\left(\rho + 3\frac{P}{c^2}\right)\\
\dot{\rho} &= -3\frac{\dot{a}}{a}\left(\rho + \frac{P}{c^2}\right)
\end{align*}
where $\rho=\rho(t)$ is the \textit{energy density}, and $P=P(t)$ is the \textit{isotropic pressure}.\\
We define the \textbf{Hubble parameter} as:
\begin{align*}
H(t) \equiv \frac{\dot{a}}{a}
\end{align*}
For a flat universe, we have:
\begin{align*}
H^2 = \frac{8\pi G}{3}\rho - \frac{kc^2}{a^2}
\end{align*}
\begin{align*}
k=a \Rightarrow \rho_c(t) \equiv \frac{3H^2(t)}{8\pi G}
\end{align*}
Then we define:
\begin{align*}
\Omega(t) = \frac{\rho(t)}{\rho_c(t)} = \frac{8\pi G \rho(t)}{3H^2(t)}
\end{align*}
called the \textbf{density parameter}.
\begin{itemize}
\item $\Omega > 1\Leftrightarrow k=1$
\item $\Omega = 1 \Leftrightarrow k=0$
\item $\Omega < 1 \Leftrightarrow k=-1$
\end{itemize}
So, by measuring the density parameter, we can infer properties of the geometry of spacetime.\\

$H_0 = H(t_0)$, with $t_0$ meaning the current proper time, is the Hubble constant (current spacetime expansion rate), and has been measured:
\begin{align*}
H_0 = 100h  \frac{\si{\km}/\si{s}}{\si{\mega pc}} =  70 \frac{\si{\kilo\m}/\si{\s}}{\si{\mega pc}}
\end{align*}
where $h=0.7$.\\
All current measurements suggest that $\Omega_0 \approx 1$, which means a \textit{flat universe}. This is due to the presence of three main components: visible matter, dark matter and dark energy. Unfortunately, $95\%$ of of the universe is made of the unobservable (\q{dark}) parts.\\

Suppose that we want to measure the current energy density in galaxies $\rho_{0g}$. This is related to the mean intrinsic luminosity of galaxies per unit volume $\mathcal{L}_g$ by:
\begin{align*}
\rho_{0g} = \mathcal{L}_g \langle \frac{M}{L}\rangle
\end{align*}
where $\langle M/L\rangle$ is the mean mass to light ratio of galaxies (usually expressed in solar units, where $M_\odot = \SI{1.99e33}{\g}$ is the mass of the sun, and $L_\odot = \SI{3.9e33}{erg \per \s} = \SI{3.9e26}{\watt}$.\\
Measuring $\mathcal{L}_g$ is not easy, as even all the weakest sources need to be considered. One trick to help with that is given by the definition of expected value of a continuous variable:
\begin{align*}
\mathcal{L}_g = \int_0^{+\infty} \mathrm{dL}\, L \phi(L)
\end{align*}
where is the luminosity function, that is the number of objects per unit volume with luminosity between $L$ and $L+dL$ (in analogy with a probability density).\\
$\phi(L)$ is somewhat easier to determine, and can be extrapolated from observations.

\end{document}


