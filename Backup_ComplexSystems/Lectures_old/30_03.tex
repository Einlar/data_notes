%&latex
%
\documentclass[../template.tex]{subfiles}

\begin{document}

\section{Last Thermodynamics}
Let's return to the biatomic case, where we do not fix $r_0 = |\bm{r}|$ anymore, but consider the two atoms as kept together by a potential $u(\bm{r})$, with a minimum at $r_0$, going to $\infty$ for $r \to 0$, and to $0$ for $r \to +\infty$. If we expand around the minimum:
\begin{align*}
    u(r) = u(r_0) + (r-r_0)^2 \frac{u''(r_0)}{2} + \dots 
\end{align*}

This leads to the Hamiltonian:
\begin{align*}
    H(\mathbb{Q},\mathbb{P}) = \sum_{i=1}^N \left[\frac{\norm{\bm{p}_{\mathrm{CM},i}}^2}{2M} + \frac{1}{2I} \left(p_{\theta,i}^2 + \frac{p_{\varphi,i}^2}{\sin^2 \theta_i} \right)  + \frac{p_{r,i}^2}{2m} + \frac{u''(r_0)}{2} (r_i-r_0)^2   \right]
\end{align*}
Compared to the previous case, we have two quadratic terms more! So, applying the equipartition theorem:
\begin{align*}
    \langle H \rangle = N\left[\frac{3}{2} k_B T + 2 \frac{k_B T}{2} +2 \frac{k_B T}{2}   \right] \Rightarrow c_v^{(m)} = R\left[\frac{3}{2} + 1 + 1 \right] = \frac{7}{2} R 
\end{align*}
One can continue and consider also the degrees of freedom associated to electrons and nucleons (protons and neutrons).

Then realizing that the nucleus is made by quarks, it seems we can continue forever. Each degree of freedom contributes to $c_v^{(m)}$ with extra $R/2$ terms. However, this does not lead to a diverging molar specific heat, because each new degree of freedom can be accessed only over a certain \textit{energy scale}. This can be experimentally verified, for example for hydrogen H$_2$. At low temperature, it behaves like a \textit{monoatomic gas}. At ambient temperature it behaves as a biatomic gas, and over $\SI{5000}{\K}$ we need also to account for the vibration of the binding length.

\medskip

The explanation for this kind of \q{hierarchical} behaviour is given by \textbf{quantum mechanics}. The total energy of a biatomic molecule can be re-written as:
\begin{align*}
    \mathcal{E} = \frac{1}{2} M \norm{\bm{v}_{\mathrm{CM} }}^2 + \frac{L^2}{2 I} + \frac{m}{2} \norm{\bm{\dot{r}}}^2 + \frac{m \omega^2}{2} (\bm{r} - \bm{r_0})^2   
\end{align*} 
where we have defined $u''(\bm{r_0}) \equiv m \omega^2$. We have also used that:
\begin{align*}
    \frac{m \bm{v}^2}{2} = \frac{m \bm{\dot{r}}^2}{2} + \frac{L^2}{2 I}   
\end{align*}
where $\bm{L}$ is the angular momentum wrt the centre of mass:
\begin{align*}
    \bm{L} = \bm{r_1} \times (m_1 \bm{v_1}) + \bm{r_2} \times (m_2 \bm{v_2}) + \dots = m \bm{r} \times \bm{v}
\end{align*}
In quantum mechanics:
\begin{itemize}
    \item $\frac{L^2}{2I}$ can assume only discrete values $(\hslash = h/2\pi$):
    \begin{align*}
        \frac{1}{2 I} \hslash^2 l(l+1) \qquad l = 0,1,2,\dots 
    \end{align*} 
    with degeneracy $2l+1$.
    \item $\displaystyle \frac{m}{2} \dot{\bm{r}}^2 + \frac{m \omega^2}{2} (\bm{r}-\bm{r_0})^2$ can assume only discrete values:  
    \begin{align*}
        \hslash \omega \left(n + \frac{1}{2} \right) \qquad n=0,1,2,\dots
    \end{align*}
    \item $\frac{1}{2} M \bm{V_{\mathrm{CM}}}^2$ can assume any positive value for macroscopically large volume. 
\end{itemize}
So, the \textit{hierarchy} of degrees of freedom is explained by the fact that while energy can vary essentially in a continuum, rotational and vibrational energies can only assume a set of fixed discrete values. In particular, the vibrational states have higher energy than the rotational ones, and so they appear only at a higher energy scale.

\medskip

If we account for all these kind of energies in the partition function, we get:
\begin{align*}
    Z_{\mathrm{QS} } = \frac{1}{N! h^{7N}} (Z_{\mathrm{trasl} } \cdot Z_{\mathrm{rot}} \dot Z_{\mathrm{vib}})
\end{align*}
with:
\begin{align*}
    Z_{\mathrm{trasl}} &= \int \dd[3]{\bm{p}} \exp\left(-\frac{\beta \bm{p}^2}{2m} \right) = (M k_B T 2 \pi)^{3/2}\\
    Z_{\mathrm{rot}} &= \sum_{l=0}^{\infty} (2l+1) \exp\left(-\beta \frac{\hslash^2}{2 I} l(l+1) \right) = \sum_{l=0}^\infty (2l+1) \exp\left(-\frac{T_{\mathrm{rot}}}{T} l(l+1) \right) \qquad T_{\mathrm{rot}} = \frac{\hslash^2}{2 k_B I}\\
    Z_{\mathrm{vib}} &= \sum_{n=0}^\infty \exp\left(-\beta \hslash \omega \left(n + \frac{1}{2} \right)\right)  = \sum_{n = 0}^\infty \exp\left(-\frac{T_{\mathrm{vib} }}{T} \left(n + \frac{1}{2} \right) \right) \qquad T_{\mathrm{vib} } = \frac{\hslash \omega}{k_B} 
\end{align*}

For the vibrational energy we have just a geometric series, which evaluates to:
\begin{align*}
    Z_{\mathrm{vib}} = \sum_{n=0}^\infty \exp\left(-\frac{T_{\mathrm{vib} }}{T} \left(n+\frac{1}{2} \right) \right) = \exp\left(-\frac{T_{\mathrm{vib} }}{2 T} \right) \frac{1}{1 - e^{-T_{\mathrm{vib} }/T}} 
\end{align*}
Then the average vibrational energy per molecule is:
\begin{align*}
    \frac{\langle E_{\mathrm{vib} } \rangle}{N} = -\pdv{\beta} \ln Z_{\mathrm{vib} } = \frac{1}{2} k_B T_{\mathrm{vib} } + \frac{k_B T_{\mathrm{vib} }}{e^{T_{\mathrm{vib} }/T} - 1} = \frac{\hslash \omega}{2}   +
\end{align*}

And at the end we arrive to:


\medskip

When $T \gg T_{\mathrm{vib}}$, we recover the classical result, i.e. the last term in eq. (68). We can see this without directly evaluating the geometric sum, by observing that:
\begin{align*}
    \sum_{n=0}^\infty \exp\left(-\frac{T_{\mathrm{vib} }}{T} n \right) = \sum_{n=0}^\infty e^{-x_n} \qquad x_n = \frac{T_{\mathrm{vib} }}{T} n 
\end{align*}

...


We can use a similar argument for $Z_{\mathrm{rot}}$, which cannot be directly computed in general:
\begin{align*}
    Z_{\mathrm{rot}} &= \sum_{l=0}^{\infty} (2l+1) \exp\left(-\frac{T_{\mathrm{rot} }}{T} l(l+1) \right) \approx\\
    &\approx \begin{cases}
        1 + 3 \exp\left(-\frac{T_{\mathrm{rot} }}{T} 2 \right) + O\left(\exp\left(-\frac{T_{\mathrm{rot} }}{T} 6\right)\right) & T \ll T_{\mathrm{rot}}\\
        \sqrt{\frac{T}{T_{\mathrm{rot}}} } \sum_{l=0}^{\infty} \Delta x_l \exp\left({-x_l^2} - x_l \sqrt{\frac{T_{\mathrm{rot} }}{T} }\right) \approx \frac{2T}{T_{\mathrm{rot}}} \int_0^\infty \dd{x} e^{-x^2} = 2 \sqrt{\pi} T/T_{\mathrm{rot}} 
    \end{cases}
\end{align*}
where:
\begin{align*}
    k
\end{align*}

\section{Quantum stuff}

\section{Black Body Radiation}
A \textbf{black body} is something able to absorb any kind of incoming radiation, without reflecting anything back. However, note that this requirement does not prevent it to \textbf{emit} radiation by itself.

\medskip

The simplest model for a black body is given by a \textbf{cavity} with a small hole, filled with standing electromagnetic waves in equilibrium at temperature $T$. 

\medskip

The energy density inside the cavity corresponding to electromagnetic waves with frequencies in $(\nu, \nu+\dd{\nu})$ is given by:
\begin{align*}
    u_{\nu}(\nu, T) \dd{\nu} \underset{\mathclap{\nu = c/\lambda}}{=}  u_\nu \left(\frac{c}{\lambda}, T \right) \left|\dv{\nu}{\lambda}\right|\dd{\lambda} = u_{\nu} \left(\frac{c}{\lambda}, T \right) \frac{c}{\lambda^2} \dd{\lambda} \equiv u_{\lambda}(\lambda, T) \dd{\lambda} 
\end{align*}
In classical electromagnetism, an electromagnetic wave of frequency $\nu$ behaves like an harmonic oscillator where the electric and magnetic field behave like the kinetic and potential energy. So, we have two quadratic terms, contributing (through the equipartition theorem) to:
\begin{align*}
    \langle \epsilon_\nu \rangle = \underbrace{\frac{k_B T}{2}}_{\mathclap{\parbox{5em}{\footnotesize\centering Average kinetic energy}}}  + \underbrace{\frac{k_B T}{2}}_{\mathclap{\parbox{5em}{\footnotesize\centering Average potential energy}}}   = k_B T  
\end{align*}


\end{document}
