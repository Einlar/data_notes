%&latex
%
\documentclass[../template.tex]{subfiles}

\begin{document}

%Dal clustering in poi

The clustering coefficient represents the probability that any two nodes connected to a certain node $A$ are connected to each other (e.g. two friends of $A$ are friends to each other).

\medskip

Let $l_{ij}$ be the fraction of edges in a network that connect a vertex of type $i$ to one of type $j$. For an undirected network $l_{ij} = l_{ji}$. Then denote with $a_i$ the fraction of each type of end of an edge that are attached \textbf{to} vertices of type $i$, and similarly $b_i$ for the \textbf{from} connections. In formulas:
\begin{align*}
    a_i = \sum_j l_{ij} \quad b_j = \sum_i l_{ij}
\end{align*} 

%Add some graphics

The assortativity coefficient is then:
\begin{align*}
    r = \frac{ \sum_i l_{i i} - \sum_{i} a_i b_i}{ 1 - \sum_{i} a_i b_i}
\end{align*}
In matrix form, if $E$ is the matrix with entries $l_{ij}$, then:
\begin{align*}
    r = \frac{\operatorname{Tr}(E) - \norm{E^2} }{1 - \norm{E^2}} 
\end{align*}
where $\norm{E} = \sum_{ij} E_{ij}$.

\medskip

If $r=0$, there is no assortative mixing, and so $l_{ij} = a_i b_j$. If $r=1$ the matrix is perfectly assortative. 

\medskip

There various choice for the \textit{labelling} of nodes. One possibility is:
\begin{align*}
    \mu = \frac{\sum_{ij} A_{ij} x_i}{\sum_{ij} A_{ij}} = \frac{1}{2L} \sum_i x_i k_i 
\end{align*} 
leading to the assortativity coefficient:
\begin{align*}
    \rho(x_i, x_j) = \frac{1}{2m} \sum_{ij} \left(A_{ij} - \frac{k_i k_j}{2m} \right) x_i x_j
\end{align*}
where $2m$ is the total number of links. We can take for example $x_i = k_i$, leading to:
\begin{align*}
    \rho = ... 
\end{align*}

To be completed

\section{Random networks}

\subsection{Erdos-Renyi Random Graph}
Let $p$ be the probability that two nodes are connected. Then the degree $k$ of any node in a network of $N$ nodes follows a Bernoulli distribution:
\begin{align*}
    p(k) = {N-1 \choose k} p^k (1-p)^{N-1-k}
\end{align*}
This follows from the following facts:
\begin{enumerate}
    \item Each node is connected with any other $N-1$ vertices with equal and independent probability $p$
    \item For a node there are ${N-1 \choose k}$ ways to choose other $k$ vertices with which to connect. 
    \item The probability to be connected with a particular group of $k$ nodes is $p^k(1-p)^{N-1-k}$
\end{enumerate}
For $N \to \infty$, $p(k)$ tends to a Poisson distribution:
\begin{align*}
    p(k) = e^{-\langle k \rangle} \frac{\langle k \rangle^k}{k!} 
\end{align*}
where $\langle k \rangle$ is the average node degree:
\begin{align*}
    \langle k \rangle = p(N-1)
\end{align*}
The clustering coefficient, i.e. the probability that two nodes are interconnected if they share a common connection with a node $A$, is just $p$ - because all connections are independent:
\begin{align*}
    c = p = \frac{\langle k \rangle}{N-1} \xrightarrow[]{N \gg 1} 0 
\end{align*}

For E-R newtorks, the graph diameter scales as $\log N$ (and this can be proved analytically).




\end{document}