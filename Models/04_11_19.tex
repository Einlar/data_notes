%&latex
%
\documentclass[../template.tex]{subfiles}
\begin{document}
\section{Continuity of Brownian Path}
\lesson{6}{04/11/19}
Consider a particle starting in $x=0$ at $t=0$, and traversing $N$ points $\{x_i\}_{i=1,\dots,N}$ such that all increments $\Delta x_i = x_i - x_{i-1}$ are \textit{independent} and described by a \textit{gaussian pdf}. The density function for such a trajectory $\{x_i\}$ is the usual product of transition probabilities:
\begin{align}
    \dd{\mathbb{P}}_{t_1, \dots, t_N} (x_1, \dots, x_N) = \left(\prod_{i=1}^N
    \frac{\dd{x_i}}{\sqrt{4 \pi \Delta t_i D}}  \right) \exp\left(-\sum_{i=1}^N  \frac{(\Delta x_i)^2}{4 D \Delta t_i} \right) \quad \substack{\Delta t_i = t_i-t_{i-1}\\ \Delta x_i = x_i - x_{i-1}}
    \label{eqn:dP}
\end{align}
We now show that, taking the continuum limit $\max_i \Delta t_i \to 0$ leads to paths $\{x(\tau)\}$ that are \textit{almost surely continuous}. In other words, for any interval $T \subseteq \mathbb{R}$, the subset $N \subset \mathbb{R}^T$ of functions that are discontinuous has $0$ Wiener measure. 

Mathematically, we want to show that, as $\Delta t_i \to 0$, the probability that $\Delta x_i$ is close to $0$ approaches certainty:   
\begin{align*}
    \lim_{\Delta t_i \to 0} \mathbb{P}(|\Delta x_i| < \epsilon) = 1 \quad \forall \epsilon > 0
\end{align*}   
This is just the probability that, during time $\Delta t_i$, the particle makes a jump of size lower than $\epsilon$:
\begin{align*}
    \mathbb{P}(|\Delta x_i| < \epsilon) &= \mathbb{P}(x_{i-1} - \epsilon< x_{i} < x_{i-1}+\epsilon|x(t_{i-1}) = x_i) =\\
    &= \int_{x_{i-1}-\epsilon}^{x_{i-1}+\epsilon} \frac{\dd{x_i}}{\sqrt{4 \pi D \Delta t_i}} \exp\left(-\frac{(x_i - x_{i-1})^2}{4 D \Delta t_i} \right) =\\
    &\underset{(a)}{=}  \int_{-\epsilon}^{+\epsilon} \frac{\dd{\Delta x_i}}{\sqrt{4 \pi D \Delta t_i}} \exp\left(-\frac{(\Delta x_i)^2}{4 D \Delta t_i} \right)
\end{align*}
where in (a) we translated the variable of integration $\Delta x_i = x_{i}- x_{i-1}$.\\
With another change of variables:
\begin{align*}
    \frac{(\Delta x_i)^2}{\Delta t_i} = z^2 \Rightarrow z = \frac{\Delta x_i}{\sqrt{ \Delta t_i}} \Rightarrow \dd{\Delta x_i} = \dd{z} \sqrt{\Delta t_i}  
\end{align*}
we get:
\begin{align*}
    \mathbb{P}(|\Delta x_i| < \epsilon) = \int_{|z| < \epsilon/\sqrt{\Delta t_i }} \frac{\dd{z}\cancel{\sqrt{\Delta t_i}}}{\sqrt{4 \pi D \cancel{\Delta t_i}}}  \exp\left(-\frac{z^2}{4D} \right)
\end{align*}
And taking the continuum limit leads to:
\begin{align*}
    \lim_{\Delta t_i \to 0} \mathbb{P}(|\Delta x_i| < \epsilon) = \int_{-\infty}^{+\infty} \frac{\dd{z}}{\sqrt{4 \pi D}} \exp\left(-\frac{z^2}{4D} \right) =1
\end{align*}

 

\section{Differentiability of Brownian Path}
With a very similar calculation (here omitted) we can also show that:
\begin{align*}
    \lim_{\Delta t_i \downarrow 0} \left( \left|\frac{\Delta x_i}{\Delta t_i}\right| > k \right) = 1 \quad \forall k > 0
\end{align*}
meaning that Brownian paths are \textit{almost surely everywhere non-differentiable}.

\medskip

Nonetheless, it is sometimes useful to consider \q{formal derivatives} of a Brownian path, that acquire a definite meaning only when considering a \textit{finite discretization}. For example, we can start from (\ref{eqn:dP}) and rewrite it as:
\begin{align*}
    d\mathbb{P}_{t_1, \dots, t_N} (x_1, \dots, x_N) = \left(\prod_{i=1}^N \frac{\dd{x_i}}{\sqrt{4 \pi \Delta t_i }}\right) \exp\left(-\frac{1}{4 D} \sum_{i=1}^N \textcolor{Red}{\Delta t_i} \left(\frac{\Delta x_i}{\textcolor{Red}{\Delta t_i}} \right)^2 \right) 
\end{align*}
Then, in the continuum limit $\Delta t_i \to 0$, the sum in the exponential argument becomes a Riemann integral:
\begin{align*}
    \sum_{i=1}^N \Delta t_i \left(\frac{\Delta x_i}{\Delta t_i} \right)^2 \xrightarrow[\Delta t \to 0]{} \int_0^t \dd{\tau} \underbrace{\left(\frac{\dd{x_i}}{\dd{\tau}} \right)^2}_{\dot{x}^2(\tau)}  \qquad t= t_N
\end{align*} 
where $t = t_N$. Substituting it back leads to:
\begin{align*}
    \dd{x_w}(\tau) = \prod_{\tau = 0^+}^t \frac{\dd{x}(\tau)}{\sqrt{4 \pi D \dd{\tau}}} \exp\left(-\frac{1}{4 D} \int_0^t \dot{x}^2 (\tau) \dd{\tau} \right) 
\end{align*}
This expression has no rigorous meaning in this form ($\dot{x}(\tau)$  \textit{does not} exists!) but can be \textit{formally manipulated} into other expressions that \textit{have} a definite meaning, thus proving useful for the discussion.  

\section{Forces on the particle}
Let's return to the beginning. We started with a particle capable of moving in discrete steps (with probabilities $P_+$ and $P_-$ of going to the right or to the left), leading to the Master Equation, then to the Diffusion Equation and finally to the Path Integral.\\
In particular, we arrived at:
\begin{align*}
    \pdv{t} w(\bm{x}, t | \bm{x_0}, t_0) = \nabla^2 w(\bm{\bar{x}}, t| \bm{\bar{x}_0}, t_0)
\end{align*}
We want now to generalize this expression to the presence of \textit{forces} acting on the particle. We recall the usual discretization $x_i = i \cdot l$, $t_n = n \cdot \epsilon$. In general, the probability that the particle will be at position $i$ at time $t_{n+1}$ depends only on the states (probabilities) of the previous time step $t_n$:
\begin{align*}
    w_{i}(t_{n+1}) = \sum_j W_{ij}(t_n) w_j(t_n)
\end{align*}     
where $W_j(t_n)$ is the probability of the particle being at $j$ at $t_n$, and $W_{ij}(t_n)$ is the probability of jumping from $j$ to $i$ at $t_n$ (transition probability).\\
In the first lecture we assumed that:
\begin{align*}
    w_{ij}(t_n) = \delta_{j,i-1} P_+ + \delta_{j,i+1} P_-
\end{align*}        
i.e. the particle only jumps from adjacent positions, one step at a time, and cannot remain at the same place.\\
Now we drop that assumption, leading to:
\begin{align*}
    w(x, t_{n+1})\dd{x} = \int \dd{z} W(z | x-z, t_n) w(x-z, t_n) \dd{x}
\end{align*}
i.e. the particle can make jumps of \textit{any} size, and is not restricted to the discretization. The integrand is just the probability for a particle starting from $[x,x+\dd{x}]$ to make a jump of size $z$ and arriving at $[x-z, x+\dd{x}-z]$.\\
In the discrete case we previously considered, the condition $\sum_i W_{ij} (t_n) = 1$ (along with the master equation $w_i(t_{n+1}) = \sum_j W_{ij} (t_n) w_j(t_n)$) implied the conservation of probability $\sum_i w_i(t_{n+1}) = \sum_i w_i(t_n) = \dots = \sum_i w_i(0) = 1$.\\
Now, in the more general case, we have:
\begin{align*}
    w(x,t_{n+1}) &= \int \dd{z} W(z | x-z, t_n) w(x-z, t_n) = \\
    &= \int \dd{y} \int \dd{z} W(z|y,t_n) w(y, t_n) = \int \dd{y} w(y,t_n)
\end{align*}
with $y = x-z$ only if:
\begin{align*}
    \int \dd{z} W(z|y,t) = 1 \quad \forall y, \forall t \geq 0
\end{align*} 
We now consider the continuum limit in time:
\begin{align*}
    w(x,t_{n+1}) - w(x,t_n) = \int \dd{z} W(z| x-z, t_n) w(x-z, t_n) - w(x,t_n)
\end{align*}
Multiplying by $1$:
\begin{align*}
    w(x, t_{n+1} ) - w(x,t_n) &= \int \dd{z} W(z| x-z, t_n) w(x-z, t_n) - \textcolor{Blue}{\int \dd{z} W(z| x, t_n)} w(x,t_n) =\\
    &= \int \dd{z} \left[\underbrace{W(z| x-z,t_n) w(x-z, t_n) }_{F_z(x-z)} - \underbrace{W(z|x,t_n) w(x,t_n)}_{F_z(x)} \right] =\\
    &= F_z(x) - z \pdv{x} F_z(x) + \frac{z^2}{2} \pdv[2]{x} F_z(x) + \dots =\\
    &= -\int \dd{z} z \pdv{x} \left[W(z| x, t_n) w(x,t_n)\right] + \frac{1}{2} \int \dd{z} z^2 \pdv[2]{x} \left[W(z|x,t_n)w(x,t_n)\right] + \dots 
\end{align*}
We then consider a sort of $k$-th moment of the distribution: 
\begin{align*}
    \mu_k(x,t) = \int \dd{z} z^k W(z|x,t)
\end{align*}
so that $\mu_0(x,t) = 1$ due to the normalization. Returning to the previous expression:
\begin{align*}
    w(x,t_{n+1}) - w(x,t_n) &= \sum_{k=1}^\infty \frac{(-1)^k}{k!} \pdv{k}[\mu_k(x,t) w(x,t_n)] =\\
&= \pdv{x} \left\{
\sum_{k=1}^\infty  (-1)^k  \pdv[k-1]{x} [\mu_k(x,t) w(x,t_n)] 
\right\} = -\pdv{x}J(x,t_n) \\
\end{align*} 
where the right side can be interpreted as the flux $J$.\\ 
If we integrate over $x$, the left side is equal to $0$ (due to the normalization):
\begin{align*}
    0 = \sum_{k\geq 1} (-1)^k \int_{-\infty}^{+\infty} \left(\pdv{x}\right)^k [\mu_k(x,t) w(x,t_n)]
\end{align*}  
So, if the flux at the boundaries is $0$, then the probability is conserved (e.g. if the system is closed).

If we now divide by the time interval (to get the derivative):
\begin{align*}
    \frac{w(x,t_{n+1})- w(x,t_n)}{t_{n+1}-t_n} = \pdv{x} \left\{
    \sum_{k=1}^\infty \frac{(-1)^k}{k!}  \pdv[k-1]{x} \frac{\mu_k(x,t_n) w(x,t_n)}{t_{n+1}-t_n} 
    \right\} 
\end{align*}
Letting $t_{n+1}-t_n=\epsilon$, in the limit $\epsilon \to 0$  the left side will be $\dot{w}(x,t)$. Now, recall that for the pdf:
\begin{align*}
    \frac{1}{\sqrt{4 \pi D \epsilon}} \exp\left(-\frac{(\Delta x)^2}{4 D \epsilon} \right)
\end{align*}
we have $\langle  \Delta x \rangle = 0$ and $\langle (\Delta x)^2 \rangle = 2 D \epsilon$, with $|\Delta x| \approx \sqrt{\epsilon}$.\\
If the particle is subject to a force, then it will move in a \textit{preferred direction}. So we want:
\begin{align*}
    \langle  z \rangle = \int z W(z|x,t) \propto \epsilon f(x)
\end{align*}   
where $f(x)$ is the force (e.g. $f(x) = mg$ for gravity). In fact, in a constant regime, the velocity of a particle is constant (when friction and the force balance out).\\
And then, the variance of $z$ should be:
\begin{align*}
    \langle (z- \langle  z \rangle)^2 \rangle = \operatorname{Var}(z) \propto \epsilon  
\end{align*}  
So we make an \textit{ansatz}:
\begin{align*}
    W(z|x,t) = F\left(\frac{z- \epsilon f(x,t)}{\sqrt{\epsilon \hat{D}(x,t)}} \right) \frac{1}{\sqrt{\epsilon\hat{D}(x,t) }} 
\end{align*} 
Let's see why this guess is good. First, we integrate over $z$ to check the normalization:
\begin{align*}
    1 &\overset{!}{=}  \int_{-\infty}^{+\infty} \dd{z} W(z| x,t) = \frac{1}{\sqrt{\epsilon \hat{D}(x,t)}} \int_{-\infty}^{+\infty} \dd{z} F\left(\frac{z-\epsilon f(x,t) }{\sqrt{\epsilon \hat{D}(x,t)}} \right) =\\
    &= \int_{-\infty}^{+\infty} \dd{y} F(y)
\end{align*}  
Then:
\begin{align*}
    \langle z \rangle = \int \dd{z} z F\left(\frac{z- \epsilon f(x,t)}{\sqrt{\epsilon\hat{D}(x,t) }} \right) \frac{1}{\sqrt{\epsilon \hat{D}(x,t )}} = \int \dd{y} (\epsilon f(x,t) + y\sqrt{\epsilon \hat{D} (x,t)}) F(y) \overset{!}{=}  \epsilon f(x,t)
\end{align*}
So, summarizing, we require two conditions for $F$:
\begin{align*}
    1 &= \int \dd{y} F(y)\\
    0 &= \int \dd{y} y F(y)
\end{align*} 
Note that all normalized even functions (where $F(y) = F(-y)$) satisfy both of them (but they are not the only ones).\\
Consider now the higher moments:
\begin{align*}
    \mu_2 (x,t) &= \frac{1}{\sqrt{\epsilon \hat{D}(x,t)}} \int \dd{z} z^2 F\left(\frac{z- \epsilon f(x,t)}{\sqrt{\epsilon \hat{D}(x,t)}} \right) = \int \dd{y} (\epsilon f(x,t) + y \sqrt{\epsilon \hat{D}(x,t)})^2 F(y) = \\
    &= \int \dd{y} F(y) \left[ \epsilon^2 f^2 + 2 \epsilon \sqrt{\epsilon} \hat{D} f y + y^2 \hat{D} \epsilon \right] =\\
    &= \epsilon ^2 f^2  + 0 + \epsilon \hat{D} \int \dd{y} y^2 F(y) =\\
    &= \epsilon^2 f^2 + \epsilon \hat{D} \langle y^2 \rangle_F  \\
\end{align*}  
and so:
\begin{align*}
    \operatorname{Var}(z) = \mu_2 - \mu_1^2 = \epsilon \hat{D} \langle y^2 \rangle_F 
\end{align*}
which is consistent with the random walk properties.\\
Summarizing:
\begin{align*}
    \mu_1(x,t)
 = \epsilon f(x,t) \qquad \mu_2(x,t) = \epsilon \hat{D} (x,t) \langle y^2 \rangle_F = 2D(x,t)
\end{align*}
Now, for the third moment:
\begin{align*}
    \langle z^3 \rangle = \dots = \int \dd{y} F(y) (\epsilon f + y \sqrt{\epsilon \hat{D}} )^3 
\end{align*}
Recall that:
\begin{align*}
    \frac{w(x,t_{n+1}) - w(x,t_n)}{\epsilon} &= -\pdv{x} \left[w(x,t) \underbrace{\frac{\mu_1(x,t)}{\epsilon}}_{f(x,t)}  \right] + \frac{1}{2} \pdv[2]{x} \left[w(x,t) \underbrace{\frac{\mu_2(x,t)}{\epsilon}}_{2D(x,t)}  \right]  + \frac{1}{3!} \pdv[3]{x} \left[w \frac{\mu_3}{\epsilon} \right] + \dots
\end{align*}
The higher terms are at least of $O(\sqrt{\epsilon})$, and thus vanish in the limit. 
\begin{align*}
    \dot{w}(x,t) = -\pdv{x} \underbrace{\left[f(x,t) w(x,t) - \pdv{x}(D(x,t) w(x,t))\right]}_{J(x,t)} 
\end{align*}
Then, if $f(x,t) = 0$ (free particle) and $D(x,t) = D$ constant, then we get the diffusion equation:
\begin{align*}
\dot{w}(x,t) = D\pdv[2]{x} w(x,t) 
\end{align*}
However, this equation is much more general, and it is called the \textbf{Fokker-Plank}  equation.  

\section{Langenvin equation}
Following the approach of the first lessons, we now search for a general solution for the motion.\\
In general, the position at the next time step for a free particle is given by:
\begin{align*}
    x(t_{i+1}) = x(t_i) + \Delta x(t_i)
\end{align*}
where $\Delta x(t_i)$ is sampled from a gaussian distribution:
\begin{align*}
    \Delta x(t_i) \sim \frac{1}{\sqrt{4 \pi D \Delta t_i}} \exp\left(-\frac{\Delta x^2}{4 D \Delta t_i } \right)
\end{align*} 
We can rewrite it as:
\begin{align*}
    x(t_{i+1}) = x(t_i) + \sqrt{2D } \Delta B(t_i)
\end{align*}
with:
\begin{align*}
    \Delta x_i = \sqrt{2D} \Delta B(t_i); \quad \Delta B(t_i) \sim \frac{1}{\sqrt{2 \pi \Delta t_i}} \exp\left(-\frac{\Delta B_i^2}{2 \Delta t_i} \right) 
\end{align*}
and $\langle \Delta B^2 (t_i) \rangle = \Delta t_i$.\\
So, if we define $\Delta B(t_i) = \Delta t_i \xi(t_i)$, dividing both sides by $\Delta t_i$ we get:
\begin{align*}
    \dot{x}(t) = \sqrt{2D} \xi(t)
\end{align*}    
which is the \textbf{Langenvin} equation for a Brownian particle.\\
However, recall that the spatial derivative for a brownian trajectory does not exist - so this is a \textit{quasi-equation}, without a rigorous meaning.
\begin{align*}
    \xi(t_i) \sim \sqrt{\frac{\Delta t_i}{2 \pi} } \exp\left(-\frac{\Delta t_i \xi_i^2}{2 } \right)
\end{align*}  
and then:
\begin{align*}
    P(\xi \dots ) \propto \exp\left(-\frac{1}{2} \int \xi^2 (\tau) \dd{\tau} \right)
\end{align*}
and $\langle \xi(\tau) \rangle = 0$, $\langle \xi (\tau) \xi (\tau') \rangle = \delta(\tau - \tau')$.

\begin{align*}
    \dd{x}(t) = \sqrt{2D} \dd{B}; \qquad \dd{B} \sim \frac{1}{\sqrt{2 \pi \dd{t}}} \exp\left(-\frac{\dd{B}^2}{2 \dd{t}} \right) 
\end{align*}
Now consider a particle moving in 3D:
\begin{align*}
    m \ddot{\bm{r}} (t) = - \gamma\dot{\bm{r}} + \bm{F}_{\mathrm{ext} } + \bm{F}_{\mathrm{noise} }(t)
\end{align*}
e.g. $\bm{F}_{\mathrm{ext} 
} = -\bm{\hat{z}} g m$ for gravity, and $\bm{F}_{\mathrm{noise} }$ is the \textit{random force} due to collisions with other particles.\\
If $\bm{F}_{\mathrm{ext} }$ is conservative, we can write:
\begin{align*}
    \bm{F}_{\mathrm{ext} }(\bm{r},t) = - \bm{\nabla} V(\bm{r},t)
\end{align*}     
with a certain potential $V(\bm{r}, t)$.\\
Dividing both sides by $\gamma$:
\begin{align*}
    \frac{m}{\gamma} \ddot{\bm{r}}(t) = - \dot{\bm{r}} + \frac{\bm{F}_{\mathrm{ext} }(\bm{r},t)}{\gamma} + \frac{\bm{F}_{\mathrm{noise} }(t)}{\gamma}  
\end{align*}  
where, by Stokes law, $\gamma = 6 \pi a \eta$ and $\eta$ is the viscosity of the surrounding fluid. Note that $[m \gamma^{-1}] = [t]$ - so this ratio sets a \textit{timescale} $\tau$, i.e. the characteristic time $\bar{v} - v(t) = e^{-t/\tau}$, where $\bar{v}$ is the final (constant) velocity.\\
If $m/\gamma$ is much smaller than the timescale of observation, then we can neglect the acceleration term (i.e. the final constant state is reached almost immediately).\\
We can then write:
\begin{align*}
    \dot{\bm{r}} = \underbrace{\frac{\bm{F}_{\mathrm{ext} }}{\gamma}}_{f}  + \frac{\bm{F}_{\mathrm{noise} }}{\gamma} \underset{d=1}{ \Rightarrow}  \dot{x}(t) = f(x,t) + \sqrt{2 D(x,t)}
\end{align*}
This has no meaning unless we define a discretization:
\begin{align*}
    \dd{x}(t) = f(x,t) \dd{t} + \sqrt{2D} \dd{B}\qquad \dd{B} \sim \frac{1}{\sqrt{2\pi \dd{t}}} \exp\left(-\frac{(\dd{B})^2}{2\dd{t}} \right) 
\end{align*}
which is a \textit{stochastic differential equation}. But how to integrate it? 

\end{document}
