%&latex
%
\documentclass[../template.tex]{subfiles}
\begin{document}

\section{A first functional}
\lesson{6}{28/10/19}
Consider a Brownian trajectory $x(\tau)$ (from now on, we will assume that all trajectories start in $x=0$ at $t = 0$), and a functional that \textit{weights} every traversed point $x(\tau)$ with a function $a\colon \mathbb{R} \to \mathbb{R}$, and then applies another function $F\colon \mathbb{R} \to \mathbb{R}$ to the total integral:
\begin{align*}
    F[x(\tau)] = F\left(\int_0^t a(\tau) x(\tau) \dd{\tau}\right)
\end{align*}    
For simplicity, we set $D = 1/4$, so that:
\begin{align*}
    \dd{\mathbb{P}}_{t_1,\dots,t_n} (x_1,\dots,x_n|0,0) = \exp\left(-\sum_{i=1}^n \frac{(x_i - x_{i-1})^2}{\pi \Delta t_i} \right) \prod_{i=1}^n \frac{\dd{x_i}}{\sqrt{\pi \Delta t_i}} 
\end{align*} 
This is equivalent to a time rescaling $t \to \tau = 4Dt$.

We want now to compute $\langle F \rangle$:
\begin{align*}
    I_3 \equiv \langle F[x(\tau)] \rangle_w = \int_{\mathcal{C}\{0,0;t\}} \dd{_Wx(\tau)} F[x(\tau)]
\end{align*}

\begin{expl}
    \textbf{Note}: the next computations will follow the book. Prof. Maritan's method for evaluating $I_3$ is quicker, but more advanced, and will be presented at the end.  
\end{expl}
Then we start by discretizing, by choosing a time grid $0 = t_0 < t_1 < \dots < t_N = t$:
\begin{align*}
    I_3 &= \lim_{N \to \infty} I_3^{(N)}\\
    I_3^{(N)} &= \int_{-\infty}^{+\infty} \frac{\dd{x_1}}{\sqrt{\pi \Delta t_1}} \dots \int_{-\infty}^{+\infty} \frac{\dd{x_N}}{\sqrt{ \pi \Delta t_N }}  F\left(\sum_{i=1}^N a_i x_i \Delta t_i\right) \exp\left(-\sum_{i=1}^N \frac{(\hlc{Yellow}{x_i - x_{i-1}})^2}{\Delta t_i} \right) \qquad
\substack{\displaystyle a_i \equiv a(t_i)\\\displaystyle x_i \equiv x(t_i)}
\end{align*}
This integral can be evaluated by transforming it to a \textit{gaussian integral} that we already know. So we start by changing variables:
\begin{align}
    x_i - x_{i-1} = y_i \qquad i=1,\dots,N \label{eqn:first-cov}
\end{align}
Note that:
\begin{align*}
    \sum_{j=1}^i y_j = \cancel{x_1} - \underbrace{x_0}_{=0}  + x_2 - \cancel{x_1} + \dots + x_i - \bcancel{x_{i-1}} = x_i \qquad 1 \leq i \leq N
\end{align*}
So, when we compute the transformation of the volume element:
\begin{align*}
    \operatorname{det}\left|\pdv{\{x_i\}}{\{y_j\}}\right| = \operatorname{det} \left|\begin{array}{cccc}
    1 & 0 & 0 & 0 \\ 
    1 & 1 & 0 & 0 \\ 
    \vdots & \vdots & \ddots & 0 \\ 
    1 & 1 & \cdots & 1
    \end{array}\right|_{N\times N} = 1
\end{align*}
as the determinant of a lower triangular matrix is equal to the product of the diagonal entries.

All that's left is to transform the argument of $F$. Let's start by writing the first terms of the sum and apply the change of variables:
\begin{align}\nonumber
    \sum_{i=1}^N a_i x_i \Delta t_i &= a_1 x_1 \Delta t_1 + a_2 x_2 \Delta t_2 + \dots = \\ \nonumber
    &= a_1(y_1) \Delta t_1 + a_2(y_1 + y_2) \Delta t_2 + \dots =\\ \nonumber
    &= y_1\left(\sum_{j=1}^N a_j \Delta t_j\right) + y_2\left(\sum_{j=2}^N a_j \Delta t_j\right) + \dots + y_N a_N \Delta t_N =\\
    &= \sum_{i=1}^N y_i \underbrace{\left(\sum_{j=i}^N a_j \Delta t_j\right)}_{A_i} \equiv \sum_{i=1}^N A_i y_i \label{eqn:Ai}
\end{align} 
Substituting everything back:
\begin{align*}
    I_3^{(N)} = \int_{-\infty}^{+\infty} \frac{\dd{y_1}}{\sqrt{\pi \Delta t_1}} \dots \int_{-\infty}^{+\infty} \frac{\dd{y_N}}{\sqrt{\pi \Delta t_N}} F\left(\sum_{i=1}^N A_i y_i\right) \exp\left(-\sum_{i=1}^N \frac{y_i^2}{\Delta t_i} \right) \qquad A_i = \sum_{j=i}^N a_j \Delta t_j
\end{align*}
We can simplify this integral a bit more by rescaling the $y_i$:
\begin{align*}
    z_i = A_i y_i\qquad \dd{y_i} = \frac{\dd{z_i}}{A_i} 
\end{align*} 
As each $y_i$ is transformed independently, the jacobian is diagonal.
\begin{align*}
    I_3^{(N)} = \int_{-\infty}^{+\infty} \frac{\dd{z_1}}{\sqrt{\pi A_1^2 \Delta t_1}} \dots \int_{-\infty}^{+\infty} \frac{\dd{z_N}}{\sqrt{ \pi A_N^2 \Delta t_N}} F(z_1 +\dots + z_N) \exp\left(-\sum_{i=1}^N \frac{z_i^2}{A_i^2 \Delta t_i} \right)  
\end{align*}
This is the expected value of a function of the \textit{sum} of $N$ normally distributed random variables $\{z_i\}$. The idea is now to \textit{isolate} one of them from the argument of $F$, integrate over it, and reiterate. This is done by changing variables yet again:
\begin{align*}
    \begin{cases}
        \eta = z_1 + z_2\\
        \xi = z_2
    \end{cases} \Rightarrow \begin{cases}
        z_1 = \eta - \xi\\
        z_2 = \xi
    \end{cases} \Rightarrow \operatorname{det} \left|\pdv{\{z_1,z_2\}}{\{\eta,\xi\}}\right| = \left|\begin{array}{cc}
    1 & -1 \\ 
    0 & 1
    \end{array}\right| = 1
\end{align*}     
leading to:
\begin{align*}
    I_3^{(N)} &= \int_{-\infty}^{+\infty} \frac{\dd{\eta}}{\sqrt{ \pi A_1^2 \Delta t_1}}\int_{-\infty}^{+\infty} \frac{\dd{\xi}}{\sqrt{ \pi A_2^2 \Delta t_2 }} \int_{-\infty}^{+\infty} \frac{\dd{z_3}}{\sqrt{\pi A_3^2 \Delta t_3}}\cdots \int_{-\infty}^{+\infty} \frac{\dd{z_N}}{\sqrt{\pi A_N^2 \Delta t_N}} \cdot\\
    &\qquad \cdot F(\eta + z_3 + \dots + z_N) \exp\left(-\frac{(\eta-\xi)^2}{A_1^2 \Delta t_1} - \frac{\xi^2}{A_2^2 \Delta t_2} - \sum_{i=3}^N \frac{z_i^2}{A_i^2 \Delta t_i} \right)
\end{align*}
Note how $\xi$ does not enter in the $F$ argument, and so we can integrate over it:
\begin{align*}
    I_\xi &= \int_{-\infty}^{+\infty} \dd{\xi} \frac{1}{\sqrt{\pi A_1^2 \Delta t_1} \sqrt{\pi A_2^2 \Delta t_1 }}\exp\left(-\frac{(\eta - \xi)^2}{A_1^2 \Delta t_1} - \frac{\xi^2}{A_2^2 \Delta t_2} \right) =\\
    &= \int_{-\infty}^{+\infty} \dd{\xi} (\cdots) \exp\left(-\frac{\hlc{Yellow}{\xi^2 (A_2^2 \Delta t_1 + A_2^2 \Delta t_2)} - \hlc{SkyBlue}{(2\eta A_2^2 \Delta t_2)} - \hlc{ForestGreen}{(-\eta^2 A_2^2 \Delta t_2)}}{A_1^2 A_2^2 \Delta t_1 \Delta t_2} \right)
\end{align*}  
Recall the gaussian integral formula:
\begin{align}
    \int_{-\infty}^{+\infty} \exp(-\hlc{Yellow}{a}x^2 +\hlc{SkyBlue}{ b}x + \hlc{ForestGreen}{c}) \dd{x} = \sqrt{\frac{\pi}{a} } \exp\left(\frac{b^2}{4a} + c \right)\label{eqn:gauss-int}
\end{align}
which evaluates to:
\begin{align*}
    I_\xi = \frac{1}{\sqrt{\pi (A_1^2 \Delta t_1 + A_2^2 \Delta t_2)}} \exp\left(-\frac{\eta^2}{A_1^2 \Delta t_1 + A_2^2 \Delta t_2} \right) 
\end{align*}
and substituting back in $I_3^{(N)}$:
\begin{align*}
    I_3^{(N)} &= \int_{-\infty}^{+\infty} \frac{\dd{\eta}}{\sqrt{\pi A_1^2 \Delta t_1 + \pi A_2^2 \Delta t_2}} \int_{-\infty}^{+\infty} \frac{\dd{z_3}}{\sqrt{ \pi A_3^2 \Delta t_3}} \cdots \int_{-\infty}^{+\infty} \frac{\dd{z_N}}{\sqrt{\pi A_N^2 \Delta t_N }} \cdot  \\
    &\qquad \cdot F(\eta + z_3 + \dots + z_N) \exp\left(-\frac{\eta^2}{A_1^2 \Delta t_1 + A_2^2 \Delta t_2} - \sum_{i=3}^N \frac{z_i^2}{A_1^2 \Delta t_i} \right)
\end{align*} 
We can now reiterate this procedure until only one integration is left:
\begin{align*}
    I_3^{(N)} = \int_{-\infty}^{+\infty} \frac{\dd{z}}{\sqrt{\pi \sum_{i=1}^N A_i^2 \Delta t_i}} F(z) \exp\left(-\frac{z^2}{\sum_{i=1}^N A_i^2 \Delta t_i} \right) 
\end{align*}
We are now finally ready to take the continuum limit $\Delta t_i \to 0$, $N \to\infty$. Note that:
\begin{align}
    \lim_{\Delta t_i \to 0} A_i = \int_{\tau}^t a(s) \dd{s} = A(\tau)
    \label{eqn:Atau}
\end{align}  
as the discrete sum goes from $t_i = \tau$ to $t_N = t$. Then:
\begin{align*}
    R \equiv \lim_{\Delta t \to 0 } \sum_{i=1}^N A_i^2 \Delta t_i = \int_0^t \dd{\tau} \left(\int_{\tau}^t \dd{s} a(s) \right)^2
\end{align*}  
and so:
\begin{align*}
    I_3 = \lim_{N \to\infty } I_3^{(N)} = \int_{-\infty}^{+\infty} \dd{z} \frac{F(z)}{\sqrt{\pi R}} \exp\left(-\frac{z^2}{R} \right) 
\end{align*}
And to recover $D$ we can just substitute $R \to 4 D R$.   



\subsection{Alternative method}
We consider now a different (quicker) technique to compute $I_3$. So we start again from:
\begin{align*}
    I_3 \equiv \langle F[x(\tau)] \rangle_w = \int_{C\{0,0;t\}} \dd{_Wx(\tau)} F\left(\int_0^t a(\tau) x(\tau) \dd{\tau}\right)
\end{align*}
It is convenient to apply the change of variables we did in (\ref{eqn:Ai}). We can do \textit{before} discretizing, by defining $A(\tau)$ as in (\ref{eqn:Atau}):
\begin{align}
    A(\tau) \equiv \int_\tau^t a(s) \dd{s}
    \label{eqn:Atau2}
\end{align}
Note that $\dot{A}(\tau) = -a(\tau)$, and so the argument of $F$ becomes:
\begin{align*}
    \int_0^t a(\tau) x(\tau) \dd{\tau} &= -\int_0^t \partial_\tau A(\tau) x(\tau) \dd{\tau} 
\intertext{Integrating by parts:}
    &= -\cancel{\left[x(\tau) A(\tau)\right]}_{\tau=0}^{\tau=t} + \int_0^t A(\tau) \dot{x}(\tau) \dd{\tau}
\end{align*}
And now we discretize the path over the instants $0 = t_0 < t_1 < \dots < t_N$, so that:
\begin{align*}
    \int_0^t A(\tau) \dot{x}(\tau) \dd{\tau} &= \lim_{\Delta t_i \to 0} \sum_{i=1}^N A(t_i) \frac{x(t_i) - x(t_{i-1})}{\Delta t_i} \Delta t_i =\\
    &= \lim_{N \to \infty} \sum_{i=1}^N A_i (x_i - x_{i-1})
    = \lim_{N \to \infty} \sum_{i=1}^N A_i \Delta x_i
    \qquad \substack{\displaystyle x_i \equiv x(t_i)\\\displaystyle A_i \equiv A(t_i)}
\end{align*}
Substituting back (here $D = 1/4$ for simplicity):
\begin{align*}
    I_3 &= \lim_{N \to\infty } I_3^{(N)}\\
    I_3^{(N)} &= \int_{\mathbb{R}^N} \left(\prod_{i=1}^N \frac{\dd{x_i}}{\sqrt{ \pi \Delta t_i}} \right)\exp\left(-\sum_{i=1}^N \frac{(\Delta x_i)^2}{\Delta t_i} \right)  F\left(\sum_{i=1}^N A_i \Delta x_i \right)
\end{align*}
The idea is now to \textit{apply} a change of random variable, rewriting the average $\langle F[x(\tau)] \rangle_w$ (according to the distribution of \textit{paths}) as the average $\langle F(z) \rangle_{p(z)}$, where $p(z)$ is the distribution followed by the argument of $F$:
\begin{align*}
    \sum_{i=1}^N A_i \Delta x_i
\end{align*}  
So, we begin by inserting the appropriate $\delta$:
\begin{align*}
    I_3^{(N)} = \int_{\mathbb{R}^N} \left(\prod_{i=1}^N \frac{\dd{x_i}}{\sqrt{ \pi \Delta t_i }}\right) \exp\left(-\sum_{i=1}^N \frac{(\Delta x_i)^2}{\Delta t_i} \right)F\left(\sum_{i=1}^N A_i \Delta x_i\right) \underbrace{\textcolor{Blue}{\int_{\mathbb{R}} \dd{z} \delta\left(z - \sum_{i=1}^{N} A_i \Delta x_i\right)} }_{=1} 
\end{align*}   
Exchanging the integrals leads to:
\begin{align*}
    &\langle F\left(\sum_{i=1}^N A_i \Delta x_i\right) \rangle_w = \langle F(z) \rangle_{p(z)} =\span\\
   & =\int_{\mathbb{R}} \dd{z} F(z) \underbrace{\int_{\mathbb{R}^N} \left(\prod_{i=1}^N \frac{\dd{x_i}}{\sqrt{\pi \Delta t_i }}\right) F\left(\sum_{i=1}^N A_i \Delta x_i\right) \delta\left(z- \sum_{i=1}^N A_i \Delta x_i\right) \exp\left(-\sum_{i=1}^N \frac{(\Delta x_i)^2}{\Delta t_i} \right)}_{p(z)} 
\end{align*}
We can evaluate $I_3^{(N)}$ by transforming it to a \textit{gaussian integral}. First, we remove the $\delta$ with a Fourier transform:
\begin{align*}
    2\pi \delta (x) = \int_{\mathbb{R}} e^{i \alpha x} \dd{\alpha}
\end{align*}    
which, in this case, leads to:
\begin{align*}
    \delta\left(z - \sum_{i=1}^N A_i \Delta x_i\right) = \int_{\mathbb{R}} \frac{\dd{\alpha}}{2\pi} \exp\left(i \alpha \left(z- \sum_{i=1}^N A_i \Delta x_i\right)\right) 
\end{align*} 
Substituting back:
\begin{align*}
    I_3^{(N)} = \int_{\mathbb{R}} \frac{\dd{\alpha}}{2 \pi} \int_{\mathbb{R}} \dd{z} F(z) e^{i \alpha z} \int_{\mathbb{R}^N} \left(\prod_{i=1}^N \frac{\dd{x_i}}{\sqrt{\pi \Delta t_i}}\right) \exp \left(-\sum_{i=1}^N \frac{\Delta x_i^2}{\Delta t_i} - i \alpha \sum_{i=1}^N A_i \Delta x_i \right)  
\end{align*}
We see that the last term is similar to a multivariate gaussian with a imaginary term, that we know how to integrate. We just need to remove the \textit{differences} in the exponential with a change of variables (as in (\ref{eqn:first-cov})):
\begin{align*}
    y_1 &= \Delta x_1 = x_1 -\overbrace{x_0}^{=0} = x_1\\
    y_2 &= \Delta x_2 = x_2 - x_1\\
    &\>\>\vdots\\
    y_N &= \Delta x_N = x_N - x_{N-1}
\end{align*}

The volume element will be transformed by the determinant of the Jacobian:
\begin{align*}
    J = \operatorname{det} \frac{\partial (x_1 \dots x_N)}{\partial(y_1 \dots y_N)} = \left[\operatorname{det} \frac{\partial (y_1 \dots y_N)}{\partial (x_1 \dots x_N)}  \right]^{-1} = \left|
    \begin{array}{ccccc}
    1 & 0 & 0 & \dots & 0 \\ 
    -1 & 1 & 0 & \dots & 0 \\ 
    0 & -1 & 1 & 0 & \vdots \\ 
    \vdots & \ddots & \ddots & \ddots & \vdots \\ 
    0 & 0 & 0 & \dots & 1
    \end{array}
    \right|^{-1} = 1
\end{align*}
where we used the fact that $\operatorname{det}A^{-1} = (\operatorname{det}A)^{-1}$, and that the determinant of a lower triangular matrix is just the product of the diagonal entries.

The integral then becomes:
\begin{align*}
    I_3^{(N)} &= 
    \int_{\mathbb{R}} \frac{\dd{\alpha}}{2 \pi} \int_{\mathbb{R}} \dd{z} F(z) e^{i \alpha z} \int_{\mathbb{R}^N} \left( \prod_{i=1}^N \frac{\dd{y_i}}{\sqrt{\pi \Delta t_i}}\right) \exp\left(-\sum_{i=1}^N \frac{y_i^2}{\Delta t_i} - i \alpha \sum_{i=1}^N A_i y_i \right) =\\
    &= \int_{\mathbb{R}} \frac{\dd{\alpha}}{2 \pi} \int_{\mathbb{R}} \dd{z} F(z) e^{i \alpha z} \left[\prod_{i=1}^N  \int_{\mathbb{R}} \frac{\dd{y_i}}{\sqrt{\pi \Delta t_i}} \exp\left(-\frac{y_i^2}{ \Delta t_i} - i \alpha A_i y_i \right) \right]   
\end{align*}
The terms in the product are all independent gaussian integrals. Recall that:
\begin{align}
    \int_{\mathbb{R}} \dd{k} e^{-ia k^2 - ibk} = \sqrt{\frac{\pi}{ia} } \exp\left(\frac{ib^2}{4 a} \right)
    \label{eqn:gaussian-integral1}
\end{align}
So, with $ia = 1/\Delta t_i$ and $b = \alpha A_i$ we get:
\begin{align*}
    \int_{\mathbb{R}} \frac{\dd{y_i}}{\sqrt{\pi \Delta t_i}} \exp\left(-\frac{y_i^2}{\Delta t_i} - i \alpha A_i y_i \right)  = \exp\left(-\frac{\alpha^2 A_i^2 \Delta t_i}{4} \right)
\end{align*}  
and substituting back in the integral leads to:
\begin{align*}
    I_3^{(N)} &= \int_{\mathbb{R}} \frac{\dd{\alpha}}{2 \pi} \int_{\mathbb{R}} \dd{z} F(z)  e^{i \alpha z} \left[ \prod_{i=1}^N \exp\left(-\frac{\alpha^2 A_i^2 \Delta t_i}{4} \right) \right] =\\
    &= \int_{\mathbb{R}} \frac{\dd{\alpha}}{2 \pi} \int_{\mathbb{R}} \dd{z} F(z) e^{i \alpha z} \exp\left(-\frac{\alpha^2}{4} \sum_{i=1}^N A_i^2 \Delta t_i \right)
\end{align*}
Applying the continuum limit ($N \to\infty$, $\Delta t_i \to 0$), the exponential argument becomes the limit of a Riemann sum, i.e. a integral:
\begin{align*}
    \sum_{i=1}^N A(t_i)^2 \Delta t_i  \xrightarrow[N \to \infty]{}  \int_0^t A^2(\tau) \dd{\tau} \underset{(\ref{eqn:Atau2})}{=}  \int_0^t \dd{\tau} \left(\int_{\tau}^t \dd{s} a(s)\right)^2 \equiv R(t)
\end{align*}
Substituting back:
\begin{align*}
    I_3 \equiv
    \langle F\left(\int_{0}^t a(\tau )x(\tau)\right) \rangle = \lim_{N \to\infty } I_3^{(N)} = \int_{\mathbb{R}} \dd{z} \int_{R} \frac{\dd{\alpha}}{2 \pi} \exp\left(-\frac{\alpha^2}{4} R(t) + i \alpha z \right) 
\end{align*}
All that's left is to evaluate the last gaussian integral thanks to (\ref{eqn:gaussian-integral1}) with $ia = R(t)/4$ and $b = -z$, leading to:
\begin{align*}
    I_3 = \int_{\mathbb{R}} \dd{z} F(z) \frac{1}{2 \pi} \sqrt{\frac{4 \pi}{R(t)} } \exp\left(-\frac{z^2}{R(t)} \right) = \frac{1}{\sqrt{\pi R(t)}} \int_{\mathbb{R}} \dd{z} F(z) \exp\left(-\frac{z^2}{R(t)} \right)  
\end{align*}

So, we showed that:
\begin{align}
    \langle F\left(\int_0^t a(\tau) x(\tau)\dd{\tau}\right) \rangle_w = \sqrt{\frac{1}{ \pi R(t)} } \int_{\mathbb{R}} \dd{z} F(z) \exp\left(-\frac{z^2}{R(t)} \right); \qquad R(t) \equiv \int_0^t \dd{\tau}\left(\int_\tau^t a(s) \dd{s}\right)^2
    \label{eqn:I3-res}
\end{align}

\newpage 
\begin{example}[Generating function]
    Let $F(z) = e^{hz}$. Inserting in (\ref{eqn:I3-res}) results in:
    \begin{align}
        \langle \exp\left(h \int_0^t a(\tau) x(\tau) \dd{\tau}\right) \rangle_w = \frac{1}{ \sqrt{\pi R}}  \int_{\mathbb{R}} \dd{z} \exp\left(-\frac{z^2}{R} + hz \right) \underset{(a)}{=}  \exp\left(\frac{h^2 R}{4} \right) \equiv G(h)
        \label{eqn:Is}
    \end{align}
    where in (a) we used formula (\ref{eqn:gauss-int}) with $a=1/R$ and $b=h$. 
    
    Note that $G(h)$ is the \textbf{generating function} (CFR def. in 10/10 notes) of the integral:
    \begin{align*}
        I=\int_0^t a(\tau) x(\tau) \dd{\tau}
    \end{align*}   
    We can then retrieve the $n$-th moment of $I$ by computing the $n$-th derivative of $G(h)$:
    \begin{align*}
        \dv[n]{h} G(h)\Big|_{h = 0} = \langle I^n \rangle_w
    \end{align*} 
    We can see this by differentiating the left side of (\ref{eqn:Is}):      
    \begin{align*}
        G'(h) = \langle \int_0^t a(\tau) x(\tau) \dd{\tau} \exp\left(h \int_0^\tau a x \dd{\tau}\right) \rangle_w
    \end{align*}
    and then setting $h = 0$:
    \begin{align*}
        G'(0) = \langle \int_0^t a(\tau) x(\tau) \dd{\tau} \rangle_w = \langle I \rangle_w
    \end{align*}  
    Then, differentiating the right side of (\ref{eqn:Is}) we have immediately the result:
    \begin{align*}
        \langle I \rangle_w = G'(h)\Big|_{h=0} =  \frac{h}{2} R \exp\left(\frac{h^2 R}{4} \right) \Big|_{h = 0} = 0
    \end{align*}

    If we differentiate again we get the second moment:
    \begin{align*}
        G''(h) = \frac{R}{2} \exp\left(\frac{h^2 R}{4} \right) + \frac{h^2}{4} R^2 \exp\left(\frac{h^2 R}{4} \right) \Rightarrow G''(0) = \langle I^2 \rangle_w = \frac{R}{2}  
    \end{align*} 
    
    Consider now a generic odd moment:
    \begin{align*}
        \langle \left(\int_0^t a(\tau)x(\tau) \dd{\tau}\right)^{2k+1} \rangle_w = 0 \qquad \forall k \in \mathbb{N}
    \end{align*}
    In fact, if we expand $G(h)$, we get:
    \begin{align*}
        G(h) = \sum_{n=0}^\infty
     \left(\frac{R}{4} \right)^n \frac{1}{n!} h^{2n} \end{align*}
    Since all the powers are even, if we differentiate an odd number of times and set $h=0$ we are \q{selecting} an odd power - which just is not there - and so the result will be $0$.\\
    On the other hand, an even moment leads to:
    \begin{align*}
        \langle \left(\int_0^t a(\tau) x(\tau) \dd{\tau}\right)^{2k} \rangle_w = \left(\frac{R}{2} \right)^2 \frac{(2k)!}{2^{k} k!} 
    \end{align*}
    (computations omitted).
        
\end{example}

\section{Exponential functional}
We consider now the following functional:
\begin{align*}
    F[x(\tau)] = \exp\left(-\int_0^t \dd{\tau} P(\tau) x^2(\tau) \right) 
\end{align*}
As before, we wish to compute $\langle F \rangle_w$. We start by discretizing the path over a \textbf{uniform}\footnote{The same result can be proved without this assumption, but which a much more heavy notation.}  grid $0 = t_0 < t_1 < \dots < t_N = t$ so that $\Delta t_i = t_i - t_{i-1} \equiv \epsilon = t/N$. 
\begin{align} \nonumber
    I_4 &\equiv \int_{\mathcal{C}\{0,0;t\}} \dd{_Wx(\tau)} \exp\left(-\int_0^t \dd{\tau} P(\tau) x^2(\tau)\right) = \lim_{N \to \infty} I_4^{(N)}\\
    I_4^{(N)} &= \int_{-\infty}^{+\infty} \frac{\dd{x_1}}{\sqrt{ \pi \epsilon}} \cdots \int_{-\infty}^{+\infty} \frac{\dd{x_N}}{\sqrt{\pi \epsilon }} \exp\left(-\sum_{i=1}^N P_i x_i^2 \epsilon - \sum_{i=1}^N \frac{(x_i - x_{i-1})^2}{\epsilon} \right)  \qquad \substack{\displaystyle x_i \equiv x(ti)\\\displaystyle P_i \equiv P(t_i)} \label{eqn:I4N}
\end{align}  
The exponential argument is a \textbf{quadratic form}:
\begin{align*}
    &-\epsilon (P_1 x_1^2 + \dots + P_N x_N^2) - \frac{1}{\epsilon} [\cancel{x_0^2 }+ x_1^2 - \cancel{2x_0 x}_1 + x_1^2 + x_2^2 -2 x_1 x_2 + \dots + x_{N-1}^2 + x_N^2 -2x_{N-1}x_N] =\\
    &=-\epsilon \sum_{i=1}^N P_i x_i^2 - \frac{1}{\epsilon} \left[2 \sum_{i=1}^{N-1} x_i^2 + x_N^2 -2 \sum_{i=1}^N x_{i-1} x_i \right]=\\
    &=-\left[x_1^2 \left(\hlc{Yellow}{\epsilon P_1 + \frac{2}{\epsilon}} \right) + \dots + x^2_{N-1}\left(\hlc{Yellow}{\epsilon P_{N-1} + \frac{2}{\epsilon}} \right) + x_N^2\left(\hlc{SkyBlue}{\epsilon P_N + \frac{1}{\epsilon}} \right)+\hlc{ForestGreen}{\frac{2}{\epsilon}} \sum_{i=1}^N x_i x_{i-1} \right] =\\
    &=- \sum_{i,j=1}^N A_{ij} x_i x_j 
\end{align*} 
where $A_{ij}$ are matrix elements of a matrix $A_N$:
\begin{align*}
    A_{ij} = \delta_{ij} a_i - \hlc{ForestGreen}{\frac{1}{\epsilon}} (\delta_{i,j-1} + \delta_{i-1,j}) \qquad a_i = \hlc{Yellow}{P_i \epsilon + \frac{1}{\epsilon} (2} \hlc{SkyBlue}{- \delta_{iN}})\\
    A_N = \left(\begin{array}{ccccc}
        a_1 & - 1/\epsilon & 0 & \dots & 0 \\ 
        -\epsilon^{-1} & a_2 & -\epsilon^{-1} & 0 & \vdots\\ 
        0 & \ddots & \ddots & \ddots & 0\\ 
        0 & 0 & -\epsilon^{-1} & a_{N-1} & -\epsilon^{-1}\\
        0 & 0 & 0 & -\epsilon^{-1} & a_N
        \end{array}\right)
\end{align*} 
We can now rewrite $I_4^{(N)}$ as:
\begin{align*}
    I_4^{(N)} = \int_{\mathbb{R}^N} \left(\prod_{i=1}^N \frac{\dd{x_i}}{\sqrt{ \pi \epsilon}} \right) e^{-\bm{x}^T A_N \bm{x}} \qquad \bm{x}^T = (x_1,\dots,x_N)
\end{align*} 
This is the integral of a multivariate gaussian, and evaluates to:
\begin{align*}
    I_4^{(N)} = \frac{1}{\epsilon^{N/2} (\operatorname{det} A_N)^{1/2}} = \frac{1}{(\operatorname{det}(\epsilon A_N))^{1/2} }  
\end{align*}
as for a $N \times N$ matrix we have $\operatorname{det}(\epsilon A_N) = \epsilon^N \operatorname{det}A_N$. This has the advantage of removing all denominators in $A_N$.

To compute this determinant we use a method suggested by Gelfand and Yaglom (1960). We start by denoting with $D_k^{(N)}$ the determinant of the matrix obtained by removing the first $k-1$ rows and columns from $\epsilon A_N$:
\begin{align*}
    D_k^{(N)} \equiv \left|\begin{array}{ccccc}
        \hlc{Yellow}{\epsilon a_k} & \hlc{SkyBlue}{-1} & 0 & \dots & 0 \\ 
        -1 & \epsilon a_{k+1} & -1 & 0 & \vdots \\ 
        0 & \ddots & \ddots & \ddots & 0 \\ 
        \vdots & \ddots & -1 & a_{N-1} & -1 \\ 
        0 & \dots & 0 & -1 & \epsilon a_N
        \end{array}\right|
\end{align*}    
So that $D_1^{(N)} = \operatorname{det} \epsilon A_N$ is the determinant we want to compute.


Expanding $D_k^{(N)}$ from the first row we get:
\begin{align*}
    D_k^{(N)} &=\hlc{Yellow}{ \epsilon a_k }D_{k+1} -(\hlc{SkyBlue}{-1}) \left|\begin{array}{ccccc}
    \hlc{ForestGreen}{-1} & -1 & 0 & \dots & 0 \\ 
    0 & \epsilon a_{k+2} & -1 & \ddots & \vdots \\ 
    0 & \ddots & \ddots & \ddots & 0\\
    \vdots & \ddots  & -1 & a_{N-1} & -1\\ 
    0 & 0 & 0 &-1 & \epsilon a_N
    \end{array}\right| =\\
    &\underset{(a)}{=} \epsilon a_k D_{k+1}^{(N)} + (\hlc{ForestGreen}{-1}) D_{k+2}^{(N)} = \epsilon \left(\epsilon P_k + \frac{2}{\epsilon} \right) D_{k+1}^{(N)} - D_{k+2}^{(N)} = \\
    &= (\epsilon^2 P_k + 2) D_{k+1}^{(N)} - D_{k+2}^{(N)}
\end{align*} 
where in (a) we expanded the last determinant following the first column.

Rearranging:
\begin{align}
    \frac{D_k^{(N)} - 2 D_{k+1}^{(N)} + D_{k+2}^{(N)}}{\epsilon^2} = P_k D_{k+1}^{(N)} 
    \label{eqn:discrete-rel}
\end{align}
We introduce now the variable $\tau = (k-1)t/N$, representing the \textit{fraction} of removed rows/columns in each determinant, rescaled to the final time $t$. Performing a continuum limit $N \to \infty$ we can then map $D_k^{(N)}  \xrightarrow[N \to \infty]{}  D(s)$. Then, the relation (\ref{eqn:discrete-rel}) becomes a \textit{differential equation}: 
\begin{align}
    \dv[2]{D(\tau)}{\tau} = P(\tau) D(\tau)
    \label{eqn:detdiffeq}
\end{align}
In fact, note that the first term of (\ref{eqn:discrete-rel}) is a second derivative in the \textit{finite difference} approximation. This can be shown by Taylor expanding a generic function $f(x)$ to get the points immediately before and after:
\begin{align*}
    f(x+ \Delta x) &= f(x) + f'(x) \Delta x + \frac{1}{2} f''(x) (\Delta x)^2  + O((\Delta x)^3)\\ 
    f(x- \Delta x) &= f(x) - f'(x) \Delta x + \frac{1}{2} f''(x) (\Delta x)^2 + O((\Delta x)^3)  
\end{align*} 
Summing side by side, and denoting $f(x) \equiv f_i$, $f(x-\Delta x) \equiv f_{i-1}$ and $f(x+ \Delta x) = f_{i+1}$:
\begin{align*}
    f_{i+1} + f_{i-1} = 2 f_i + f_i'' (\Delta x)^2 + O((\Delta x)^3)
\end{align*}   
Rearranging, shifting $i \to i+1$ and ignoring the higher order terms leads to:
\begin{align*}
    \frac{f_{i+2} - 2f_{i+1} + f_i}{(\Delta x)^2} = f_{i+1}'' 
\end{align*}

\begin{expl}
    Analogously, this can be seen by computing the second derivative as \textit{the derivative of the first derivative} in terms of incremental ratios:
    \begin{align*}
        f''(x) &= \lim_{\Delta x \to 0} \frac{1}{\Delta x} \left(\frac{f(x+\Delta x) - f(x)}{\Delta x} - \frac{f(x)-f(x - \Delta x)}{\Delta x }  \right) =\\
        &= \lim_{\Delta x \to 0} \frac{f(x+ \Delta x) - 2 f(x) + f(x- \Delta x)}{(\Delta x)^2} 
    \end{align*} 
\end{expl}

Returning to the problem, we note that the determinant of the full matrix, in the continuum limit, is given by:
\begin{align*}
    \operatorname{det}(\epsilon A_N) = D_1^{(N)}  \xrightarrow[N \to \infty]{}  D(0) 
\end{align*}
(as $s = (k-1)t/N \Big|_{k=1} \equiv 0$). So, we just need to solve (\ref{eqn:detdiffeq}) and evaluate it at $\tau = 0$. 

To do this, we first need \textit{two} boundary conditions, as (\ref{eqn:detdiffeq}) is a second order differential equation. 

Noting that $D_N^{(N)}$ is just the last diagonal entry, we have:
\begin{align*}
    D_N^{(N)} = \epsilon a_N = \epsilon^2 p_N + 1  \xrightarrow[\substack{N \to \infty\\ \epsilon \to 0}]{} 1 
\end{align*}  
As $s = (k-1)t/N \big|_{k=N} = t$ for $N \to \infty$, this means that:
\begin{align*}
    D(t) = 1
\end{align*}
For the second boundary condition, we search a relation for the first derivative at $\tau = t$:
\begin{align*}
    \dv{D(\tau)}{\tau} \Big|_{\tau = t} = \lim_{N \to \infty} \frac{D_N^{(N)} - D_{N-1}^{(N)}}{\epsilon} 
\end{align*}   
$D^{(N)}_{N-1}$ can be computed directly:
\begin{align*}
    D_{N-1}^{(N)} = \left|\begin{array}{cc}
    P_{N-1} \epsilon^2 +2 & -1 \\ 
    -1 & P_N \epsilon^2 + 1
    \end{array}\right| = P_{N-1} P_N \epsilon^4 + \epsilon^2(P_{N-1}+ 2P_N) + 1 
\end{align*} 
leading to:
\begin{align*}
    \dv{D(\tau)}{\tau}\Big|_{\tau=t} = \lim_{\epsilon \to 0} \frac{\epsilon^2 P_N + 1 - P_{N-1}P_N \epsilon^4 - \epsilon^2(P_{N-1} + 2 P_N)-1}{\epsilon}  = 0
\end{align*}

Summarizing, we found that:
\begin{align*}
    I_4 \equiv \langle \exp\left(-\int_0^t \dd{\tau} P(\tau) x^2(\tau) \right)\rangle_w = \frac{1}{\sqrt{D(0)}} 
\end{align*}
where $D(\tau)$ is the solution of the differential equation:
\begin{align*}
    \dv[2]{D(\tau)}{\tau} = P(\tau) D(\tau)
\end{align*}  
with the following boundary conditions:
\begin{align*}
    \begin{cases}
        D(t) = 1\\
        \dot{D}(t) = \dv{D(\tau)}{\tau}\Big|_{\tau = t} = 0
    \end{cases}
\end{align*}

\begin{example}[$P(\tau) = k^2$, free end-point]
    Let's compute $I_4$ with the choice of $P(\tau) = k^2$. The differential equation becomes:
    \begin{align*}
        \dv[2]{D(\tau)}{\tau} = k^2 D(\tau)
    \end{align*}  
    which is that of a \textit{harmonic repulsor}. The solution is a linear combination of exponentials:
    \begin{align}
        D(\tau) = A e^{k \tau} + Be^{-k \tau}
        \label{eqn:Dtau}
    \end{align} 
    Differentiating:
    \begin{align*}
        \dot{D}(\tau) = k(Ae^{k \tau} - B e^{-k \tau})
    \end{align*}
    We can now impose the boundary conditions:
    \begin{align*}
        \begin{cases}
            D(t) \overset{!}{=}  1 = A e^{kt} + Be^{-kt} & (a)\\
            \dot{D}(t) \overset{!}{=} 0 = \cancel{k}(A e^{kt} - Be^{-kt}) & (b)
        \end{cases}
    \end{align*}
    leading to:
    \begin{align*}
        (a) + (b)&\colon 2 A e^{kt} = 1 \Rightarrow A = \frac{1}{2} e^{-kt}\\
        (a) - (b)&\colon 2Be^{-kt} = 1 \Rightarrow B = \frac{1}{2}e^{kt}  
    \end{align*}
    So the solution is:
    \begin{align}
        D(\tau) = \frac{1}{2}\left[e^{k(t-\tau)} + e^{-k(t-\tau)}\right] = \cosh(k(t-\tau)) 
        \label{eqn:Dtau1}
    \end{align}
    from which:
    \begin{align*}
        I_4 = \lim_{N \to\infty }^{(N)} = \frac{1}{\sqrt{D(0)}} = \frac{1}{\sqrt{\cosh(kt)}}  
    \end{align*}
\end{example}

\subsection{Fixed endpoint}
We consider now a small variation of $I_4$, where we integrated on paths with a fixed \textit{end-point} $x(t) \equiv x_t$:
\begin{align*}
    \hat{I}_4 = \langle \exp\left(-\int_0^t P(\tau) x^2(\tau) \dd{\tau}\right) \delta(\hlc{Yellow}{x}\hlc{SkyBlue}{-x(t)} \rangle_w = \int_{\mathcal{C}\{0,0;x_t,t\}} \exp\left(-\int_0^t P(\tau) x^2(\tau) \dd{\tau}\right)
\end{align*}   
First, we rewrite the $\delta$ in terms of a Fourier transform:
\begin{align*}
    I_4' = \int_{-\infty}^{+\infty} \frac{\dd{\alpha}}{2\pi} \hlc{Yellow}{e^{i \alpha x}} \langle \exp \left(-\int_0^t P(\tau) x^2(\tau) \dd{\tau}\right) \hlc{SkyBlue}{e^{-i \alpha x(t)}} \rangle_w 
\end{align*} 
Then we discretize the path as before, with $0 = t_0 < t_1 < \dots < t_N = t$ uniformly distributed ($\Delta t_i = t_i - t_{i-1} \equiv \epsilon = t/N$):
\begin{align*}
    \hat{I}_4 &= \lim_{N \to \infty} \hat{I}_4^{(N)}\\
    \hat{I}_4^{(N)} &= \textcolor{Red}{\int_{\mathbb{R}} \frac{\dd{\alpha}}{2 \pi}} \textcolor{Red}{e^{i \alpha x}} \int_{\mathbb{R}^N} \left(\prod_{i=1}^N \frac{\dd{x_i}}{\sqrt{ \pi \epsilon}} \right) \exp\left(-\sum_{i=1}^N P_i x_i^2 \epsilon - \sum_{i=1}^N \frac{(x_i - x_{i-1})^2}{\epsilon} \textcolor{Red}{- i \alpha x_N} \right)
\end{align*}
where the red terms are the only differences from (\ref{eqn:I4N}). We can rewrite the quadratic form with the matrix $A_N$ as before:
\begin{align*}
    \hat{I}_4^{(N)} = \int_{\mathbb{R}} \frac{\dd{\alpha}}{2 \pi} e^{i \alpha x} \int_{\mathbb{R}^N} \left(\prod_{i=1}^N \frac{\dd{x_i}}{\sqrt{\pi \epsilon}} \right) \exp(-\bm{x}^T A_N \bm{x} - i \alpha x_N)
\end{align*} 
Also, we can express $i \alpha x_N$ as a scalar product: 
\begin{align*}
    i \alpha x_N = \bm{h}^T \bm{x} \qquad h_l = \delta_{lN} (-i \alpha)
\end{align*}
So that we can now use the gaussian integral:
\begin{align*}
    \int_{\mathbb{R}^N} \dd[N]{\bm{x}} \exp\left(-\frac{1}{2} \bm{x}^T A \bm{x} + \bm{b} \cdot \bm{x} \right) = \exp \left(\frac{1}{2} \bm{b} \cdot A^{-1}\bm{b} \right) (2 \pi)^{N/2} (\operatorname{det} A)^{-1/2}
\end{align*}
with $A = 2 A_N$ and $\bm{b} = \bm{h}$:
\begin{align*}
    I' &\equiv \frac{1}{\sqrt{(\pi \epsilon)^N}} \int_{\mathbb{R}^N} \dd[N]{\bm{x}} \exp(-\bm{x}^T A \bm{x} + \bm{h}^T \bm{x}) = \frac{1}{\sqrt{(\pi \epsilon)^N}}\exp\left(\frac{1}{4} \bm{h}^T A^{-1} \bm{h} \right) (\cancel{2}\pi)^{N/2} (\cancel{2^N} \operatorname{det}A_N )^{-1/2} =\\
    &= \frac{1}{\sqrt{(\cancel{\pi} \epsilon)^N}}\sqrt{\frac{\cancel{\pi ^N}}{\operatorname{det} A_N } } \exp\left(\frac{1}{4}(-i \alpha)^2 (A^{-1}_N)_{NN} \right) = \underbrace{\sqrt{\frac{1}{\epsilon^N \operatorname{det} A_N} }}_{I_0}  \exp\left(-\frac{1}{4} \alpha^2 (A_N^{-1})_{NN} \right)
\end{align*}
where $(A_N^{-1})_{NN}$ is the last diagonal element of the inverse matrix of $A_N$. Substituting back:
\begin{align*}
    \hat{I}_4^{(N)} = I_0 \int_{\mathbb{R}} \frac{\dd{\alpha}}{2 \pi} \exp\left(i \alpha x - \frac{1}{4} \alpha^2 (A_N^{-1})_{NN} \right)
\end{align*}
which is again a gaussian integral, and following formula (\ref{eqn:gaussian-integral1}) with $a = (A_N^{-1})_{NN}/4$ and $b = -x$ leads to: 
\begin{align}
    \hat{I}_4^{(N)} = \frac{I_0}{\bcancel{2} \pi} \sqrt{\frac{\bcancel{4} \pi}{(A_N^{-1})_{NN}} } \exp\left(-\frac{x^2}{(A_N^{-1})_{NN}} \right) = \frac{I_0}{\sqrt{ \pi (A_N^{-1})_{NN}}} \exp\left(-\frac{x^2}{(A_N^{-1})_{NN}} \right) 
    \label{eqn:latestI4N}
\end{align}

All that's left is to compute $(A_N^{-1})_{NN}$ and take the continuum limit. Recall from linear algebra that:
\begin{align*}
    A^{-1} = \frac{1}{\operatorname{det}A } C_{ji} 
\end{align*} 
where $C_{ij}$ are the \textit{cofactors} of $A$, i.e. the determinants of the matrices obtained from $A$ by removing the $i$-th row and $j$-th column. In our case:
\begin{align*}
    (A_N^{-1})_{NN} = \frac{C_{NN}}{\operatorname{det} A_N} 
\end{align*}      
Before, we obtained $\operatorname{det} A_N$ by means of $D_k^{(N)}$, i.e. the determinants of the matrices obtained by removing the first $k-1$ rows and columns, so that $D_1^{(N)} = \epsilon^N \operatorname{det} A_N$. This leads to:
\begin{align*}
    (A_N^{-1})_{NN} = \frac{\epsilon^N}{D_1^{(N)}} C_{NN} 
\end{align*}
For $C_{NN}$ we have to compute the determinant of the $(N-1)\times (N-1)$  matrix $A_*^{(N-1)}$, obtained by removing the last row and column from $A_N$. Note that $A_*^{(N-1)} \neq A^{(N-1)}$, as they differ for the \textit{last diagonal element} which is:
\begin{align}
    (A_*^{(N-1)})_{N-1,N-1} = P_{N-1} \epsilon + \hlc{Yellow}{\frac{2}{\epsilon}} \neq (A^{(N-1)}_{N-1,N-1}) = P_{N-1} \epsilon + \hlc{Yellow}{\frac{1}{\epsilon}}  
    \label{eqn:element-difference}
\end{align} 
We proceed in a similar manner, defining $\hat{D}_k^{(N-1)}$ to be the determinant of the matrix obtained by removing the first $k-1$ rows and columns from $\epsilon A_*^{(N-1)}$ (again, we multiply by $\epsilon$ to remove denominators):
\begin{align*}
    \hat{D}_{k}^{(N-1)} = \left|\begin{array}{ccccc}
        \epsilon a_k & -1 & 0 & \dots & 0 \\ 
        -1 & \epsilon a_{k-1} & -1 & 0 & \vdots \\ 
        0 & \ddots & \ddots & \ddots & \vdots \\ 
        \vdots & \ddots & -1 & \epsilon a_{N-2} & -1 \\ 
        0 & \dots & \dots & -1 & \epsilon a_{N-1}
        \end{array}\right|
\end{align*}

So $\hat{D}_1^{(N-1)} = \epsilon^{N-1} \operatorname{det} A_*^{(N-1)} = \epsilon^{N-1} C_{NN}$ leading to:
\begin{align*}
    (A_N^{-1})_{NN} = \frac{\epsilon^N}{D_1^{(N)}} \frac{1}{\epsilon^{N-1}} \hat{D}_1^{(N-1)} = \epsilon \frac{\hat{D}_1^{(N-1)}}{D_1^{(N)}}   
\end{align*}
For simplicity, it is convenient to define $\tilde{D}_1^{(N-1)} \equiv \epsilon \hat{D}_1^{(N-1)}$, so that:
\begin{align}
    (A_N^{-1})_{NN} = \frac{\tilde{D}_1^{(N-1)}}{D_1^{(N)}} 
    \label{eqn:latestelement}
\end{align}
Repeating the steps for the continuum limit, we get the same differential equation for $\tilde{D}(\tau)$:
\begin{align*}
    \partial_\tau^2 \tilde{D}(\tau) = P(\tau) D(\tau)
\end{align*} 
However, due to (\ref{eqn:element-difference}), the boundary conditions are now different:
\begin{align*}
    \tilde{D}^{(N-1)}_{N-1} &= \epsilon(\epsilon^2 P_{N-1} + 2) = P_{N-1} \epsilon^3 + 2 \epsilon  \xrightarrow[\epsilon \to 0]{} 0 = \tilde{D}(t)\\
    \tilde{D}^{(N-1)}_{N-2} &= \epsilon \left|\begin{array}{cc}
    P_{N-2}\epsilon^2 + 2 & -1 \\ 
    -1 & P_{N-1}\epsilon^2 + 2
    \end{array}\right| = \epsilon(P_{N-1}P_{N-2} \epsilon^4 + 2(p_{N-1}+P_{N-2})\epsilon^2 + 3)\\
    \frac{\tilde{D}_{N-1}^{(N-1)}-\tilde{D}_{N-2}^{(N-1)}}{\epsilon} &= -1 + O(\epsilon^2)  \xrightarrow[\epsilon \to 0]{} -1 = \dv{\tilde{D}(\tau)}{\tau}\Big|_{\tau = t}
\end{align*}

Then, substituting (\ref{eqn:latestelement}) in (\ref{eqn:latestI4N}) we get:
\begin{align} \nonumber
    \hat{I}_4^{(N)} &= \frac{I_0}{\sqrt{\pi (A_N^{-1})_{NN}}} \exp\left(-x^2\frac{D_1^{(N)}}{\tilde{D}_1^{(N-1)}} \right) \qquad I_0 = \frac{1}{\sqrt{\epsilon^N \operatorname{det}A_N  }} = \frac{1}{\sqrt{ D_1^{(N)}}}\\
    I_4 &= \lim_{N \to\infty } \hat{I}_4^{(N)} = \frac{1}{\sqrt{\pi \tilde{D}(0)}} \exp\left(-x^2 \frac{D(0)}{\tilde{D}(0)} \right) 
    \label{eqn:hatI4}
\end{align}

Where $D(\tau)$ and $\tilde{D}(\tau)$ are solutions of the following differential equations with the following boundary conditions:
\begin{align*}
    \tilde{D}''(\tau) &= P(\tau) \tilde{D}(\tau)
    && \begin{cases}
        \tilde{D}(t) = 0\\
        \dv{\tilde{D}(\tau)}{\tau}\Big|_{\tau = t} = -1
    \end{cases}\\
    D''(\tau) &= P(\tau) D(\tau) &&
    \begin{cases}
        D(t) = 1\\
        \dv{D(\tau)}{\tau}\Big|_{\tau=t} = 0
    \end{cases}
\end{align*}

\begin{example}[$P(\tau) = k^2$ with fixed end-point]
    Let $P(\tau) = k^2$, with $k \in \mathbb{R}$ constant. We already solved the equation for $D(\tau)$ with the right boundary conditions in (\ref{eqn:Dtau1}):
    \begin{align*}
        D(\tau) = \cosh(k(t-\tau))
    \end{align*}    
    For $\tilde{D}(\tau)$ we start from the general integral (\ref{eqn:Dtau}) and impose the appropriate boundary conditions:
    \begin{align*}
        \begin{cases}
            \tilde{D}(t) = \tilde{A} e^{kt} + \tilde{B}e^{-kt} = 0 & (a)\\
            \dv{\tilde{D}(\tau)}{\tau}\Big|_{\tau= t} = k(\tilde{A} e^{kt}-\tilde{B}e^{-kt}) = -1 & (b)
        \end{cases}
    \end{align*} 
    so that:
    \begin{align*}
        k(a) + (b) &\colon 2 \tilde{A} k e^{kt} = -1 \Rightarrow \tilde{A} = -\frac{1}{2k} e^{-kt}\\
        k(a) - (b) &\colon 2 B ke^{-kt} = 1 \Rightarrow \tilde{B} = \frac{1}{2k} e^{kt} 
    \end{align*}
    leading to the solution:
    \begin{align*}
        \tilde{D}(\tau) = \frac{1}{2k}(e^{k(t-\tau)} - e^{-k(t-\tau)}) = \frac{1}{k} \sinh(k(t-\tau))  
    \end{align*}
    Finally, using the result we found in (\ref{eqn:hatI4}):
    \begin{align*}
        \langle \exp\left(-k^2 \int_0^t x^2(\tau) \dd{\tau} \delta(x - x(t))\right) \rangle_w &= \int_{\mathcal{C}\{0,0;x_t,t\}} \exp\left(-k^2 \int_0^t x^2(\tau) \dd{\tau}\right) \dd{_Wx(\tau)} =\\
        &= \sqrt{\frac{k}{\sinh(kt)} } \exp(-k x_t^2 \coth(kt))
    \end{align*}


    



\end{example}

\end{document}

