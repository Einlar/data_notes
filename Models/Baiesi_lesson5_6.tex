%&latex
%
\documentclass[../template.tex]{subfiles}
\begin{document}

\chapter{Ising Model and disorder}

Introduction

\section{Mean Field Ising Model}
To be completed

\section{Random Field Ising Model}
Suppose that the spins are not immersed in the same field $h$, but each of them experiences a \textit{local field} $h_i$, where $h_i$ is sampled at random from a certain distribution $p(h)$. For a given realization $\bm{h}$ of the local fields, the system's energy in the state $\bm{S}$ is given by:\marginpar{\vspace{2.5em}RFIM energy}
\begin{align}\label{eqn:disordered-energy}
    H_{\bm{h}}[\bm{S}] = -\frac{J}{N} \sum_{i,j=1}^N S_i S_j - \sum_{i=1}^N h_i S_i
\end{align}

The system is now \textbf{disordered},\marginpar{Quenched disorder} as it has no symmetries nor uniformities. As the $h_i$ are fixed once and for all, i.e. they do not change over time, we call the disorder \textbf{quenched}.\index{Quenched disorder}  

\medskip

For simplicity, we choose the disorder distribution $p(h)$ to be gaussian:
\begin{align} \label{eqn:disorder-dist}
    p(h_i) = \frac{1}{\sqrt{2 \pi \delta^2}} \exp\left(-\frac{h_i^2}{2 \delta^2} \right) \qquad \forall i=1,\dots,N
\end{align}
We suppose that, for any given temperature $T$, we can choose a disorder \q{strong enough} (i.e. a $\delta$ high enough) so that (for a random choice of the disorder) the spins will (on average) \textbf{not} align all in the same direction - meaning that the local fields \textit{prevail} over the spin-spin interactions. Surprisingly, we can verify this fact \textit{analytically}.   

\medskip

Our aim is to compute the average value of the \textbf{Helmholtz free energy} $F$, over all possible choices for the \textit{disorder} $\bm{h}$:
\begin{align}\label{eqn:mean-free-energy}
    \bar{F} = -\frac{1}{\beta}  \overline{\log Z_{\bm{h}}} =  - \frac{1}{\beta}  \int_{\mathbb{R}^N} \left(\prod_{i=1}^N \dd{h_i} p(h_i)\right) \log Z_{\bm{h}} 
\end{align} 
where we used the \textbf{independence} of the local fields to factorize the average. 

\medskip

The presence of the logarithm makes (\ref{eqn:mean-free-energy}) very difficult to compute. To overcome this problem, we introduce the \textbf{replica trick}\index{Replica trick}\marginpar{1. Replica trick} - a \textit{not totally mathematically rigorous} argument which, at the end, will still lead to the correct result.

We start by rewriting $\log Z$ as follows:
\begin{align} \label{eqn:replica-log}
    \overline{\log Z} = \pdv{n} \overline{Z^n} \Big|_{n=0}
\end{align}
In fact:
\begin{align*}
    \pdv{n} \overline{Z^n} \Big|_{n=0} = \pdv{n} \exp(n \overline{\log Z}) \Big|_{n = 0} = \exp(n \overline{\log Z}) \overline{\log Z} \Big|_{n=0} = \overline{\log Z}
\end{align*}

%Geometric interpretation?

\begin{expl} \textbf{Other equivalent expressions}.  
    To do
\end{expl}

The idea is that $Z^n$, with $n$ integer, has a physical interpretation: it is the partition function for a system made of $n$ replicas of the original one, i.e. $n$ non-interacting \textit{copies}, each with the \textbf{same} quenched disorder (but, in general, different spin states). The replica trick involves computing $\overline{Z^n}$ assuming $n$ integer, and then take the limit $n \to 0$ of its derivative \textit{as if} $n$ could vary continuously. Of course this is a problematic step - and in fact it produces unphysical results (such as a negative entropy) in some cases. However, in the RFIM all works out nicely, leading to the same results that can be obtained with much more involved methods.

\medskip

So, let's denote with $a = 1, \dots, n$ each replica. The spin-state of the $a$-th replica is $\bm{S^a} = \{S_1^a, \dots, S_N^a\}$, and the state of the entire system is denoted with $\{\bm{S^a}\} = \{\bm{S^1}, \dots, \bm{S^n}\} = \{\bm{S}\}^{\otimes n}$. Then:
\begin{align} \nonumber
    \overline{Z^n} &= \overline{\sum_{\{\bm{S^a}\}} \prod_{a=1}^n e^{-\beta H[\bm{S^a}]}} =\\ \nonumber
    &\underset{(\ref{eqn:disordered-energy})}{=} \overline{\sum_{\{\bm{S^a}\}} \exp\left(\frac{\beta J}{N} \sum_{a=1}^n \sum_{i,j=1}^N S_i^a S_j^a \right) \exp\left(\beta \sum_{i=1}^N \sum_{a=1}^n S_i^a h_i\right)}
    \shortintertext{Note that only the $h_i$ are disordered, and so, by linearity of the expectation, we can move the average only to the last term:}
    &=\sum_{\{\bm{S^a}\}} \exp\left(\frac{\beta J}{N} \sum_{a=1}^n \sum_{i,j=1}^N S_i^a S_j^a \right) \overline{\exp(\hlc{Yellow}{\beta} \hlc{Yellow}{\sum_{a=1}^n} \sum_{i=1}^N \hlc{Yellow}{S_i^a }h_i)} \label{eqn:st1}
\end{align}
To compute the disorder average,\marginpar{2. Disorder average} we first \textit{isolate} the $h_i$. The pre-factor for each of them is: 
We then define:
\begin{align} \label{eqn:lambdai}
    \lambda_i = \hlc{Yellow}{\beta \sum_{a=1}^n S_i^a}
\end{align}
And so the last term becomes:
\begin{align*}
    \overline{\exp\left(\beta \sum_{a=1}^n \sum_{i=1}^N S_i^a h_i\right)} = \overline{\exp\left(\sum_{i=1}^N \lambda_i h_i\right)} = \prod_{i=1}^N \overline{e^{\lambda_i h_i}}
\end{align*}
where the average factorizes because the $h_i$ are \textbf{independent.} 

\medskip

All that's left is to compute the gaussian integral:
\begin{align*}
    \overline{e^{\lambda_i h_i}} &= \int_{\mathbb{R}} \dd{h_i} p(h_i) e^{\lambda_i h_i} \underset{(\ref{eqn:disorder-dist})}{=}  \int_{\mathbb{R}} \dd{h_i} \frac{1}{\sqrt{2 \pi \delta^2}} \exp\left(-\frac{h_i^2}{2 \delta^2} + \lambda_i h_i \right) =\\
    &= \exp\left(\frac{\delta^2 \lambda_i^2}{2} \right) \underset{(\ref{eqn:lambdai})}{=} \exp\left(\frac{\beta^2 \delta^2}{2} \left[\sum_{a=1}^n S_i^a\right]^2 \right)
\end{align*}
Substituting back in (\ref{eqn:st1}) we get:
\begin{align} \nonumber
    \overline{Z^n} &= \sum_{\{\bm{S^a}\}} \exp\left(\frac{\beta J}{N} \sum_{a=1}^n \sum_{i,j=1}^N S_i^a S_j^a \right) \exp\left(\frac{\beta^2 \delta^2}{2} \sum_{i=1}^N \left[\sum_{a=1}^n S_i^a\right]^2 \right) =
    \shortintertext{We then rewrite the sum $\sum_{ij} = (\sum_i)^2$ and merge the exponentials:}
    &= \sum_{\{\bm{S^a}\}} \exp\left(\frac{\beta J}{N} \sum_{a=1}^n \left[\sum_{i=1}^N S_i^a\right]^2 + \frac{\beta^2 \delta^2}{2} \sum_{i=1}^N \left[\sum_{a=1}^n S_i^a\right]^2  \right) \label{eqn:st2}
\end{align}
Note that there are now terms $S_i^a S_i^b$ with $a \neq b$ originating \marginpar{Coupled replicas}from the square $(\sum_a S_i^a)^2$. So, after averaging over disorder, \textit{corresponding} spins from different replicas behave like if they were \q{close together} and \textbf{interacting}, meaning that replicas are not anymore independent from each other. This should be expected, as they are subject to the \textit{same quenched local fields} $h_i$.

\medskip

We then proceed like we did in the Mean Field Ising Model,\marginpar{3. Disentangling spins with H-S transform} by \q{disentangling} spins and focusing on each of them as if they were independent. The only term in (\ref{eqn:st2}) where spins are coupled is $(\sum_i S_i^a)^2$, because it produces mixed terms like $S_i^a S_j^a$ with $i\neq j$. The idea is to reinterpret this pairwise interaction as the interaction with a \textit{common field}. Mathematically, this is done through the \textbf{Hubbard-Stratonovich transformation}\index{Hubbard Stratonovich transformation}, which is just a gaussian integral \q{done in reverse}:
\begin{align} \label{eqn:HS}
    \exp\left(\frac{b}{2} z^2 \right) = \frac{1}{\sqrt{2 \pi b}} \int_{\mathbb{R}} \dd{x} \exp\left(-\frac{x^2}{2b} \pm zx \right)
\end{align}
Note that it can be used to move the square from $z$ to an auxiliary variable $x$, representing the \textit{common field}. In our case we apply it to the first exponential term in (\ref{eqn:st2}):
\begin{align*}
    \exp\left(\frac{\beta J}{N} \sum_{a=1}^n \left[\sum_{i=1}^N S_i^a\right]^2 \right) &= \prod_{a=1}^n\exp\left(\frac{\textcolor{Red}{2} \beta J}{\textcolor{Red}{2} N} \left[\sum_{i=1}^N S_i^a\right]^2 \right) = \prod_{a=1}^n \exp\left(\frac{b}{2} z_a^2 \right) =\\
    &\underset{(\ref{eqn:HS})}{=} \prod_{a=1}^n\int_{\mathbb{R}}\frac{1}{\sqrt{2 \pi b}} \dd{x_a} \exp\left(-\frac{x_a^2}{2b} + z x_a \right)  \\
        z_a &= \sqrt{2 \beta J} \left[\sum_{i=1}^N S_i^a \right]; \qquad b = \frac{1}{N} 
\end{align*}
Leading to:
\begin{align*}
    \overline{Z^n} &= \sum_{\{\bm{S^a}\}} \int_{\mathbb{R}^n} \left(\prod_{a=1}^n \dd{x_a}\right) \left(\frac{N}{2 \pi} \right)^{n/2} \exp\left(-\frac{N}{2} \sum_{a=1}^n x_a^2 + \sqrt{2 \beta J} \sum_{a=1}^n \sum_{i=1}^N S_i^a x_a \right) \cdot\\
    &\qquad \cdot \exp\left(\frac{\beta^2 \delta^2}{2} \sum_{i=1}^N \left[\sum_{a=1}^n S_i^a\right]^2 \right)
\shortintertext{Now all spins are \textit{independent} from each other (there are no mixed terms $S_i^a S_j^a$). So we isolate each $S_i$:}
    &=\sum_{\{\bm{S^a}\}}  \int_{\mathbb{R}^n} \left(\prod_{a=1}^n \dd{x_a}\right) \left(\frac{N}{2 \pi} \right)^{n/2} \exp\left(-\frac{N}{2} \sum_{a=1}^n x_a^2 \right)\cdot \\
    &\qquad\cdot \prod_{i=1}^N    \exp\left( \sqrt{2 \beta J} \sum_{a=1}^n S_i^a x_a  + \frac{\beta^2 \delta^2}{2} \left[\sum_{a=1}^n S_i^a\right]^2\right)
\shortintertext{By linearity of integration we can bring the sum over all states inside the integral, and factor the first exponential:}
&= \int_{\mathbb{R}^n} \left(\prod_{a=1}^n \dd{x_a}\right) \left(\frac{N}{2 \pi} \right)^{n/2} \exp\left(-\frac{N}{2} \sum_{a=1}^n x_a^2 \right)\cdot \\
&\qquad\cdot \sum_{\{\bm{S^a}\}}  \prod_{i=1}^N    \exp\left( \sqrt{2 \beta J} \sum_{a=1}^n S_i^a x_a  + \frac{\beta^2 \delta^2}{2} \left[\sum_{a=1}^n S_i^a\right]^2\right)
\end{align*}  
As the $S_i$ are independent, the sum over all possible states factorizes:
\begin{align*}
    \sum_{\{\bm{S^a}\}}  \prod_{i=1}^N    \exp\left( \sqrt{2 \beta J} \sum_{a=1}^n S_i^a x_a  + \frac{\beta^2 \delta^2}{2} \left[\sum_{a=1}^n S_i^a\right]^2\right) = \span \\
    &= \prod_{i=1}^N \underbrace{\sum_{\substack{\{S^a_i = \pm 1\}\\a=1,\dots,n}} \exp\left( \sqrt{2 \beta J} \sum_{a=1}^n S_i^a x_a  + \frac{\beta^2 \delta^2}{2} \left[\sum_{a=1}^n S_i^a\right]^2\right)}_{Z_i(x_1, \dots, x_n)} 
\end{align*}

\begin{expl} To make the last passage clearer, we can rewrite the sum over all states as a rescaled average, and then apply in (*) the independence of $S_i$ to factorize the average:
    \begin{align*}
        \sum_{\{\bm{S^a}\}} \prod_{i=1}^N f(S_i^a) = \underbrace{|\{\bm{S^a}\} |}_{2^{Nn}} \langle \prod_{i=1}^N f(S_i^a) \rangle \underset{(*)}{=}  2^{Nn} \langle f(S_1^a) \rangle \cdots \langle f(S_N^a) \rangle
    \end{align*} 
    where the averages in the last step are over the \textit{replicas} (i.e. the $a$ index).    
\end{expl}


As we are summing over \textit{every possible state} of the entire system, each $Z_i(x_1, \dots, x_n)$ is the same, i.e. $Z_i(x_1, \dots, x_n) \equiv Z_1(\bm{x_a})$. So we have a product of $N$ equal terms, leading to a $Z_1^N = \exp(N \log Z_1)$, that we then bring inside the exponential:
\begin{align} \label{eqn:st3}
    \overline{Z^n} &=  \left(\frac{N}{2 \pi} \right)^{n/2} \int_{\mathbb{R}^n} \left(\prod_{a=1}^n \dd{x_a}\right) \exp\left[N\left(-\frac{1}{2} \sum_{a=1}^n x_a^2 + \log Z_1(\bm{x_a}) \right)\right]
\end{align}
$Z_1$ represents the partition function for a \textit{single spin} (replicated $n$ times). So, effectively, we succeeded in rewriting the RFIM as a system of non-interacting spins, as we did in the MFIM, but at the cost of introducing replicas.

\medskip

To proceed, we drop the $i$ subscript in $Z_1$, and write:
\begin{align} \nonumber
    Z_1(\bm{x_a}) &= \sum_{\substack{\{S^a = \pm 1\}\\a=1,\dots,n}} \exp\left( \sqrt{2 \beta J} \sum_{a=1}^n S^a x_a  + \frac{\beta^2 \delta^2}{2} \left[\sum_{a=1}^n S^a\right]^2\right) \equiv  \sum_{\substack{\{S^a = \pm 1\}\\a=1,\dots,n}} e^{A[\bm{S}, \bm{x_a}]}\\
    A[\bm{S}, \bm{x_a}] &\equiv \sqrt{2 \beta J} \sum_{a=1}^n S^a x_a + \frac{\beta^2 \delta^2}{2} \left[\sum_{a=1}^n S^a\right]^2 \label{eqn:Aterm}
\end{align}
where $S^a$ is an arbitrary spin in the $a$-th replica - as we are summing over every combination, its position does not matter (however, in the sum over all replicas we pick always corresponding positions). 

\medskip

We assume that (in the thermodynamic limit) no replica is \q{special}, meaning that all the fields $x_a$ are the \textit{same} $x$ (\textbf{full replica symmetry}). Then:\marginpar{4. Full replica symmetry}
\begin{align*}
    \sum_{a=1}^n x_a = nx; \qquad \sum_{a=1}^n x_a^2 = n x^2
\end{align*} 
Substituting in (\ref{eqn:st3}), now all the integrands are the same:
\begin{align}\label{eqn:st4}
    \overline{Z^n} = \left(\frac{N}{2 \pi} \right)^{n/2} \left[\int_{\mathbb{R}} \dd{x}  \exp\left(N \left[-\frac{1}{2} n x^2 + \log Z_1(x) \right]\right)\right]^n
\end{align}
In the limit $N \to \infty$, we can use the saddle-point approximation.\marginpar{Saddle-point integration} So we stationarize the argument:
\begin{align*}
    \pdv{x} \left(-\frac{1}{2} n x^2 + \log Z_1(x) \right) = 0 \Rightarrow nx = \pdv{x} \log Z_1(x)
\end{align*}
The solution $x_m$ satisfies:
\begin{align*}
    n x_m &= \frac{1}{\displaystyle\sum_{\substack{\{S^a = \pm 1\}\\a=1,\dots,n}} e^{A[S, x]}} \sum_{\substack{\{S^a = \pm 1\}\\a=1,\dots,n}} e^{A[S,x_m]} \underbrace{\pdv{x} A[S,x]}_{\sqrt{2 \beta J} \sum_a S^a} \Big|_{x=x_m}  =\\
    &= \sqrt{2 \beta J}\frac{\displaystyle  \sum_{\substack{\{S^a = \pm 1\}\\a=1,\dots,n}} e^{A[S, x_m]} \left(\sum_{a=1}^n S^a\right)}{\displaystyle \underbrace{\sum_{\substack{\{S^a = \pm 1\}\\a=1,\dots,n}} e^{A[S, x_m]}}_{Z_1(x_m)}} = \langle \sqrt{2 \beta J} \sum_{a=1}^n S^a \rangle
\end{align*}
where we recognized the canonical ensemble average with partition function $Z_1(x_m)$. We interpret the average spin over replicas as the system's \textbf{magnetization} $m$:\marginpar{\vspace{5em}Magnetization $m$}
\begin{align} \label{eqn:magnetization}
    \frac{x_m}{\sqrt{2 \beta J}} = \langle \frac{1}{n} \sum_{a=1}^n S^a  \rangle \equiv m = \frac{\displaystyle \sum_{\substack{\{S^a = \pm 1\}\\a=1,\dots,n}} e^{A[S, x_m]} \left(\frac{1}{n} \sum_{a=1}^n S^a \right)}{Z_1(x_m)} 
\end{align}

\begin{expl}
    The magnetization $m$ is here defined as the average of replicated spins. In other words, consider any arbitrary position $i$ (with $1 \leq i \leq N$), and take the average value of the spins lying in that position in every replica:
    \begin{align*}
        m_{\mathrm{R}} = \frac{1}{n} \sum_{a = 1}^n S_i^a
    \end{align*}
    At this stage, each spin locus is \textit{independent} from all others, so the choice of $i$ does not matter - and that's why we omit the $i$ subscript.

    \medskip

    Note that this is a \textit{generalization} of the magnetization in the MFIM, where we do not need replicas:
    \begin{align*}
        m_{\mathrm{M}} = \frac{1}{N} \sum_{i=1}^N S_i
    \end{align*} 
\end{expl} %Is this magnetization a physical (measurable) quantity in the RFIM?

Then, ignoring all the pre-factors, the integral in (\ref{eqn:st4}) becomes:
\begin{align} \nonumber
    \overline{Z^m} \propto \exp\left[N \left(-\frac{1}{2} n x_m^2 + \log Z_1(x_m) \right)\right]
    \shortintertext{We then express everything in terms of $m$. Substituting $x_m^2 = 2 \beta J m^2$ leads to:}
    \overline{Z^n} \propto \exp\left[N \left(-n \beta J m^2 + \log Z_1(m)\right)\right] \label{eqn:Znm}
\end{align}

Note that if we express $x_m$ in terms of $m$ in the rhs of (\ref{eqn:magnetization}) we get a self-consistent equation for $m$. So we proceed to do this for $Z_1$ and (\ref{eqn:Aterm}):
\begin{align*}
    Z_1(m) &= \sum_{\substack{\{S^a = \pm 1\}\\a=1,\dots,n}} e^{A[\bm{S}, m]}\\
    A[\bm{S},m] &= 2 \beta J m \left(\sum_{a=1}^n S^a\right) + \frac{\beta^2 \delta^2}{2} \left[\sum_{a=1}^n S^a\right]^2
\end{align*}

To go further,\marginpar{5. Decoupling replicas with H-S transform} we need to get rid of the quadratic sum $(\sum_a S^a)^2$. If we do this, we could factor a $\sum_a S^a$, so that $A[\bm{S},m] = (\sum_a S^a) B$ and write $Z_1$ as a product of exponentials $\prod_a e^{S_a B}$, and finally bring the sum over all $S^a$ inside. In other words, we need a way to make replicas \textit{decoupled}. This can be done, again, by introducing a common field $\nu$ and doing a Hubbard-Stratonovich trasformation.
We start from:
\begin{align*}
    e^{A[\bm{S},m]} = \exp\left(2 \beta J m \left[\sum_{a=1}^n S^a\right] \right) \exp\left(\frac{\beta^2 \delta^2}{2} \left[\sum_{a=1}^n S^a\right]^2 \right)
\end{align*} 
then rewrite the second exponential as:
\begin{align*}
    \exp\left(\frac{\beta^2 \delta^2}{2} \left[\sum_{a=1}^n S^a\right]^2 \right) = \exp\left(\frac{b}{2} z^2 \right) = \frac{1}{\sqrt{2 \pi b}} \int_{\mathbb{R}} \dd{\nu} \exp\left(-\frac{\nu^2}{2b} + z \nu \right) \\
    z = \beta \delta \sum_{a=1}^n S^a; \qquad b= 1 \span
\end{align*}
leading to:
\begin{align} \nonumber
    Z_1(m) &= \sum_{\substack{\{S^a = \pm 1\}\\a=1,\dots,n}} e^{A[\bm{S},m]} =\\ \nonumber
    &= \sum_{\substack{\{S^a = \pm 1\}\\a=1,\dots,n}}\exp\left(2 \beta J m \left[\sum_{a=1}^n S^a\right]\right) \int_{\mathbb{R}} \frac{\dd{\nu}}{\sqrt{2 \pi}} \exp\left(-\frac{1}{2} \nu^2 + \beta \delta \nu \left[\sum_{a=1}^n S^a\right]  \right) =\\ \nonumber
    &= \sum_{\substack{\{S^a = \pm 1\}\\a=1,\dots,n}}\int_{\mathbb{R}} \frac{\dd{\nu}}{\sqrt{2 \pi}} \exp\left(-\frac{1}{2} \nu^2 + \left[\sum_{a=1}^n S^a\right] (2 \beta J m + \beta \delta \nu) \right) =\\ \nonumber
    &= \sum_{\substack{\{S^a = \pm 1\}\\a=1,\dots,n}}\int_{\mathbb{R}} \frac{\dd{\nu}}{\sqrt{2 \pi}} \exp\left(-\frac{1}{2} \nu^2 \right)  \prod_{a=1}^n \exp\left[(2 \beta J m + \beta \delta \nu) S_a\right]
    \shortintertext{As the replicas are independent, we can \q{factor the average}, i.e. bring the sum inside the product as we did before:}
    &= \int_{\mathbb{R}} \frac{\dd{\nu}}{\sqrt{2 \pi}}  \exp\left(-\frac{1}{2} \nu^2 \right) \prod_{a=1}^n \sum_{S^a = \pm 1} \exp\left[(2 \beta J m + \beta \delta \nu) S_a\right] = \nonumber
    \shortintertext{Now the sum is over only two states, and so we can compute it:} \nonumber
    Z_1(m)&= \int_{\mathbb{R}} \frac{\dd{\nu}}{\sqrt{2 \pi}} \exp\left(-\frac{1}{2} \nu^2 \right) \left[\frac{e^{2 \beta J m + \beta \delta \nu} + e^{-(2 \beta J m + \beta \delta \nu)}}{\textcolor{Red}{2}} \textcolor{Red}{2} \right]^n =\\ \nonumber
    &= \int_{\mathbb{R}} \frac{\dd{\nu}}{\sqrt{2 \pi}}  \exp\left(-\frac{1}{2} \nu^2 \right) [2 \cosh (2 \beta J m + \beta \delta \nu)]^n =\\
    &= \int_{\mathbb{R}} \frac{\dd{\nu}}{\sqrt{2 \pi}} \exp\left(-\frac{1}{2} \nu^2 + n \log[2 \cosh (2 \beta J m+ \beta \delta \nu)] \right)
    \label{eqn:z1m}
\end{align}
We can finally compute the \textbf{free energy}:\marginpar{Free energy for the RFIM}
\begin{align*}
    \bar{F} &\underset{(\ref{eqn:mean-free-energy})}{=}  - \frac{1}{\beta}  \overline{\log Z} \underset{(\ref{eqn:replica-log})}{=}  -\frac{1}{\beta}  \pdv{n} \overline{Z^n} \Big|_{n = 0} \underset{(\ref{eqn:Znm})}{\approx}  -\frac{1}{\beta} \pdv{n} \exp\left[N \left(- n \beta J m^2 + \log Z_1(m) \right)\right]_{n=0} =\\
    &= -\frac{1}{\beta} \exp(N[-n \beta J m^2 + \log Z_1(m)])\left(-N \beta J m^2 + \frac{N}{Z_1(m)} \pdv{n} Z_1 \right) \Big|_{n=0}
\end{align*} 
Note that, for $n=0$, $Z_1(m)$ becomes:
\begin{align*}
    Z_1(m) \underset{n \to 0}{=}  \int_{\mathbb{R}} \frac{\dd{\nu}}{\sqrt{2 \pi}}  \exp\left(-\frac{1}{2} \nu^2 \right) = 1
\end{align*}
and so the first exponential is:
\begin{align*}
    \exp \Big(N[\cancel{-n \beta J m^2} + \underbrace{\log Z_1(m)}_{0} ] \Big) \Big|_{n = 0} = 1
\end{align*}
Leading to:
\begin{align*}
    \bar{F} &= -\frac{N}{\cancel{\beta}} \left[-\cancel{ \beta} J m^2 + \frac{\textcolor{Red}{\cancel{\beta}}}{\textcolor{Red}{\beta} Z_1} \pdv{n} Z_1 \right]_{n=0} = N \left[Jm^2 - \frac{1}{\beta} \pdv{n} Z_1(m)\right]_{n=0} =\\
    &\underset{(\ref{eqn:z1m})}{=} N\left[J m^2 - \frac{1}{\beta} \pdv{n} \int_{\mathbb{R}} \frac{\dd{\nu}}{\sqrt{2 \pi}} \exp\left(-\frac{1}{2} \nu^2 + n \log[2 \cosh (2 \beta J m+ \beta \delta \nu)] \right) \right]_{n=0} =\\
    &=N\left[Jm^2 - \frac{1}{\beta} \int_{\mathbb{R}} \frac{\dd{\nu}}{\sqrt{2 \pi}} \exp\left(-\frac{1}{2}\nu^2 \right) \log[2 \cosh (2 \beta J m + \beta \delta \nu)] \right]
    \shortintertext{Finally, we extract the disorder amplitude $\delta$ from the field $\nu$ with a change of variables $h= \delta \nu$, allowing to explicitly see $\bar{F}$ as an average over disorder:}
    &= N\left[Jm^2 - \frac{1}{\beta} \int_{\mathbb{R}} \frac{\dd{h}}{\sqrt{2 \pi \delta^2}} \exp\left(-\frac{h^2}{2 \delta^2} \right) \log[2 \cosh(\beta (2 Jm +h))]  \right]
\end{align*}

%Missing: self-consistent equation and phase-space discussion

%Disegna le varie connessioni durante gli step

\end{document}
