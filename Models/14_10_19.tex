%&latex
%
\documentclass[../template.tex]{subfiles}
\begin{document}

\section{The continuum limit - part 2}
\lesson{4}{14/10/19}

\textbf{Summary of the last lessons}.
Analysing the dynamics of the Diffusion Problem led to the Master Equation, which in the symmetrical case reduces to:
\begin{align}
    w_i(t_{n+1}) = \frac{1}{2}(w_{i-1}(t_n) + w_{i+1}(t_n)) 
    \label{eqn:master_equation}
\end{align}
where $w_i(t_n)$ is the (mass) probability function for a particle, i.e. $w_i(t_n)$ is the probability that a particle will be at position $x_i = i\cdot l$ at time $t_n = n \cdot \epsilon$.\\
Further evaluation, in terms of the Binomial distribution, leads to the exact solution:
\begin{align}
    w_i(t_n) = \frac{1}{2^n} {n\choose n_+} 
    \label{eqn:discrete_sol}
\end{align}
where $n_+$ is a random variable representing the number of steps in which the particle moves to the \textit{right}.\\
Then we considered the continuum limit, with $l, \epsilon \downarrow 0$ (keeping $l^2 \epsilon^{-1} = 2D$ constant), arriving to:
\begin{align*}
    w_i(t_n) = \frac{1}{\sqrt{4 \pi D t}} \exp\left(-\frac{x^2}{4 D t} \right) 
\end{align*} 
which is the pdf for a particle that starts at $x=0$ at $t=0$.\\  

The same result can be obtained by taking directly the continuum limit of the ME (\ref{eqn:master_equation}), which will lead to a differential equation that can
\textit{then} be solved.\\
We start by recalling that, for a fine discretization, $w_i(t_n)$ is approximately equal to the probability of being around a generic $(x,t)$ (i.e. $W(x,t)\Delta x$), up to a normalization constant:
\begin{align*}
    W(x_0,t_n)\Delta x = \mathbb{P}(x \in [x_0 - \Delta x/2, x_0 + \Delta x/2]) \approx \frac{\Delta x}{2l} w_{i_0}(t_n) \qquad i_0 = \lfloor \Delta x/l \rceil
\end{align*} 
And so, with a slight abuse of notation:
\begin{align*}
    w_i(t_n) \approx 2l W(x,t) \qquad i = \lfloor x/l \rceil, n = \lfloor t/\epsilon \rceil
\end{align*}
Substituting in (\ref{eqn:master_equation})
leads to:
\begin{align}
    \cancel{2l} W(x,t+\epsilon) = \cancel{2l}\frac{1}{2}(W(x-l,t) + W(x+l,t))
    \label{eqn:master_2} 
\end{align}
which means that an analogous Master Equation holds even for $W(x,t)$, which is a continuous pdf, and thus can be differentiated. In particular, we can compute $W(x,t+\epsilon)$ in terms of $W(x,t)$ (and derivatives) by expanding around $\epsilon = 0$:
\begin{align}
    W(x,t+ \epsilon) &= W(x,t) + \epsilon \pdv{\tau}W(\chi, \tau) \Big|_{(x,t)} + \frac{\epsilon^2}{2} \pdv[2]{\tau} W(\chi, \tau)\Big|_{(x,t)} + O(\epsilon^3) \label{eqn:time_der}
\intertext{And also $W(x \pm l, t)$ by expanding around $l = 0$:}
W(x \pm l, t) &= W(x,t) \pm l\pdv{\chi}W(\chi,\tau)\Big|_{(x,t)} + \frac{l^2}{2} \pdv[2]{\chi} W(\chi, \tau)\Big|_{x,t} + O(l^3) \label{eqn:space_der}
\end{align} 
We then introduce the following notation for the space and time derivatives:
\begin{align*}
    \dot{W}(x,t) = \pdv{\tau} W(\chi, \tau)\Big|_{(x,t)} \qquad W'(x,t) = \pdv{\chi} W(\chi, \tau)\Big|_{(x,t)}
\end{align*}
so that a space derivative is denoted with $a'$ ($a''$ for the second derivative), and a time derivative with $\dot{a}$ ($\ddot{a}$ for the second derivative).\\
We can now substitute back in (\ref{eqn:master_2}). We start with the right side:
\begin{align*}
    W(x+l,t) + W(x-l, t) = 2 W(x,t) + l^2 W''(x,t) + O(l^4)
\end{align*}
where the $O(l^4)$ is given by the cancellation of the odd powers (including $l^3$). Equating to the left side of (\ref{eqn:master_2}) leads to:  
\begin{align*}
    \cancel{W(x,t)}+ \epsilon \dot{W}(x,t) + \frac{\epsilon^2}{2} = \cancel{W(x,t)} + \frac{l^2}{2} W''(x,t) + O(l^4)  
\end{align*}
Dividing by $\epsilon$:
\begin{align*}
    \dot{W}(x,t) + \frac{\epsilon}{2} \ddot{W}(x,t) &= \underbrace{\frac{l^2}{2 \epsilon}}_{D}  W''(x,t) + O\left(\frac{l^4}{\epsilon} \right)  \\
    &= D W''(x,t) + O(4 \epsilon D^2)
\end{align*} 
If we now take the continuum limit, then $\epsilon, l \to 0$ with the ratio $D = l^2/(2\epsilon)$ fixed, and both $\ddot{W}(x,t)$ and the error term vanish, leading to:
\begin{align*}
    \partial_t W(x,t) = D \partial_x^2 W(x,t)
\end{align*}  

Recall that the solution for the discrete case was:
\begin{align*}
    \frac{1}{\sqrt{4 \pi D t}} \exp\left(-\frac{x^2}{4 D t} \right) 
\end{align*}
We want now to check if this solution can be derived from the equation in the continuum limit, without having to resort to a discretization.\\
So, we start from:
\begin{align*}
    \partial_t W(x,t) = D \partial_x^2 W(x,t)
\end{align*}
and we search a solution in the field of reals ($\mathbb{R}$), so that the boundary condition becomes:
\begin{align*}
    \lim_{|x| \to \infty} W(x,t) = 0
\end{align*}
as the probability that the particle is somewhere should be one:
\begin{align*}
    \int_{\mathbb{R}} W(x,t) \dd{x} = 1
\end{align*}

If we had a semi-infinite domain, like $[0, \infty)$, we would have to choose a certain boundary condition at $0$. For example, one could choose a \textit{reflective} boundary, so that a particle reaching $0$ cannot cross it, and remains stationary until it moves right again. Otherwise, an \textit{absorbing} boundary would \textit{remove} every particle that attempts to cross $0$.\\

Note also that $W(x,t)$ needs to be $\geq 0$, as it is a probability density. However, it is not clear if the time evolution, given by solving the differential equation, will satisfy $W(x,t) \geq 0 \quad \forall t$, if one start with a $W(x,0) > 0$.\\
Fortunately, that is guaranteed by the nature of that partial differential equation.\\

Note that the equation is translationally invariant, meaning that if $W(x,t)$ is a solution, also $W(x + y,t)$ is a solution. This is because the space coordinate appears only in a \textit{second-order derivative}.\\
This suggests a way to solve the equation, by starting from the eigenfunctions of the laplacian, i.e. the solutions of the eigenvalue equation:
\begin{align*}
    \partial_x^2 \varphi_k(x) = \lambda_k \varphi_k(x) \qquad \lambda_k \equiv -k^2
\end{align*}
which are:
\begin{align*}
    \varphi_k(x) = A_k e^{\pm i k x} \qquad k\in \mathbb{R}
\end{align*}
as can easily be checked by substitution.\\
Note that, as $k \in \mathbb{R}$, the $\pm$ is redundant and can be removed:
\begin{align*}
    \varphi_k(x) = A_k e^{ikx}
\end{align*}  
These eigenfunctions are the basis of the Fourier transform - that is every function can be expressed as a (infinite) linear combination of these $\varphi_k(x)$.\\
We choose $A_k = 1$ for simplicity, and then:
\begin{align*}
    W(x,t) = \int_{\mathbb{R}} \frac{dk}{2 \pi} e^{ikx} c_k(t) 
\end{align*}  
where the $2\pi$ factor is inserted by convention (as in Fourier transforms).\\
Recall the orthogonality relation:
\begin{align*}
    \int_{\mathbb{R}} \dd{x} \varphi_k(x)^* \varphi_{k'}(x) = \int_{\mathbb{R}} \dd{x} e^{i(k'-k)x} = 2\pi \delta(k-k')
\end{align*} 
and also:
\begin{align*}
    \int_{\mathbb{R}} \frac{\dd{k}}{2 \pi} \varphi_k(x)^* \varphi_k (x') = \dots = \delta(x-x') 
\end{align*}
So, by multiplying both sides by $e^{-ik'x}$ and integrate over $x$ we get:
\begin{align*}
    \int_{\mathbb{R}} W(x,t) e^{-ik'x} \dd{x} = \int_{\mathbb{R}} \frac{\dd{k}}{2\pi}  \int_{\mathbb{R}} \dd{x} e^{i(k-k')x} c_k(t)
\end{align*}  
and we can apply the orthogonality relation, arriving to:
\begin{align*}
    \int_{\mathbb{R}} W(x,t) e^{-ik'x}\dd{x} = \int_{\mathbb{R}} \dd{k} \delta(k-k') c_k(t) = c_{k'}(t)
\end{align*}
and so:
\begin{align*}
    c_k(t) = \int_{\mathbb{R}} \dd{x} e^{-ikx} W(x,t)
\end{align*}
Differentiating wrt $t$:
\begin{align*}
    \dot{c}_k(t) &= \int_{\mathbb{R}} \dd{x} e^{-ikx} \dot{W}(x,t) = D \int_{-\infty}^{\infty} e^{-ikx} W''(x,t) \dd{x} =  \\
    &= D W'(x,t) e^{-ikx}\big|_{-\infty}^{\infty} - D \int_{-\infty}^{\infty} \partial_x (e^{-ikx}) W'(x,t) \dd{x} = \\
    &= -D\underbrace{(\partial_x e^{-ikx})}_{-ik e^{-ikx}} W(x,t) \big|_{-\infty}^{\infty} + D\int_{\mathbb{R}} \underbrace{\partial_x^2 (e^{-ikx})}_{-k^2 e^{-ikx}} W(x,t) \dd{x} \\
\end{align*} 
And so:
\begin{align*}
    \dot{c}_k(t) = \int_{\mathbb{R}} \dd{x} e^{-ikx} \dot{W}(x,t) = - D k^2 c_k(t)
\end{align*}
and by integrating we arrive at the solution:
\begin{align*}
    c_k(t) = e^{-D k^2 t} c_{k_-}(0) = e^{-D k^2 t} \int \dd{x_0} e^{-ikx} W(x_0,0)
\end{align*}
and finally:
\begin{align*}
    W(x,t) &= \int \frac{\dd{k}}{2 \pi} e^{ikx - Dk^2 t} \int \dd{x_0} e^{-ik x_0} W(x_0, 0) =\\
    &= \int \dd{x_0} \left[\int_{-\infty}^{\infty}  \frac{\dd{k}}{2 \pi}  e^{-D k^2 t + i(k(x-x_0))}\right] W(x_0, 0)
\end{align*}
This integral can be computed with the Cauchy residual theorem, by shifting the integral path by $ik(x-x_0)$ on the complex plane. We then arrive to:
\begin{align*}
    \frac{1}{\sqrt{4 \pi D t}} \exp\left(-\frac{(x-x_0)^2}{4 D t} \right) 
\end{align*} 
and the general solution is:
\begin{align*}
    W(x,t) = \int \dd{x_0} \frac{\exp \left(-\frac{(x-x_0)^2}{4 D t} \right)}{\sqrt{4 \pi D t}} W(x_0, 0) 
\end{align*}
If we choose an infinite density for the initial condition:
\begin{align*}
    W(x_0, 0) = \delta(x_0)
\end{align*}
then the solution (given that the particle was at $x_0$ at $t=0$) will be:
\begin{align*}
    W(x,t | x_0, 0) = \frac{1}{\sqrt{4 \pi D t}} \exp \left(-\frac{(x-x_0)^2}{4 D t} \right) 
\end{align*}
More generally:
\begin{align*}
    W(x,t | x_0, t_0 ) = \frac{1}{\sqrt{4 \pi D (t-t_0 )}} \exp \left(-\frac{(x-x_0 )^2}{4 D (t-t_0)} \right) \qquad W(x,t|x_0, t_0 ) = \delta(x- x_0)
\end{align*}
We will refer to this as a \textbf{propagator}.\\
Note that we can rewrite it as:
\begin{align*}
    W(x,t|x_0, t_0 ) = \int \dd{x_0 } W(x,t|x_0, t_0) W(x_0, t_0)
\end{align*} 

Let $x_0 = 0 = t_0$, leading to:
\begin{align*}
    W(x,t | 0, 0) = (4 \pi D t )^{-1/2} \exp\left(-\frac{x^2}{4 D t} \right)
\end{align*} 
We will now talk about the concept of \textbf{scale invariance}.\\
We start from:
\begin{align*}
    \partial_t W(x,t) = D \partial_x^2 W(x,t); \qquad x'=\lambda x, t' = \lambda^2 t
\end{align*} 
so that:
\begin{align*}
    \pdv{t'} = \frac{1}{\lambda^2}  \pdv{t}; \qquad \pdv{x'} = \frac{1}{\lambda} \pdv{x} 
\end{align*}
By rearranging, we can write the differential equation as the action of an operator:
\begin{align*}
    (\partial_t - D \partial_x^2) W(x,t) = 0
\end{align*}
which satisfies:
\begin{align*}
    \pdv{t} - D \pdv[2]{x} = \frac{1}{\lambda^2} \left(\pdv{t'} - D \pdv[2]{{x'}} \right)  
\end{align*}
So if $W(x,t)$ is a solution, then also $W(\lambda x, \lambda^2 t)$ is a solution. For a general integral:
\begin{align*}
    W(x,t|0,0) &=  (4 \pi D t)^{-1/2} \exp\left(-\frac{x^2}{4D t} \right) \\
    W(\lambda x, \lambda^2 t | 0, 0) &= (4 \pi D \lambda^2 t)^{-1/2} \exp\left(-\frac{\cancel{\lambda^2} x^2}{4 D t\cancel{ \lambda^2}} \right)\\
     &= \frac{1}{\lambda} W(x,t|0,0)  \\
\end{align*}  
But why are we getting an extra factor of $\lambda$?\\
Recall that $W(x,t)$ is a probability density, so that it is normalized:
\begin{align*}
    1 = \int_{-\infty}^{\infty} \dd{x} W(x,t)
\end{align*}  
So, if we rearrange:
\begin{align*}
    \lambda W(\lambda x, \lambda^2 t | 0, 0) = W(x,t | 0,0)
\end{align*}
and then integrate both sides:
\begin{align*}
    \int \dd{x} \lambda W(\lambda x, \lambda^2 t|0,0) = \int \dd{z} W(z, \lambda^2 t|0,0) = 1
\end{align*}
The two integrals are the same up to a change of variables: $x' = \lambda x$, $\dd{x'} = \lambda \dd{x}$.\\
By choosing $\lambda= 1/ \sqrt{t}$ we get:
\begin{align*}
    W(x,t|0,0) = \frac{1}{\sqrt{t}} W\left(\frac{x}{\sqrt{t}}, 1|0,0 \right) = \frac{1}{\sqrt{t}} f\left(\frac{x}{\sqrt{t}} \right) 
\end{align*}   
Note that this property can be derived even if we do not know the explicit solution:
\begin{align*}
    f(z) = \frac{1}{\sqrt{4 \pi D}} \exp\left(-\frac{z^2}{4D} \right) 
\end{align*}
In fact, by an argument of dimensional analysis, note that $[D] = L^2/t$, and so $[Dt] = [x^2]$. Recall that $[W] = 1/x$, as it is a pdf, and $W\dd{x}$ is a pure number. So:
\begin{align*}
    W(x,t|0,0) = \frac{1}{x} \frac{x}{\sqrt{D t}} ) = \frac{1}{\sqrt{Dt}} \underbrace{\frac{\sqrt{Dt}}{x} F \left(\frac{x}{\sqrt{Dt}} \right)}_{f(x/\sqrt{Dt})}  
\end{align*}  
where $F$ is dimensionless, and $1/x$ restores the correct dimensions.\\
But if we consider also the initial conditions, we have an extra parameter that can be added to the function:
\begin{align*}
    W(x,t|x_0, t_0)
\end{align*}
However, by translational invariance, we can simply translate time and space:
\begin{align*}
    W(x-x_0, t-t_0 | 0,0)
\end{align*}

Now, we start again from:
\begin{align*}
    W(x,t) &= \int \dd{x_0} \frac{1}{\sqrt{4 \pi D (t-t_0 )}} \exp\left(-\frac{(x-x_0 )^2}{4 D (t-t_0)} \right)  W(x_0, t_0) =\\
    &= \int d\dd{x_0} W(x,t|x_0, t_0) W(x_0, t_0)
\end{align*}
and ask what is the probability that a particle will be at position $x_2$ at $t=t_2$, given that the initial condition was $x_1$ at $t_1$.\\
We now that:
\begin{align*}
    \mathbb{P}(x,t|x_0, t_0) \equiv W(x,t|x_0, t_0)
\end{align*}
But can we derive the same result by using the propagator?
\begin{align*}
    W(x_2, t_2) &= \int \dd{x_0} W(x_2, t_2|x_0, t_0) W(x_0, t_0)\\
    W(x_1, t_1) &= \int \dd{x_0} W(x_1, t_1|x_0, t_0) W(x_0, t_0) 
\end{align*}
In principle, we can also write what happens at $t_2$ in terms of what happens at $t_1$:
\begin{align*}
    W(x_2, t_2) = \int \dd{x_1} W(x_2, t_2 | x_1, t_1) W(x_1, t_1)
\end{align*}  
If we substitute $W(x_1, t_1)$ in there:
\begin{align*}
    W(x_2, t_2) = \iint \dd{x_1} \dd{x_0} W(x_2, t_2 | x_1, t_1) W(x_1, t_1 | x_0, t_0) W(x_0, t_0)
\end{align*} 
By comparing with the previous integrals, we find that:
\begin{align*}
    W(x_2, t_2 | x_0, t_0) = \int \dd{x_1} W(x_2, t_2 | x_1, t_1) W(x_1, t_1 | x_0, t_0)
\end{align*}
This is the \textbf{ESCK} property of the propagator.\\
Then, using gaussian integration:
\begin{align*}
    W(x,t|x_0, t_0) = \frac{1}{\sqrt{4 \pi D (t-t_0) }} \exp\left(-\frac{(x-x_0)^2}{4 D (t-t_0)} \right) 
\end{align*} 
(verify it as exercise).\\

Returning to the integral we found:
\begin{align*}
    \mathbb{P}(x_2, t_2; x_1, t_1; x_0, t_0) = W(x_2, t_2|x_1, t_1) W(x_1, t_1 | x_0, t_0) W(x_0, t_0)
\end{align*}
This is the joint probability that the particle arrives at $x_1$ at $t_1$ and then at $x_2$ at $t_2$, given that it started in $x_0$ at $t_0$.\\
We can then compute:
\begin{align*}
    \langle x(t_2) x(t_1) \rangle = \iint \dd{x_1} \dd{x_2} \mathbb{P}(x_2, t_2; x_1, t_1) x_2 x_1
\end{align*}

Let's do an example. Consider:
\begin{align*}
    \mathbb{P}(x_2, t_2; x_1, t_1 | 0, 0) = \mathbb{P}(x_2, t_2; x_1, t_1; 0,0)\frac{1}{W(0,0)} = W(x_2, t_2 | x_1, t_1) W(x_1, t_1 | 0,0)
\end{align*}
we want to compute $\langle x(t_2) x(t_1) \rangle$:
\begin{align*}
    \langle x(t_2) x(t_1) \rangle = \iint \dd{x_1} \dd{x_2} x_1 x_2 \frac{\exp\left(-\frac{(x_2 - x_1)^2}{4D(t_2 -t_1)} \right) }{\sqrt{4 \pi D (t_2 -t_1)}} \frac{\exp\left(-\frac{x_1^2}{4 D t_1} \right)}{\sqrt{4 \pi D t_1}} 
\end{align*}
Changing variables ($x_1 = y_1$, $x_2 - x_1 = y_2$) we get:
\begin{align*}
    = \frac{1}{\sqrt{4 \pi D (t_2-t_1) }}  \frac{1}{\sqrt{4 \pi D t_1}} \iint \dd{y_1} \dd{y_2} y_1 (y_1 + y_2)\exp\left(-\frac{y_2^2}{4D(t_2 -t_1)} - \frac{y_1^2}{4D t_1} \right) 
\end{align*}
Notice that the exponential is an even function, and $y_1 y_2$ is odd, so only the term with $y_1^2$ remains. We arrive at:
\begin{align*}
    \langle x(t_2) x(t_1) \rangle &= \frac{1}{\sqrt{4 \pi D(t_2-t_1)}} \frac{1}{\sqrt{4 \pi D t_1}} \int \dd{y_1} y_1^2 \exp\left(-\frac{y_1^2}{4 D t_1} \right)  \cdot \int \dd{y_2} \exp\left(-\frac{y_2^2}{4 \pi D (t_2-t_1) } \right) =\\
    &= 2D t_1
\end{align*}
Here we supposed $t_1 < t_2$. In the general case, we would have:
\begin{align*}
    \langle x(t) x(t') \rangle = 2D \min(t, t')
\end{align*} 

Generalizing:
\begin{align*}
    \mathbb{P}(x_i, t_i; i=0, \dots, n) &= \mathbb{P}(x_n, t_n; x_{n-1}, t_{n-1}; \dots; x_1, t_1; x_0, t_0) =\\
    &= \prod_{i=1}^n W(x_i, t_i | x_{i-1}, t_{i-1}) W(x_0, t_0)
\end{align*}
This is the joint probability for a \textit{discrete trajectory}, meaning that we care only about what happens at certain discrete times.\\
For the average value of a generic function $f$ of the trajectory points:
\begin{align*}
    \langle f(x(t_n), x(t_{n-1}), \dots, x(t_0)) \rangle
\end{align*}  
we need to use the joint probability:
\begin{align*}
    = \int \prod_{i=0}^n W(x_i, t_i; i=0, \dots, n) \cdot f(x_n, x_{n-1}, \dots, x_0)
\end{align*}
In the next lecture we will try to see how to generalize this kind of calculation to a function that also depends on the \textit{inbetween points}, that is on a \textit{infinite set of values of the trajectory}. For example:
\begin{align*}
    \langle \exp \left(-\int_0^t a(\tau) x(\tau) \dd{\tau}\right) \rangle
\end{align*}  
depends on the \textit{whole} trajectory.
\end{document}
