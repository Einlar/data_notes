%&latex
%
\documentclass[../template.tex]{subfiles}
\begin{document}

\section{RFIM - Part 2}
\lesson{?}{02/12/19}
We were trying to compute the \textit{mean free energy} $\bar{F}$ of a Random Field Ising Model (RFIM). We noted that $\bar{F}$ depends on $\overline{\ln Z}$, which can be computed more easily by first evaluating $\overline{Z^m}$, for which we found the following expression:
\begin{align}
    \overline{Z^n} = \sum_{\{S^a\}} \exp\left[\frac{\beta J}{N} \sum_a \left(\sum_i S_i^a\right)^2 + \frac{\beta^2 \delta^2}{2} \sum_i \left(\sum_a S_i^a\right)^2 \right]
    \label{eqn:Znbar}
\end{align}   
where we are using the \textit{replica trick}, averaging over $n$ replicas of a system with $N$ spins. The notation $\{S^a\}$ denotes a sum over every spin of every replica. 

To remove the squares, we use the Hubbard-Stratonovich transformation. Let $b > 0$: 
\begin{align*}
    \exp\left(\frac{b}{2} z^2 \right) = \frac{1}{\sqrt{2 \pi b}} \int \dd{x} \exp\left(-\frac{x^2}{2b} \pm zx \right) 
\end{align*}
Otherwise, if the exponential argument is negative:
\begin{align*}
    -\exp\left(-\frac{b}{2}z^2 \right) = \frac{1}{\sqrt{2 \pi b}} \int \dd{x} \exp\left(-\frac{x^2}{2b} \pm i z x\right) 
\end{align*} 
These are just kinds of multivariate Gaussian integrals.

\medskip
In our case, we choose:
\begin{align*}
    z_a = \sqrt{2 J \beta} \sum_i S_i^a
\end{align*}
(the $2$ factor is necessary to have a $b/2$ in the exponential) and $b = 1/N$, leading to:
\begin{align*}
    \exp\left(\frac{b}{2} z_a^2 \right) = \frac{1}{\sqrt{2 \pi b}} \int \dd{x_a} \exp\left(-\frac{x_a^2}{2b} + z_a x_a \right)  \qquad \forall a
\end{align*} 
Substituting back in (\ref{eqn:Znbar}) we get:
\begin{align*}
    \overline{Z^n} = \left(\frac{N}{2 \pi} \right)^{n/2} \sum_{\{S^a\}} \int \prod_a \dd{x_a} \exp\left[-\frac{N}{2} \sum_a x^2_a + \sqrt{2J \beta} \sum_i \sum_a S_i^a x_a + \frac{\beta^2 \delta^2}{2} \sum_i \left(\sum_a S_i^a \right)^2 \right]
\end{align*}
Note that, at least in the first term, $S_i^a$ appear \textit{by itself} - there is $S_i^a x_a$, not $S_i^a S_j^a$ (however, this still happens with the last $(\sum_a S_i^a)^2$. We will deal with that later).  

\begin{align*}
    \overline{Z^n} &= \left(\frac{N}{2\pi} \right)^{n/2} \prod_a \dd{x_a} \exp \left[N\left(-\frac{1}{2} \sum_a x_a^2 + \log Z_1 \right)\right]\\
    Z_1(x_a) &= \sum_{\{S^a = \pm 1\}} \exp\left[\sqrt{2 \beta J} \sum_a x_a S^a + \frac{\beta^2 \delta^2}{2} \left(\sum_a S^a\right)^2 \right]
\end{align*}
We now use compute the integrals with the \textit{saddle point approximation} for $N \to \infty$. Also, we assume that $x_a = x \quad \forall a = 1, \dots, n$, which works for this specific system. This means that we can simplify sums:
\begin{align*}
    \sum_a x_a^2 = n x^2
\end{align*} 
So, we proceed to find the exponential maximum by differentiating:
\begin{align*}
    \pdv{x} \left[-\frac{1}{2} n x^2 + \log Z_1 (x) \right] \overset{!}{=}  0 \Rightarrow nx = \pdv{x} \log Z_1(x)
\end{align*}
Denote the solution as $x_m$. Then we have:
\begin{align*}
    n x_m = \frac{\displaystyle \sqrt{2 \beta J}\sum_{S^a = \pm 1} \left(\sum_a S^a\right) e^{A[S, x_m]}}{\displaystyle \sum_{S^a = \pm 1} e^{A[S, x_m]}}
\end{align*} 
where:
\begin{align*}
    A[S,x] = \sqrt{2 \beta J} x \sum_a S^a + \frac{\beta^2 \delta^2}{2} \left(\sum_a S^a\right)^2 
\end{align*}
Note that $n x_m$ looks like the \textit{average} over a certain ensemble:
\begin{align*}
    \langle y \rangle = \frac{\displaystyle \sum_S y e^{- \beta H}}{ \displaystyle \sum_S e^{- \beta H}} 
\end{align*}  
Rearranging:
\begin{align*}
    \frac{x_m}{\sqrt{2 \beta J}} = \frac{\displaystyle \sum_{\{S^a = \pm 1\}} \frac{1}{n} \sum_a S^a e^A }{\displaystyle \sum_{\{S^a = \pm 1\}} e^A} \equiv \langle S \rangle_A = m
\end{align*}
where $m$ is the \textbf{magnetization} of the system.   

Note that:
\begin{align*}
    A = - \beta \tilde{H} = - \beta\left(\frac{1}{\sqrt{\beta}} (\dots) + \beta(\dots) \right)
\end{align*}
so generally only one of the two terms will be dominant at a given temperature.

Recalling that $x_m = \sqrt{2 \beta J}m$ we have, in summary:
\begin{align*}
    \overline{Z^n} &\approx \exp\left(N [-n \beta J m^2 + \log Z_1(m)]\right)\\
    Z(m) &= \sum_{\{S^a = \pm 1\}} e^{A[S,m]}\\
    A[S,m] &= 2 \beta J m \sum_a S^a + \frac{\beta^2 \delta^2}{2} \left(\sum_a S^a\right)^2 \\
    m &= \frac{1}{Z_1(m)} \sum_{S^a = \pm 1} \left(\frac{1}{n} \sum_a S^a \right) e^{A[S,m]}  
\end{align*} 
We now need to get rid of the remaining $(\sum_a S^a)^2$ by applying a second Hubbard-Stratonovich transformation. So we start from:
\begin{align*}
    e^A = \exp\left(2 \beta Jm \sum_a S_a\right) \exp\left(\frac{\beta^2 \delta^2}{2} \left[\sum_a S^a\right]^2 \right)
\end{align*}
Applying H-S with $b=1$, $z = \beta \delta \sum_a S^a$ and $x = \nu$ we get:
\begin{align*}
    \exp\left(\frac{\beta^2 \delta^2}{2} \left[\sum_a S^a\right]^2 \right) = \int \frac{\dd{\nu}}{2 \pi} \exp\left(-\frac{1}{2} \nu^2 + \nu \left[\beta \delta \sum_a S_a\right] \right)
\end{align*}   
And so:
\begin{align*}
    e^{A[S,m]} &= \int \frac{\dd{\nu}}{\sqrt{2 \pi}} \exp\left(-\frac{1}{2} \nu^2 + \underbrace{(2 \beta J m + \beta \delta \nu)}_{\eta} \sum_a S^a  \right)\\
    Z_1(m) &= \sum_{\{S^a = \pm 1\}} e^{A[S,m]} = \int \frac{\dd{\nu}}{\sqrt{2 \pi}} \exp\left(-\frac{1}{2} \nu^2 \right) \prod_a \sum_{S^a = \pm 1} e^{\nu S^a} =\\
    &= \int \frac{\dd{\nu}}{\sqrt{2 \pi}} \exp\left(-\frac{1}{2} \nu^2 \right) [2 \cosh \nu]^n = \\
    &= \int \frac{\dd{\nu}}{\sqrt{2\pi}} \exp\left(-\frac{1}{2} \nu^2 + n \log [2 \cosh \nu] \right) 
\end{align*} 
and so $Z_1 (m)  \xrightarrow[n \to 0]{}  1$. 

\begin{exo}[Magnetization]
    Prove that:
    \begin{align*}
        m = \frac{1}{Z_1(m)} \int \frac{\dd{\nu}}{\sqrt{2\pi}} \exp\left(-\frac{1}{2} \nu^2 + n \log [2 \cosh (2 \beta J m + \beta \delta \nu)] \right)  \tanh [2 \beta J m + \beta \delta \nu]
    \end{align*}
    
\end{exo}

Note that as $n \to 0$ the magnetization becomes:
\begin{align*}
    m = \int \frac{\dd{\nu}}{\sqrt{2 \pi}} \exp\left(-\frac{1}{2}\nu^2 \right) \tanh (2 \beta J m + \beta \delta \nu) 
\end{align*}
We can finally go back, recalling that $h = \delta \nu$ (gaussian noise), and so:
\begin{align*}
    m = \int \underbrace{\frac{\dd{h}}{\sqrt{2\pi \delta^2}} \exp\left(-\frac{h^2}{2 \delta^2} \right)}_{p(h)}   \tanh( \beta (2 J m + h)) = \overline{\tanh(\beta(2 J m + h))}
\end{align*} 
If $\beta \to \infty$ ($T \to 0$), the tangent becomes like a \textit{periodic step function}, which is averaged with gaussian weights. 

To find the \textit{critical line} separating the \textit{paramagnetic} and \textit{ferromagnetic} behaviours, we need to evaluate:
\begin{align}
    \pdv{m} m_{\mathrm{sc} }(m) \Big|_{m=0} = 1
    \label{eqn:cl}
\end{align}   

\begin{exo}[Critical line]
    Prove that the critical line satisfies the condition:
    \begin{align*}
        2 \beta J \int \dd{h} p(h) \frac{1}{[\cosh (\beta h)]^2} = 1  
    \end{align*}
    (just differentiate (\ref{eqn:cl}))
\end{exo}

This can be written in a different manner. First, let:
\begin{align*}
    J' = \frac{J}{\delta} \qquad \beta' = \beta \delta \qquad \tilde{h} = \beta h 
\end{align*}
so that:
\begin{align*}
    2 \beta' J' \int \frac{\dd{\tilde{h}}}{\sqrt{2 \pi}} \exp(-\frac{\tilde{h}^2}{2 {\beta'}^2} ) \frac{1}{(\cosh \tilde{h})^2} = 1 
\end{align*}

\begin{exo}[Critical ratio at $T = 0$]
    Show that the \textit{para-ferro} transition at $T=0$ takes place when:
    \begin{align*}
        \frac{2J}{\delta} = \sqrt{\frac{\pi}{2}}  
    \end{align*}  
\end{exo}

Finally, we can compute the \textbf{free energy} (let $k_B = 1$):
\begin{align*}
    \bar{F} &= -T \overline{\ln Z} = -T \pdv{n} \overline{Z^n} \Big|_{n=0} =\\
    &\underset{\substack{\text{Saddle}\\\text{point}}}{\approx}  -T \pdv{n} \left[\exp\left(N (-n \beta J n^2 + \log Z_1)\right)\right]_{n=0} = \\
    &= -T N \left[- \beta J m^2 + \pdv{n} \ln Z_1\right]_{n=0} =\\ 
     &= N \left[J m^2 - \frac{T}{Z_1} \pdv{n} Z_1 \right] = \\
     &\underset{Z_1 \to 1}{=}  N\left[J m^2 - T\int \frac{\dd{\nu}}{\sqrt{2 \pi}} \exp\left(-\frac{1}{2} \nu^2 \right) \ln(2 \cosh (2 \beta J m + \beta \delta \nu)) \right] = \\
     &= N\left[J m^2 - T \int \frac{\dd{h}}{\sqrt{2 \pi \delta^2 }} \exp\left(-\frac{h^2}{2 \nu^2} \right) \log \left[2 \cosh \left(\beta (2 Jm+ h)\right)\right] \right]
\end{align*} 
which is the \textit{free energy} averaged over the disorder. Note that the magnetization and the energy are interdependent.  




\end{document}
