%&latex
%
\documentclass[../template.tex]{subfiles}
\begin{document}

\lesson{2}{07/10/19}
\section{Brownian motion and Markov Processes}
Let's tackle the diffusion problem with a different approach.\\
Consider a particle moving in $d=1$. We start by discretizing time and space intervals, so that we only focus on positions $x_i$ at instants $t_n$:
\begin{align}
    x_i \equiv i \cdot l\qquad tn \equiv n \cdot \varepsilon
    \label{eqn:discretization}
\end{align} 
The diffusion process is intrinsically stochastic, meaning that the motion of grains is given by collisions at the microscopical level, which are essentially random.\\
So, suppose that the particle lies in a certain known position at $t=0$. After an instant, the particle may have moved to the right (with probability $P_+$) or to the left ($P_-$), or have remained in the same position as before ($P_0$). As these cases cover all the possibilities, it holds:
\begin{align*}
    P_+ + P_- + P_0 = 1
\end{align*}
Denote with $w_i(t_n)$ the probability that the particle lies at position $x_i$ at time $t_n$.\\
The probability for the next timestep is then given by the \textbf{Master Equation} :
\begin{align}
    w_i(t_{n+1}) = P_0 w_i(t_n) + P_+ w_{i-1}(t_n) + P_- w_{i+1}(t_n)
    \label{eqn:master-eq}
\end{align}
In fact, if the particle were at position $i$ at time $t_n$, then it will remain in the same position with probability $P_0$. Otherwise, it could have been one position left and moved to the right ($P_+$), or one position right and moved to the left ($P_-$).\\
Here we supposed that $\varepsilon$ is sufficiently small, so that the particle will only take one step at a time.\\

Stochastic systems for which the state at a certain time depends only on the state one instant before are called \textbf{Markov's Processes} .\\

Note that, as the particle cannot \q{escape from the system}, its probabily to be in \textit{any} position is conserved at any given time: 
\begin{align*}
    \sum_{i=-\infty}^{\infty} w_i(t_{n+1}) = \sum_{i=-\infty}^{\infty} w_i(t_n) = \dots = \sum_{i=-\infty}^{\infty} w_i(0)
\end{align*}

Suppose that the particle \q{always moves}, that is $P_0 = 0$, and also that it \q{is balanced} ($P_+ = P_- = 0.5$). The final position $i$ at time $t_n$ is given by the number of steps to the right $n_+$ minus the number of steps to the left $n_- \in \mathbb{N}$:
\begin{align*}
    \mathbb{Z} \ni i = n_+ - n_-
\end{align*}
This process can be simulated by flipping a coin at each timestep: if it lands on heads the particle will move to the right, otherwise to the left. So, denoting the total number of steps as $n = n_+ + n_-$, then the probability for the particle to be in position $x_i$ is 
given by a binomial distribution:  
\begin{align}
    w_i(t_n) =  {{n}\choose{n_+}} \frac{1}{2^{n_+}} \frac{1}{2^{n_-}} = \frac{1}{2^n} {{n}\choose{n_+}} = {n\choose n_-} \frac{1}{2^n}
    \label{eqn:binomial-markov}
\end{align}
This can be generalized to the case where $P_+ \neq P_-$:
\begin{align}
    w_i(t_n) = {n\choose n_-} P_+^{n_+} P_-^{n_-}
    \label{eqn:witn}
\end{align}

\begin{expl}
    Note that (\ref{eqn:binomial-markov}) satisfies the Master Equation (\ref{eqn:master-eq}), that is:
    \begin{align*}
        w_i(t_{n+1}) = \frac{1}{2}(w_{i+1}(t_n) + w_{i-1}(t_n)) 
    \end{align*}
    We start by noting that if $i = n_+ - n_-$ and $n= n_+ + n_-$, then:
    \begin{align*}
        n_+ = \frac{n+i}{2} \qquad n_- = \frac{n-i}{2}  
    \end{align*}  
    And so:
    \begin{align}
        \frac{1}{2}(w_{i+1}(t_n) + w_{i-1}(t_n)) = \frac{1}{2^{n+1}} \left[{n\choose \frac{n+i+1}{2}} + {n \choose \frac{n+i-1}{2} }\right]
        \label{eqn:binom1} 
    \end{align}
    Recall now the recurrence relation for the binomial coefficient:
    \begin{align*}
        {n\choose k} = {n-1 \choose k} + {n-1 \choose k-1}
    \end{align*}
    which leads to the desired result:
    \begin{align*}
        (\ref{eqn:binom1}) = \frac{1}{2^{n+1}} {n+1 \choose \frac{n+i+1}{2} } = w_{i}(t_{n+1})
    \end{align*}
\end{expl}

\section{Probability Generating Functions}
Let's introduce a useful mathematical tool to deal with the binomial coefficient.\\
Let $X$ be a discrete random variables taking values in the non-negative integers ($\mathbb{N}$). The \textbf{probability generating function} of $X$ is defined as:
\begin{align*}
    G(z) \equiv \mathbb{E}[z^X] = \sum_{x=0}^{+\infty} p(x) z^x 
\end{align*}  
where $p$ is the \textit{probability mass function} of $X$, i.e. $p(x)$ is the probability that $X = x$ ($p(x) = \mathbb{P}(X=x)$).\\
$G(z)$ is useful because we can retrieve $p(k)$ for any $k \in \mathbb{N}$ by simply differentiating $k$ times with respect to $z$ and setting $z = 0$:
\begin{align*}
    G(z) = p(0) + p(1)z + p(2)z^2 + \dots \Rightarrow \dv[k]{z} G(z) \Big|_{z=0} = p(k)
\end{align*}       
Note that $G(1) = 1$ because of the normalization:
\begin{align*}
    G(1) = \sum_{x=0}^{+\infty} p(x) 1^x = \sum_{x=0}^{+\infty} p(x) = 1
\end{align*} 
Then, if we evaluate the first derivative for $z=1$ we get:
\begin{align}
    G'(1) &= p(1) + 2p(2)z + 3p(3)z + \dots \Big|_{z=1} =
    \sum_{x=1}^{+\infty} p(x) x z^{x-1} \Big|_{z=1} =\\
    &= \sum_{x=1}^{+\infty} x p(x) = (0\cdot p(0)) + 1\cdot p(1) + 2\cdot p(2) + \dots = \mathbb{E}[X] 
    \label{eqn:first-moment}
\end{align} 
However, the second derivative of $G$ evaluated at $z=1$ does not give the second moment:
\begin{align*}
    G''(1) &= 2\cdot 1 p(2) + 3\cdot 2 p(3)z + 4\cdot 3 p(4)z^2 +\dots \Big|_{z=1} = \sum_{x=2}^{+\infty} x(x-1)z^{x-2} p(x) \Big|_{z=1} =\\
    &= \mathbb{E}(X(X-1))
\end{align*}  
More generally:
\begin{align*}
    G^{(k)}(1) = \mathbb{E}(X (X-1) \dots (X-k+1)) = \mathbb{E}\left(\frac{X!}{(X-k)!} \right)
\end{align*} 
which is called the $k$-th factorial moment of $X$.\\
But how can we get the \q{usual} moments from $G$? One possibility is to \q{compensate} the difference between a factorial moment and a usual one by adding other terms. For example, note that:
\begin{align*}
    G''(1) = \mathbb{E}(X(X-1)) = \mathbb{E}(X^2) - \mathbb{E}(X)
\end{align*}   
and so:
\begin{align*}
    G''(1) + G'(1) = (\mathbb{E}(X^2) - \mathbb{E}(X)) + \mathbb{E}(X) = \mathbb{E}(X^2)
\end{align*}
A more clever way is to consider the operator $\theta(z)$ defined as:
\begin{align*}
    \theta(z) \equiv z\pdv{z}
\end{align*}
on $G$. In fact:
\begin{align*}
    \theta(z) G(z) = z \pdv{z} \sum_{x=0}^{+\infty} p(x) z^x = z \sum_{x=1}^{+\infty} x p(x) z^{x-1} = \sum_{x=1}^{+\infty} x p(x) z^x
\end{align*} 
And setting $z=1$ leads back to the $\mathbb{E}[X]$. If we apply $\theta(z)$ again, however, something interesting happens:
\begin{align}
    \theta(z)^2 G(z) = \left(z \pdv{z}\right)\left(z \pdv{z}\right) G(z) = z \pdv{z} \sum_{x=1}^{+\infty} xp(x) z^x = \sum_{x=1}^{+\infty} x^2 p(x) z^{x}
    \label{eqn:second-moment}
\end{align}
Now setting $z=1$ leads to $\mathbb{E}[X^2]$. In general:
\begin{align*}
    \theta(z)^k G(z) \Big|_{z=1} = \mathbb{E}[X^k]
\end{align*}  
Note how the exponent of $z$ never changes, as it is lowered by $1$ by the $\partial_z$, and then rised back by the $z$ factor. So, every new application of the $\theta(z)$ operator merely brings down another $x$ factor, rising the $x$ exponent inside the sum - which is exactly what we want to compute moments.      

\section{The Diffused Distribution}
Let's focus on our specific (discrete) case, with the particle moving on a discretized line. At any given time $t_n$ we can compute the mass probability function $W_{t_n}\colon \mathbb{Z} \to \mathbb{R}_+$, with $W_{t_n}(x_i) \equiv w_i(t_n)$. In other words, this is the function that maps every position $x_i$ with the probability of containing a particle at time $t_n$.\\
We are interested in knowing the \textit{shape} of $W_{t_n}$, that is its moments:  
\begin{align*}
    \langle x^q \rangle_{t_n} = \sum_{i=-\infty}^{+\infty} W_{t_n}(x_i) x_i^q \underset{(\ref{eqn:discretization})}{=}  \sum_{i=-\infty}^{+\infty} w_i(t_n)  (l\cdot i)^q \qquad q \in \mathbb{N}
\end{align*}
The first moment ($q=1$) gives the average position:
\begin{align}\nonumber
    \langle x \rangle_{t_n} &= l \cdot \sum_{i=-\infty}^{+\infty} w_i(t_n) i =\\ \nonumber
    &= l \sum_{i=-\infty}^{+\infty} w_i(t_n) (2n_+ - n) =\\
    & \underset{(a)}{=}  l\left(2\langle n_+ \rangle_{t_n} - n\right)
    \label{eqn:x-nplus}
\end{align}
where in (a) we used the normalization condition ($\forall n \in \mathbb{N},\sum_i w_i(t_n) = 1$). Thus, we found that
the average position $\langle x \rangle$ of the particle at time $t_n$  is related to the value of $n_+$.\\

So, let $n_+$ be the random variable of interest. Recall that $n_+$ is sampled from a binomial distribution (\ref{eqn:witn}), and that $n_+ = (n+1)/2$ and so $i = 2n_+ - n$.\\
Then, the \textit{probability generating function} of $n_+$ is given by:   
\begin{align*}
    \widetilde{W} (z,n) &\equiv \sum_{n_+=0}^{n} z^{n_+} w_i(t_n)\big |_{i = 2n_+ -n} \underset{(\ref{eqn:witn})}{=}  \sum_{n_+=0}^{n} z^{n_+} {n\choose n_+} P_+^{n_+} P_-^{n-n_+} =\\
    & \underset{(a)}{=} (P_+ z + P_-)^n
\end{align*}
where in (a) we used the binomial theorem.\\

We can now use the property (\ref{eqn:first-moment}) of the probability generating function to compute $\langle n_+ \rangle$: 
\begin{align}
    \langle n_+ \rangle = \frac{\partial}{\partial z}\widetilde{W}(z,n) \Big|_{z=1} = n(P_+ z + P_-)^{n-1} P_+ \Big|_{z=1} = n\underbrace{(P_+ + P_-)}_{=1} P_+ = nP_+
    \label{eqn:nplus-mean}
\end{align} 

For computing the second moment, we apply the $\theta(z)$ operator, as seen in (\ref{eqn:second-moment}):
\begin{align} \nonumber
    \langle n_+^2 \rangle &= \left(z \pdv{z} \right)^2 \widetilde{W}(z,n) \Big|_{z=1}= z\pdv{z} zn(P_+ z +P_-)^{n-1}P_+ \Big|_{z=1}=\\
    \nonumber
    &= z(n(P_+ z + P_-)^{n-1}P_+ + znP_+^2 (n-1) (P_+ z + P_-)^{n-2}) \Big|_{z=1} =\\
    &= nP_+ (1+(n-1)P_+)
    \label{eqn:nplus-secondmoment}
\end{align}
We can now compute $\operatorname{Var}(n_+)$ recalling that:
\begin{align*}
    \operatorname{Var}[X] = \mathbb{E}[X^2] - (E[X])^2 
\end{align*} 
Thus:
\begin{align}
    \operatorname{Var}[n_+] = \langle n_+^2 \rangle - \langle n_+ \rangle^2 = nP_+ (1+P_+)
    \label{eqn:nplus-variance}
\end{align}
We now go back to $\langle x \rangle_{t_n}$, recalling the relation (\ref{eqn:x-nplus}):
\begin{align*}
    \langle x \rangle_{t_n} = l(2 \langle n_+ \rangle_{t_n} - n) \underset{(\ref{eqn:nplus-mean})}{=} nl(2P_+ - 1) \underset{(a)}{=}  nl(P_+ - P_-)
\end{align*}
where in (a) we used $P_+ + P_- = 1$.\\
For the variance, recall that:
\begin{align*}
    \operatorname{Var}[aX + b] = a^2\operatorname{Var}[X]  
\end{align*} 
and so, starting again from (\ref{eqn:x-nplus}):
\begin{align*}
    \operatorname{Var}[x]_{t_n} = 4l^2 \operatorname{Var}[n_+] \underset{(\ref{eqn:nplus-variance})}{=}  4nl^2 P_+(1-P_+) = 4nl^2 P_+ P_-
\end{align*}
Note that the variance is always proportional to time ($n$), even if $P_+ \neq P_-$. However, if we go back and compute the $\langle x^2 \rangle_{t_n}$, we will note that it is not linear in time:
\begin{align*}
    \langle x^2 \rangle_{t_n} &= \operatorname{Var}[x]_{t_n} + \langle x \rangle_{t_n}^2 = 4nl^2 P_+(1-P_+) + (nl(P_+ - P_-))^2 =\\
    &= nl^2 (4P_+P_- + n(P_+ - P_-)^2)
\end{align*} 

Let's evaluate the previous quantities for the simple symmetrical case, where $P_+ = P_- = 1/2$:
\begin{align*}
    \langle n_+ \rangle = \frac{n}{2}; \qquad \langle n_+^2 \rangle = \frac{n}{4}(n+1); \qquad \operatorname{Var}[n_+] = \frac{3}{4}n\\
    \langle x \rangle_{t_n} = 0;\qquad \langle x^2 \rangle_{t_n} = nl^2 \qquad \operatorname{Var}(x)_{t_n} = nl^2 \span    
\end{align*}
As expected, the average number of steps to the right is half the total steps (as $P_+ = 1/2$), and the average position is $0$. 

\section{Continuum Limit}
Recall that $t_n = n \varepsilon$, and so, for the symmetrical case ($P_+ = P_-$):
\begin{align}
    \langle x \rangle_{t_n} = 0; \qquad \langle x^2 \rangle_{t_n} = l^2 n = \frac{l^2}{\varepsilon} t_n
    \label{eqn:variance-discrete}
\end{align}
The analysis of the diffusion equation in $d=1$ showed that, for a particle starting at $x(t=0) = 0$: 
\begin{align}
    \langle x^2 \rangle_t = 2Dt
    \label{eqn:variance-diffusion}
\end{align}
The correct \textit{continuum limit} should reproduce the result of (\ref{eqn:variance-diffusion}) from (\ref{eqn:variance-discrete}).\\
Notice that if we simply let $\varepsilon \to 0$ and $l \to 0$, $\langle x^2 \rangle$ becomes undefined. Therefore, we need to fix the ratio $l^2 \varepsilon^{-1}$ during the limit. If we define this ratio as:
\begin{align*}
    \frac{l^2}{\varepsilon} \equiv 2D 
\end{align*}    
then the limit of (\ref{eqn:variance-discrete}) leads to (\ref{eqn:variance-diffusion}) as desired:
\begin{align*}
    \langle x^2 \rangle_{t_n} = \frac{l^2}{\varepsilon} t_n  \xrightarrow[l,\varepsilon \to 0]{l^2/\varepsilon = 2D} 2Dt = \langle x^2 \rangle_t  
\end{align*}
Note that $[D]=\si{\m\squared\per\s}$, and so the previous expression is dimensionally correct.\\

\marginpar{Revise wording}
This same result, however, can be also obtained in a more simple way. Suppose to have a random variable $u_n$, that at every time step can take the values: 
\begin{align*}
    u_n = \begin{cases}
        +1 & p = 1/2\\
        -1 & p = 1/2
    \end{cases}
\end{align*} 
Then, define: 
\begin{align*}
    x(t_n) = u_1 + u_2 + \dots + u_n
\end{align*}
If we take the average: 
\begin{align*}
    \langle x(t_n) \rangle = n\langle u \rangle = 0
\end{align*}
And the average of the square: 
\begin{align*}
    \langle x^2 (t_n)\rangle = \langle (u_1 + \dots + u_n)^2 \rangle = n\cdot \underbrace{\langle u^2 \rangle}_{1} + \sum_{i\neq j} \underbrace{\langle u_i \cdot u_j \rangle}_{\langle u_i \rangle \langle u_j\rangle = 0}
\end{align*}
as $u_i$ and $u_j$ are independent. So we showed that $\langle x^2 \rangle_{t_n}$ is linear in the number of time steps ($n$). Here $l^2$ has disappear because the spatial step size is set to $1$.\\

However, the master equation contains much more information. Let's start with: 
\begin{align*}
    w_i(t_n) = {n \choose n_+} \frac{1}{2^n} = \frac{n!}{n_+! n_-!} \frac{1}{2^n}; \qquad i=2n_+ - n; n_+ = \frac{1}{2}(n+i)
\end{align*}
Note that if $n$ is even, only even positions can be visited, and the same if it is odd. So, $i$ and $n$ must have the same parity.\\
That is, $i$ and $n$ have the same parity (from $i = 2n_+ - n$).\\  

\section{Evolution for large times}
We can plot $w_i(t_n)$ for a fixed $t_n$ as a function of $i$:
\begin{align*}
    w_i(t_n) = \frac{n!}{\left(\frac{n+i}{2}\right)! \left(\frac{n-i}{2}\right)!} \frac{1}{2^n}
\end{align*}   
For $n=0$ (starting time), all the particles are at $x_0 $. After one timestep, the probability \q{expands}. %Show some representation

But what happens for $n\to \infty$?\\
One way to see it is to compute the logarithm:
\begin{align*}
    \ln w_i(t_n) = -n \ln 2 + \ln n! - \ln \left(\frac{n+i}{2}\right)! - \ln \left(\frac{n-i}{2}\right)!
\end{align*}
In this way, we can use the Stirling approximation:
\begin{align*}
    \ln k! &= \ln k + \ln(k-1) + \dots + \ln 2 + \ln 1 = \\
    &\approx k\ln k - k  + \frac{1}{2} \ln(2\pi k) 
\end{align*}
We arrive at a complicated expression:
\begin{align*}
    \ln w_i(t_n) &\underset{n \gg 1}{\approx} -n \ln 2 +n\ln n - \cancel{n} + \frac{1}{2}\ln(2\pi n) \\
    &\hphantom{abcd} - \frac{n+i}{2} \ln \left(\frac{n+i}{2} \right) + \cancel{\frac{n+i}{2}} -\frac{1}{2}\ln\left(2 \pi \frac{n+i}{2} \right)  \\
    &\hphantom{abcd} -\frac{n-i}{2} \ln\left(\frac{n-i}{2} \right) +\cancel{\frac{n-i}{2}} - \frac{1}{2}\ln\left(2 \pi \frac{n-i}{2} \right)   
\end{align*}
Let's gradually simplify it. We start by collecting all the $n$:
\begin{align} \nonumber
    &\hphantom{abc}n\left(-\ln 2 + \ln n - \frac{1}{2}\ln\left(\frac{n+i}{2} \right) -\frac{1}{2}\ln\left(\frac{n-i}{2} \right) \right) =  \\ \nonumber
    &= n \ln \left(\frac{n}{2 \sqrt{\frac{n+i}{2} }\sqrt{\frac{n-i}{2} }} \right) = n \ln \left(\frac{n}{\sqrt{n^2 - i^2}} \right) = n\ln \left(\frac{1}{\sqrt{1-i^2/n^{2}}} \right) = \\
    &\underset{(a)}{=} n \ln \left(1 + \frac{1}{2}\frac{i^2}{n^2} + O\left(\frac{i^3}{n^3} \right)  \right) \underset{(b)}{=}  \frac{n}{2} \frac{i^2}{n^2} + O\left(\frac{i^3}{n^3} \right) \approx \frac{i^2}{2n} 
    \label{eqn:piece1}
\end{align} 
where in (a) and in (b) we used respectively the following Taylor expansions (as $n \to \infty$ and so $1/n \to 0$):
\begin{align*}
    (1\pm x)^n = 1 \pm nx + O(x^2) \qquad \ln(1+x) = x + O(x^2)
\end{align*}
Then we collect the $i/2$:
\begin{align} \nonumber
    &-\frac{i}{2} \left[\ln\left(\frac{n+i}{2} \right)-\ln\left(\frac{n-i}{2} \right)\right]  = -\frac{i}{2} \ln\left(\frac{n+i}{n-i}  \right) =\\ \nonumber
    &= -\frac{i}{2}\ln\left(\frac{1 + i/n}{1 - i/n}\right) \underset{(a)}{=}  -\frac{i}{2} \ln\left(\left(1+\frac{i}{n} \right)\left(1+\frac{i}{n} + O\left(\frac{i^2}{n^2} \right)\right)\right) =\\
    &=-\frac{i}{2} \ln\left(1+\frac{2i}{n} + O\left(\frac{i^2}{n^2} \right) \right) \underset{(b)}{=} -\frac{i}{2} \left(\frac{2i}{n} + O\left(\frac{i^2}{n^2} \right) \right) \approx -\frac{i^2}{n} 
    \label{eqn:piece2}
\end{align} 
And finally we consider the remaining terms:
\begin{align}\nonumber
    &\frac{1}{2}\left[\ln(2 \pi n) - \frac{1}{2} \ln\left(2 \pi \frac{n+i}{2} \right) - \ln\left(2 \pi \frac{n-i}{2} \right) \right] =\\ \nonumber
    &\frac{1}{2} \ln \left(\frac{2 \pi n}{ 2 \pi \frac{n+i}{2}  2 \pi \frac{n-i}{2} } \right)   =\frac{1}{2} \ln \left(\frac{2n}{\pi (n^2 - i^2)} \right)  = \frac{1}{2}\ln \left(\frac{2}{\pi n} \frac{1}{1-\frac{i^2}{n^2} }  \right) =\\
    &= \frac{1}{2} \ln \left(\frac{2}{\pi n} + O\left(\frac{i^2}{n^2} \right) \right) \approx \frac{1}{2} \ln\left(\frac{2}{\pi n} \right)
    \label{eqn:piece3} 
\end{align}
Putting it all back together:
\begin{align*}
    \ln w_i(t_n) \underset{n \gg 1}{\approx} (\ref{eqn:piece1}) + (\ref{eqn:piece2}) + (\ref{eqn:piece3}) = \frac{1}{2}\ln\left(\frac{2}{\pi n} \right) -\frac{i^2}{2n}
\end{align*}
And by exponentiating we get:
\begin{align*}
    w_i(t_n) \underset{n \gg 1}{\approx}  \sqrt{\frac{2}{\pi n} } \exp\left(-\frac{i^2}{2n} \right)
\end{align*}

\marginpar{Finish from here}
Then the probability to find the particle in the interval between $x-\Delta x/2$ and $x+ \Delta x/2$ is denoted with $W(x,t) \Delta x$. For $\Delta x \gg l$ we have: 
\begin{align*}
    W(x,t) \Delta x = \sum_{j=i- \Delta x/(2l)}^{i+ \Delta x/(2l)} W_j(t) = \frac{\Delta x}{l} \left(\frac{1}{2} \cdot 0 + \frac{1}{2} W_i(t)\right)
\end{align*}    
as for half the points the function is $0$, and for the other half we can use the central value as approximation.\\
Finally, we arrive at: 
\begin{align*}
    W(x,t) = \frac{1}{2l} \sqrt{\frac{2}{\pi n}} \exp\left(-\frac{i^2}{2n}\right)
\end{align*}
Substituting $x=il$ and $t = n \varepsilon$ we get: 
\begin{align*}
    W(x,t) = \sqrt{\frac{2 \varepsilon}{4l^2 \pi t}} \exp \left(-\frac{x^2}{2 \frac{l^2}{\varepsilon} t}\right)
\end{align*}  
As $l^2/\varepsilon = 2D$: 
\begin{align*}
    W(x,t) = \frac{1}{\sqrt{4 \pi D t}} \exp\left(-\frac{x^2}{4Dt}\right)
\end{align*} 
We can now compute the mean value of $x$:
\begin{align*}
    \langle x \rangle_t &= \int_{\mathbb{R}} W(x,t) x dx = 0\\
    \langle x^2 \rangle_t &= \int_{\mathbb{R}} W(x,t) x^2 dx = 2Dt
\end{align*}
This last integral can be done in many ways. First, recall that: 
\begin{align*}
    I = \sqrt{\frac{\pi}{\mu}} = \int_{-\infty}^{\infty} e^{-\mu y^2}dy
\end{align*}
Differentiating: 
\begin{align*}
    \frac{\partial I}{\partial \mu} = - \int_{\mathbb{R}} e^{-\mu y^2} y^2 dy
\end{align*}
%[?]

%Width * height of gaussian \approx 1. 

\end{document}
