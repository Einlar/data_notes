%&latex
%
\documentclass[../template.tex]{subfiles}
\begin{document}

\section{Variational methods}
\lesson{15}{21/11/19}
%Pag. 86-90 book
\begin{example}[Overdamped harmonic oscillator with variational methods]
    Consider a particle immersed in a harmonic potential $U(x) = m \omega^2 x^2/2$ and subject to thermal noise, moving in a viscous medium. In the \textbf{overdamped limit} $m/\gamma \to 0$ (where $\gamma = 6 \pi \eta a$, with $\eta$ the medium's viscosity and $a$ the particle's radius), the equation of motion becomes:
    \begin{align*}
        \dd{x(t)} = -kx(t) \dd{t} + \sqrt{2D} \dd{B(t)} \qquad k= \frac{m \omega^2}{\gamma} 
    \end{align*}
    A path $\{x(\tau)\}$ solving that equation has a \textit{infinitesimal} probability given by: 
    \begin{align*}
        \dd{P} = \left( \prod_{\tau = 0^+}^t \frac{\dd{x(\tau)}}{\sqrt{4 \pi D \dd{\tau}}} \right) \exp\left(-\frac{1}{4D} \int_0^t (\dot{x} + kx)^2 \dd{\tau} \right)
    \end{align*} 
    as we already derived. We are now interested in computing the transition probabilities:
    \begin{align*}
        W(x,t|x_0,0) = \int_{\mathbb{R}^T} \delta(x(t) - x) \dd{P} 
    \end{align*}
    Following the variational method, we arrive to:
    \begin{align}\label{eqn:harmonic-variational}
        W(x,t|x_0,0) = \Phi(t) \exp\left(-\frac{1}{4D} S[x_c(\tau)] \right)
    \end{align}
    where $S$ is the \textit{action functional} for the harmonic potential:
    \begin{align*}
        S[x(\tau)] = \int_0^t L(\dot{x}, x) \dd{\tau} \qquad L(\dot{x},x) = (\dot{x} + kx)^2
    \end{align*}
    and $x_c(\tau)$ is the path that \textit{stationarizes} $S[x(\tau)]$, meaning that $\delta S[x_c(\tau)] = 0$ and so it satisfies the Euler-Lagrange equation:
    \begin{align*}
        0 \overset{!}{=}  \pdv{L}{x} \Big|_{x_c} - \dv{t} \pdv{L}{\dot{x}} \Big|_{x_c} = 2k(\dot{x}_c + kx_c) - 2(\ddot{x}_c + k\dot{x}_c) = 2(k^2 x_c - \ddot{x}_c)
    \end{align*}   
    as:
    \begin{align*}
        \pdv{L}{x} &= 2k(\dot{x} + kx)\\
        \pdv{L}{\dot{x}} &= 2(\dot{x} + kx)
    \end{align*}
    So, to find $x_c(\tau)$ we need to solve:
    \begin{align*}
        \begin{cases}
            \ddot{x}_c = k^2 x_c\\
            x_c(0) = x_0\\
            x_c(t) = x
        \end{cases}
    \end{align*}
    This is the second order ordinary differential equation for an \textit{harmonic repulsor}, which has the following general integral:
    \begin{align*}
        x_c(\tau) = A e^{k \tau} + B e^{-k \tau}
    \end{align*}
    Imposing the boundary conditions leads to:
    \begin{align*}\span
        \begin{dcases}
            x_0 \overset{!}{=} A + B\\
            x \overset{!}{=} A e^{k t} + B e^{-kt}
        \end{dcases} \Rightarrow \begin{dcases}
            B = x_0 - A\\
            x e^{kt} = A e^{2kt} + B
        \end{dcases} \Rightarrow xe^{kt} - x_0 = A[e^{2kt} -1]\\
        \Rightarrow A &= \frac{x e^{kt} - x_0}{e^{2kt} - 1} \textcolor{Red}{\frac{e^{-kt}}{e^{-kt}}} = \frac{(x e^{kt} - x_0) e^{-kt}}{\frac{e^{kt} - e^{-kt}}{\textcolor{Red}{2} } \textcolor{Red}{2}} = \frac{x -x_0 e^{-kt}}{2 \sinh (kt)}\\
         B &=  x_0 - A = -\frac{x - x_0 e^{kt}}{2 \sinh(kt)} = 2
    \end{align*}
    Then we evaluate the action at the stationary path $x_c(\tau)$:
    \begin{align*}
        S[x_c(\tau)] &= \int_0^t (\dot{x} + kx)^2 \dd{\tau} = \int_0^t [2k A e^{k \tau}]^2 \dd{\tau} = 4k^2 A^2 \frac{1}{2k} e^{2k \tau} \Big|_0^t =\\
        &=4kA^2\hlc{Yellow}{\frac{e^{2kt} - 1}{2}} \textcolor{Red}{\frac{\hlc{Yellow}{e^{-kt}}}{e^{-kt}} } = 4kA^2 \hlc{Yellow}{\sinh(kt)} e^{kt} =\\
        &=\cancel{4}k \frac{(x-x_0 e^{-kt})^2}{\cancel{4} \sinh(kt)} e^{kt} = \frac{k(x - x_0 e^{-kt})^2}{e^{kt} - e^{-kt}} \frac{2}{e^{-kt}} = \frac{2k (x-x_0 e^{-kt})^2}{1-e^{-2kt}}   
    \end{align*}
    Substituting back in (\ref{eqn:harmonic-variational}):
    \begin{align*}
        W(x,t|x_0,0) = \Phi(t) \exp\left(-\frac{k}{2D (1-e^{-2kt})} [x-x_0 e^{-kt}]^2 \right)
    \end{align*}
    All that's left to find $\Phi(t)$ is to use the normalization condition:
    \begin{align*}
        1 &\overset{!}{=} \int_{\mathbb{R}} \dd{x} W(x,t|x_0,0) = \Phi(t) \int_{\mathbb{R}} \dd{x} \exp\Bigg(-\overbrace{\frac{k}{2D (1-e^{-2kt})}}^{\alpha} [x-x_0 e^{-kt}]^2 \Bigg) =\\
        &= \Phi(t) \sqrt{\frac{\pi}{\alpha} } = \Phi(t) \sqrt{\frac{2 \pi D (1-e^{-2kt}) }{k} } \Rightarrow \Phi(t) = \sqrt{\frac{k}{2 \pi D} } \frac{1}{\sqrt{1 - e^{-2kt}}} 
    \end{align*}
    And so the full solution is:
    \begin{align*}
        W(x,t|x_0,0) &= \sqrt{\frac{k}{2 \pi D}} \frac{1}{\sqrt{1 - e^{-2kt}}}  \exp\left(-\frac{k}{2D} \frac{(x- x_0 e^{-kt})^2}{(1-e^{-2kt})}  \right)\\  &\xrightarrow[t \to \infty]{}   \sqrt{\frac{k}{2 \pi D} } \exp\left(-\frac{k}{2D x^2} \right)
    \end{align*}
    As before, we can compute the $t \to \infty$ with a Maxwell-Boltzmann distribution $e^{- \beta U(x)}$, obtaining:
    \begin{align*}
        \frac{1}{2} \beta m \omega^2 x^2 = \frac{k}{2D} x^2 \Rightarrow D = \frac{k}{\beta m \omega^2} = \frac{1}{\beta \gamma} = \frac{k_B T}{\gamma} \Rightarrow D \gamma = k_B T     
    \end{align*}
    as we previously derived.
\end{example}

If we do not consider the overdamped limit, however, the equation of motion is given by:
\begin{align*}
    m\ddot{x} = - \gamma \dot{x} - m \omega^2 x + \sqrt{2D} \gamma \xi
\end{align*}
This can be rewritten as a system of two first order (stochastic) differential equations:
\begin{align*}
    \begin{dcases}
        \dd{x}(\tau) = v(\tau) \dd{\tau} \\
        \dd{v}(\tau) = -\frac{\gamma}{m} v(\tau) \dd{\tau} + \frac{\gamma \sqrt{2D}}{m} \dd{B}    
    \end{dcases}
\end{align*}
It is convenient to \q{symmetrize} the system, by adding a stochastic term also in the first equation:
\begin{align*}
    \begin{dcases}
        \dd{x}(\tau) = v(\tau) \dd{\tau} + \textcolor{Red}{2 \hat{D}\sqrt{\dd{\hat{B}}}}\\
        \dd{v}(\tau) = -\frac{\gamma}{m} v(\tau) \dd{\tau} + \frac{\gamma \sqrt{2D}}{m} \dd{B}  
    \end{dcases}
\end{align*}
and then we'll consider the limit $\hat{D} \to 0$.

First, as usual, we discretize, with $\{t_i\}_{i=0,\dots,n}$ and $t_0 \equiv 0$, $t_n \equiv t$, arriving to:
\begin{align*}
    \begin{dcases}
        \Delta x_i = v_{i-1} \Delta t_i + \sqrt{2 \hat{D}} \Delta \hat{B}_i\\
        \Delta v_i = - \frac{\gamma}{m} v_{i-1} \Delta t_i + \frac{\gamma}{m} \sqrt{2 D} \Delta B_i  
    \end{dcases}
\end{align*}
Where the velocity is evaluated at $t_{i-1}$ as per Ito's prescription. As $\Delta B_i$ and $\Delta \hat{B}_i$ are \textbf{independent} gaussian increments, their joint distribution is just a product:
\begin{align*}
    \dd{P} (\Delta B_1, \Delta \hat{B}_1, \dots, \Delta B_n, \Delta \hat{B}_n) = \left( \prod_{i=1}^n \frac{\dd{\Delta B}_i}{\sqrt{2 \pi  \Delta t_i}} \frac{\dd{\Delta \hat{B}_i}}{\sqrt{2 \pi \Delta t_i}} \right) \exp\left(-\frac{1}{2} \sum_{i=1}^n \frac{\Delta B_i^2}{\Delta t_i}  - \frac{1}{2} \sum_{i=1}^n \frac{\Delta \hat{B}_i^2}{\Delta t_i}    \right)  
\end{align*}
As done previously (see 14/11 notes), to get the distribution for $\Delta x_i$ and $\Delta v_i$ we make a change of random variables:
\begin{align*}
    \Delta \hat{B}_i &= \frac{\Delta x_i - v_{i-1} \Delta t_i}{\sqrt{2 \hat{D}}} \\
    \Delta B_i &= \left(\Delta v_i + \frac{\gamma}{m} v_{i-1} \Delta t_i \right) \frac{m}{\gamma \sqrt{2D}} 
\end{align*}
with jacobian:
\begin{align*}
    \operatorname{det}\left|\pdv{\{\Delta \hat{B}_i\}}{\{\Delta x_i\}}\right| &=(2 \hat{D})^{-n/2}\\ \operatorname{det}\left|\pdv{\{\Delta B_i\}}{\{\Delta x_i\}}\right| &= \operatorname{det}\left|\pdv{\{\Delta x_i\}}{\{\Delta B_i\}}\right|^{-1} = \left(\frac{\gamma}{m} \sqrt{2D} \right)^{-n} = \left(\frac{\gamma^2}{m^2} 2 D \right)^{-n/2}
\end{align*}
leading to:
\begin{align}\nonumber
    \dd{P}(\{\Delta x_i\}, \{\Delta v_i\}) &= \left(\prod_{i=1}^n \frac{\dd{\Delta x_i}}{\sqrt{4 \pi D \Delta t_i}} \frac{\dd{\Delta v_i}}{\sqrt{4 \pi D \Delta t_i \gamma^2 / m^2 }}  \right) \cdot\\ \nonumber
    &\quad \> \cdot \exp\left(-\frac{1}{2} \sum_{i=1}^n \frac{m^2}{2 \gamma^2 D} \left[\left(\frac{\Delta v_i + \gamma/m v_{i-1} \Delta t_i}{\Delta t_i} \right)^2 \Delta t_i \right]  \right) \cdot \\\nonumber
    &\quad \> \cdot \exp\left(-\frac{1}{2} \sum_{i=1}^n \frac{1}{2 \hat{D}} \left[\left(\frac{\Delta x_i - v_{i-1} \Delta t_i}{\Delta t_i} \right)^2 \Delta t_i\right]  \right) =\\\nonumber
    &= \left(\prod_{i=1}^n \frac{\dd{\Delta x_i}}{\sqrt{4 \pi D \Delta t_i}} \frac{\dd{\Delta v_i}}{\sqrt{4 \pi D \Delta t_i \gamma^2 / m^2 }}  \right) \cdot\\\nonumber
    &\quad \> \cdot \exp\left(-\frac{m^2}{4 D \gamma^2} \sum_{i=1}^n  \left[\left(\frac{\Delta v_i}{\Delta t_i} + \frac{\gamma}{m} v_{i-1}   \right)^2 \Delta t_i \right]  \right) \cdot \\ \label{eqn:discretizedP}
    &\quad \> \cdot \exp\left(-\frac{1}{4 \hat{D}} \sum_{i=1}^n  \left[\left(\frac{\Delta x_i}{\Delta t_i} - v_{i-1} \right)^2 \Delta t_i\right]  \right)
\end{align}
Taking the continuum limit $n \to \infty$ leads to:
\begin{align*}
    \dd{P}(\{x(\tau), v(\tau)\}) &= \left(\prod_{\tau = 0^+}^t \frac{\dd{x(\tau)}}{\sqrt{4 \pi \hat{D} \dd{\tau}}} \frac{\dd{v(\tau)}}{\sqrt{4 \pi D \dd{\tau} \gamma^2/m^2}}  \right)\cdot\\
    &\quad \>\cdot  \exp\left(-\frac{m^2}{4 D \gamma^2} \int_0^t \dd{\tau} \left[\dot{v}(\tau) + \frac{\gamma}{m} v(\tau) \right]^2 - \frac{1}{4 \hat{D}} \int_0^t \dd{\tau} [\dot{x}(\tau)- v(\tau)]^2 \right)
\end{align*}
In the limit $\hat{D} \to 0^+$, $1/(4 \hat{D}) \to +\infty$, and so the gaussian pdf for the $\Delta \hat{B}_i$ becomes \textit{infinitely thin}, and the only path with a non-vanishing probability will be the one where:
\begin{align*}
    \int_0^t \dd{\tau} [\dot{x} - v(\tau)]^2 = 0
\end{align*} 
As any $> 0$ value will lead to $\exp(-\infty) = 0$. In particular, the $i$-th factor of the discretization becomes:
\begin{align*}
    &\frac{1}{\sqrt{4 \pi \hat{D} \Delta t_i}} \exp\left[-\frac{1}{4 \hat{D}} \left(\frac{\Delta x_i}{\Delta t_i} - v_0^2 \right) \Delta t_i \right] =\\
    &= \frac{1}{\sqrt{4 \pi \hat{D} \Delta t_i}}  \exp\left(-\frac{1}{4 \hat{D} \Delta t_i} (\Delta x_i - v_{i-1} \Delta t_i)^2 \right)  \xrightarrow[\hat{D} \to 0]{}  \delta(\Delta x_i - v_{i-1} \Delta t_i)
\end{align*}
where we used a limit definition for the $\delta$:
\begin{align*}
    \lim_{\epsilon \to 0} \frac{1}{\sqrt{4 \pi \epsilon}} \exp\left(-\frac{x^2}{4 \epsilon} \right) = \delta(x)
\end{align*}
with $\epsilon = \hat{D} \Delta t_i$ and $x = \Delta x_i - v_{i-1} \Delta t_i$.

\begin{comment}
In the limit $\hat{D} \to 0$:
\begin{align*}
    \prod_i \dd{\Delta x_i} \delta\left(\frac{\Delta x_i}{\Delta t_i} - v_i \right) \to \prod_\tau \dd{x}(\tau) \delta(\dot{x}(\tau) - v(\tau))
 \end{align*} 
 leading to:
 \begin{align*}
     \prod_i \frac{\dd{\Delta x_i}}{\sqrt{4 \pi \hat{D} \Delta t_i}} \exp\left(-\frac{1}{4 \hat{D}}\left[\frac{\Delta x_i}{\Delta t_i} - v_i\right] \right)^2
 \end{align*}
 and
 \begin{align*}
     \dd{x} = v\dd{\tau} + \sqrt{2 \hat{D}} \dd{\hat{B}} \Rightarrow \frac{(\dd{\hat{B}})^2}{\dd{\tau}} = \frac{(\dd{x} - v \dd{\tau})^2}{\sqrt{2 \hat{D}} \dd{\tau}}  
 \end{align*}
\end{comment}

Substituting back in (\ref{eqn:discretizedP}):
\begin{align*}
    \dd{P}(\{\Delta x_i\}, \{\Delta v_i\}) &=  \left(\prod_{i=1}^n \dd{\Delta x_i} \delta(\Delta x_i - v_{i-1} \Delta t_i) \frac{\dd{\Delta v_i}}{\sqrt{4 \pi D \Delta t_i \gamma^2 / m^2 }}  \right) \cdot\\\nonumber
    &\quad \> \cdot \exp\left(-\frac{m^2}{4 D \gamma^2} \sum_{i=1}^n  \left[\left(\frac{\Delta v_i}{\Delta t_i} + \frac{\gamma}{m} v_{i-1}   \right)^2 \Delta t_i \right]  \right)
\end{align*}
Now consider the discretized transition probability:
\begin{align} \label{eqn:disW}
    W(x_n, v_n, t_n|x_0,v_0,0) &= \int_{\mathbb{R}^n \times \mathbb{R}^n} \dd{P}(\{x_i,v_i\}) \delta(x_n - x) \delta(v_n - x) =\\ \nonumber
    &= \int_{\mathbb{R}^n \times \mathbb{R}^n}   \left(\prod_{i=1}^n \dd{\Delta x_i} \delta(\Delta x_i - v_{i-1} \Delta t_i) \frac{\dd{\Delta v_i}}{\sqrt{4 \pi D \Delta t_i \gamma^2 / m^2 }}  \right) \cdot\\\nonumber
    &\quad \> \cdot \exp\left(-\frac{m^2}{4 D \gamma^2} \sum_{i=1}^n  \left[\left(\frac{\Delta v_i}{\Delta t_i} + \frac{\gamma}{m} v_{i-1}   \right)^2 \Delta t_i \right]  \right) \delta(v_n - v) \delta(x_n - x)
\end{align} 
Let's focus on the integrations over $x_i$:
\begin{align*}
    \int_{\mathbb{R}^n} \left(\prod_{i=1}^n \dd{\Delta x_i} \delta(\Delta x_i - v_{i-1} \Delta t_i)\right) \delta(x_n - x) =\\
    \span= \int_{\mathbb{R}^n} \dd{\Delta x_1}\dots \dd{\Delta x_n} \delta(\Delta x_1 - v_0 \Delta t_1)\dots \delta(\Delta x_n - v_{n-1} \Delta t_n ) \delta(x_n - x)
\end{align*}
We then perform the change of variables $\Delta x_1 = x_1 - x_0$, with $x_0$ constant, so that $\dd{\Delta x_1} = \dd{x_1}$. Then we integrate over $\dd{x_1}$, eliminating the first $\delta$ and setting $x_1 = x_0 - v_0 \Delta t_1$:
\begin{align*}
    \int_{\mathbb{R}^n} \dd{x_1} \dd{\Delta x_2} \dots \dd{\Delta x_n} \delta(x_1 - x_0 - v_0 \Delta t_1) \delta(\Delta x_2 - v_1 \Delta t_2) \dots \delta(\Delta x_n - v_{n-1} \Delta t_n) \delta(x_n - x) =\\
    \span \int_{\mathbb{R}^{n-1}} \dd{\Delta x_2} \dots \dd{\Delta x_n} \delta(x_2 - x_0 - v_0 \Delta t_1 - v_1 \Delta t_2) \dots \delta(\Delta x_n - v_{n-1} \Delta t_n) \delta (x_n - x)
\end{align*}
Repeating these steps for all the other variables except the last one, we arrive to:
\begin{align*}
    = \int_{\mathbb{R}} \dd{x_n} \delta\left(x_n - x_0 - \sum_{i=1}^n v_{i-1} \Delta t_i\right) \delta(x_n - x) = \delta\left(x- x_0 - \sum_{i=1}^n v_{i-1} \Delta t_i\right)
\end{align*}
In the continuum limit, this becomes:
\begin{align*}
    \delta\left(x-x_0 - \int_0^t v(\tau) \dd{\tau}\right)
\end{align*}
Substituting back in (\ref{eqn:disW}) and finally taking the limit $n \to \infty$:
\begin{align*}
    W(x,v,t|x_0,v_0,0) &= \int_{\mathbb{R}^T} \left(\prod_{\tau=0^+}^t \frac{\dd{v(\tau)}}{\sqrt{4 \pi D \dd{\tau} \gamma /m^2}}\right) \exp\left(-\frac{m^2}{4 D \gamma} \int_0^t \left(\dot{v}(\tau) + \frac{\gamma}{m} v(\tau) \right)^2\dd{\tau} \right)\\
    &\quad \>\cdot  \delta(v(t) -v ) \delta\left(x- x_0 - \int_0^t v(\tau) \dd{\tau}\right)
\end{align*}

We can now use the variational method to compute that integral. So, let $v_c(\tau)$ be the path, starting at $v(0) = v_0$ that \textit{stationarizes} the action functional:
\begin{align*}
    S[v(\tau)] = \int_0^t \left(\dot{v}(\tau) +  \frac{\gamma}{m} v(\tau) \right)^2 \dd{\tau}
\end{align*}
so that $\delta S[v_c(\tau)] = 0$, and also satisfies the constraints imposed by the $\delta$:
\begin{align*}
    v(t) \overset{!}{=}  v \qquad x-x_0 \overset{!}{=}  \int_0^t v(\tau) \dd{\tau}
\end{align*}
Then, the path integral is given by:
\begin{align}
    W(x,v,t|x_0,v_0,0) = \Phi(t) \exp\left(-\frac{m^2}{4 D \gamma} \int_0^t \left(\dot{v}_c(\tau) + \frac{\gamma}{m}v_c(\tau) \right)^2 \dd{\tau} \right)
    \label{eqn:underdamped-variational}
\end{align}
All that's left is to compute $v_c(\tau)$ and evaluate the integral. This is a problem of \textit{constrained optimization}, for which we use the method of Lagrange multipliers. 

\begin{expl}\textbf{Brief refresher of Lagrange multipliers}. Suppose we have two functions $F, g\colon \mathbb{R}^2 \to \mathbb{R}$, with $F(x,y)$ being the function to maximize, and $g(x,y) = c \in \mathbb{R}$ a constraint. A stationary point $(x_0,y_0)$ of $F$ subject to the constraint $g(x,y) = c$ is such that if we move slightly from $(x_0,y_0)$ along the contour $g(x,y) = c$, the value of $F(x,y)$ does not change (to first order). This happens if the contour of $F$ passing through the stationary point $F(x,y) = F(x_0, y_0)$ is parallel at $(x_0,y_0)$ to that of $g(x,y) = c$, meaning that at $(x_0, y_0)$ the gradients of $F$ and $g$ are parallel:
\begin{align*}
    \nabla_{x,y} F = \lambda \nabla_{{x,y}} g \qquad \lambda \in \mathbb{R}
\end{align*}
(Here we assume that $\nabla_{x,y} g (x_0,y_0) \neq \bm{0}$). Rearranging:
\begin{align*}
    \nabla_{x,y} (F(x,y) - \lambda g(x,y)) = \bm{0}
\end{align*}
Together with the constraint equation $g(x,y) = c$, we have now $3$ equations in $3$ unknowns ($x,y,\lambda$) that can be solve to yield the desired stationary point $(x_0, y_0)$.
\end{expl}

In this case, we have \textit{functionals} instead of functions, and \textit{functionals derivatives} (i.e. variations) instead of derivatives. So, to find the stationary points of:
\begin{align*}
    \int_0^t \left(\dot{v}(\tau) + \frac{\gamma}{m} v(\tau) \right)^2 \dd{\tau} \qquad (a)
\end{align*}
subject to the constraint:
\begin{align*}
    \int_0^t v(\tau) \dd{\tau} = x - x_0\qquad (b)
\end{align*}
we need to solve:
\begin{align*}
    \delta \int_0^t \underbrace{\left[ \left(\dot{v}(\tau) + \frac{\gamma}{m} v(\tau) \right)^2 - \lambda v(\tau)\right]}_{L(v,\dot{v})}  \dd{\tau} = 0
\end{align*}
And applying the definition of first variation (the $\delta$ above) leads to solving the Euler-Lagrange equations:
\begin{align*}
    \pdv{L}{v} - \dv{\tau} \pdv{L}{\dot{v}} \Big|_{v = v_c} \overset{!}{=}  0
\end{align*} 
Expanding the computations:
\begin{align*}
    2\left(\dot{v}_c + \frac{\gamma}{m} v_c \right) \frac{\gamma}{m} - \lambda - \dv{\tau} \left[2 \left(\dot{v}_c + \frac{\gamma}{m} v_c \right)\right] = 0 \Rightarrow \ddot{v}_c(\tau) = v_c(\tau) \left(\frac{\gamma}{m} \right)^2 - \frac{\lambda}{2} 
\end{align*}
The homogeneous solution is again a combination of exponentials:
\begin{align*}
    v_c(\tau) = A \exp\left(-\frac{\gamma}{m}\tau \right) + B \exp\left(\frac{\gamma}{m} \tau \right)
\end{align*}
And for the inhomogeneous general integral we just need to add a particular solution, for example the one with constant velocity $\dot{v}(\tau) = \text{const} \Rightarrow \ddot{v}_c(\tau) = 0$, given by:
\begin{align*}
    v_c(\tau) = \frac{\lambda}{2} \left(\frac{m}{\gamma} \right)^2
\end{align*}
Then, we need to impose the boundary conditions:
\begin{align*}
    v_c(0) = v_0 \qquad v_c(t) = v\qquad \int_0^t v(\tau)\dd{\tau} = (x-x_0)
\end{align*}  
So we have $3$ parameters (the two constants of integration $A, B$ and $\lambda$) and $3$ equations. After finding all of them, we just need to evaluate the integral (\ref{eqn:underdamped-variational}) (computations omitted).

\end{document}
