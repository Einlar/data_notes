%&latex
%
\documentclass[../template.tex]{subfiles}
\begin{document}

\section{Review}
\lesson{3}{10/10/19}
In the last lecture we introduced the concept of a tensor. We started by defining contra-variant and co-variant vectors as objects that transform in a certain way:
\begin{align*}
    V^\mu \to V'^\mu = \Lambda^\mu_{\diamond \alpha}V^\alpha; \quad V_\mu \to V'_\mu = \Lambda_\mu^{\diamond \alpha}V_\alpha
\end{align*}
with:
\begin{align*}
    \Lambda_\mu^{\diamond \alpha} \equiv \eta_{\mu \nu}\eta^{\alpha \beta} \Lambda^\nu_{\diamond \beta}
\end{align*}
Contra-variant vectors are also called rank $(1,0)$ tensors, while co-variant vectors are rank $(0,1)$ tensors.\\
A tensor with more than one index transforms by combining the specific transformation for each index, such in:
\begin{align*}
    A_{\mu \nu} \to \Lambda_\mu^{\diamond \alpha} \Lambda_\nu^{\diamond \beta} A_{\alpha \beta}
\end{align*} 

\section{Maxwell's equations in manifestly covariant form}
All this notation is really useful to write \q{manifestly covariant} expressions - that is relations that are immediately recognizable as being valid even after Lorentz Boosts/rotations.\\

In particular, we can write Maxwell's equations using the $4$-vector notation. We start by defining a tensor:
\begin{align*}
    F_{\mu \nu} = \left(\begin{array}{cccc}
    0 & E_x/c & E_y/c & E_z/c \\ 
    -E_x/c & 0 & -B_z & B_y \\ 
    -E_y/c & B_z & 0 & -B_x \\ 
    -E_z/c & -B_y & B_x & 0
    \end{array}\right)
\end{align*} 
and a $4$-vector:
\begin{align*}
    J^\mu = (c \rho, \vec{j})
\end{align*} 
where $\rho$ is the charge density, and $\vec{j}$ is the current density.\\
Then, the Maxwell's equations can be written as:
\begin{align*}
    \partial_\mu F^{\mu \nu} = \mu_0 J^\nu\\
    \partial_\mu \epsilon^{\mu \nu \alpha \beta}F_{\alpha \beta} = 0
\end{align*}
where $\epsilon^{\mu \nu \alpha \beta}$ is a totally anti-symmetric tensor (so that it changes sign when we exchange any two indices), with $\epsilon^{0123} \equiv +1$. \\
These equations are \textbf{manifestly covariant}, that is the left and right side vary in the same way under a boost: we then know that the laws of electromagnetism are the same in all inertial frames of reference (as we hoped). 

\section{The 4-velocity}
Consider a massive particle/object that is moving wrt some reference frame $O$. In relativity we need to specify, in addition to the three spatial coordinates, also the time. So we use a $4$-vector $x^\mu$.\\
The set of points in spacetime reached by the particle during its motion form a curve called \textbf{worldline}. We can parametrize this curve with the proper time $\tau$, that is the time as measured from a reference frame where the particle is at rest. So, we have $x^\mu(\tau)$.\\

We now define the $4$-velocity as the derivative of $x^\mu$ wrt proper time:
\begin{align*}
    u^\mu = \frac{dx^\mu}{d \tau} 
\end{align*}
Note that:
\begin{align*}
    ds^2 &=  -c^2 dt^2 + dx^2 + dy^2 +dz^2 \\
    &=  -c^2 d \tau^2
\end{align*}
and so $d \tau = \sqrt{-ds^2}/c$, leading to:
\begin{align*}
    u^\mu = \frac{c dx^\mu}{\sqrt{-ds^2}} 
\end{align*}
As $dx^\mu$ is a $4$-vector (in fact, we defined $4$-vectors to represent events in spacetime), and $ds^2$ is a scalar, we immediately know that $u^\mu$ is a 4-vector.\\

Recall now, from the effect of time dilation, that:
\begin{align*}
    d \tau = \sqrt{1- v^2/c^2} dt
\end{align*}
and so:
\begin{align*}
    u^\mu = \frac{dx^\mu}{\sqrt{1-v^2/c^2}dt} = \left(\frac{c \cancel{dt}}{\sqrt{1-v^2/c^2}\cancel{dt}}, \frac{d \vec{x}}{\sqrt{1-v^2/c^2}dt}  \right)  = (\gamma c, \gamma \vec{v})
\end{align*}
Note that at $v \ll c$, $\gamma = 1$, and so the spatial components of $u^i = \vec{v}^{\,i}$ are the components of the usual velocity.\\
In the rest frame of the particle, $v=0$ and so $u^\mu = (c,0)$.\\
Then, note that $u^\mu u_\mu$ is a scalar (it is a contraction of a $4$-vector with himself), and so it is the same wrt all frame of reference. Thus, we can compute it in the rest frame, where the calculations are simpler:
\begin{align*}
    u^\mu u_\mu = -c^2
\end{align*} 
Of course, even in different frames we get the same result:
\begin{align*}
    u^\mu u_\mu &= \eta_{\mu \nu} u^\mu u^\nu = \eta_{\mu \nu} \frac{dx^\mu}{d \tau} \frac{dx^\nu}{d \tau} = \\
    &=  \hlc{Yellow}{\eta_{\mu \nu}}  \frac{\hlc{Yellow}{dx^\mu}}{\sqrt{-ds^2}/c} \frac{\hlc{Yellow}{dx^\nu}}{\sqrt{-ds^2}/c} = \\
    &=  - \frac{\hlc{Yellow}{ds^2}}{ds^2/c^2} = -c^2 
\end{align*}

\section{Energy-Momentum 4-vector}
We define:
\begin{align*}
    p^\mu = m u^\mu
\end{align*}
As $m$ (the invariant mass) is a scalar, this is indeed a $4$-vector.\\
Note that:
\begin{align*}
    u^\mu = (\gamma c , \gamma \vec{v}) \Rightarrow p^\mu= (\gamma m c, \gamma m \vec{v}) 
\end{align*}
In the non-relativistic regime we have $\gamma \approx 1$, so that $p^i = mv^i$, which is the non-relativistic momentum.\\
The time component is:
\begin{align*}
    p^0 = \frac{mc}{\sqrt{1-\frac{v^2}{c^2} }} 
\end{align*}  
and we want to understand what it means in the non-relativistic limit. If we set $v=0$ we get $mc$, which is difficult to interpret. So, we need an expansion around $v=0$. Recall the common Taylor expansion:
\begin{align*}
    (1+\epsilon)^n \underset{|\epsilon| \ll 1}{\approx}  1+ n \epsilon
\end{align*} 
So by setting:
\begin{align*}
    \epsilon = -\frac{v^2}{c^2}; \qquad n=-\frac{1}{2}  
\end{align*}
we get:
\begin{align*}
    p^0 \approx mc \left[1+\frac{v^2}{2c^2} \right] = mc + \frac{1}{c} \frac{mv^2}{2}  
\end{align*}
Note that $mv^2/2$ is the kinetic energy of the particle. We can rearrange:
\begin{align*}
    cp^0 = mc^2 + \frac{1}{2} mv^2 
\end{align*} 
We then conclude that also the $mc^2$ term must be some kind of energy - and so we call it the \textbf{rest energy}, as it is the only energy that an object possesses when it is not moving. So:
\begin{align*}
    p^0 = \frac{\text{Energy}}{c} 
\end{align*}  
This justifies the name of \q{energy-momentum 4-vector} for $p^\mu$.\\

Note that we immediately know how the energy transforms, as:
\begin{align*}
    p'^\mu = \Lambda^\mu_{\diamond \alpha}p^\alpha
\end{align*}

\section{Newton's First Law in SR}
\begin{center}
    A free particle moves with constant $u^\mu$ 
\end{center}
In mathematical terms:
\begin{align*}
    \frac{du^\mu}{d \tau} = \frac{d^2 x^\mu}{d \tau^2} = 0 \text{ for a free particle}
\end{align*}
where the derivative of $u^\mu$ generalizes the concept of acceleration:
\begin{align*}
    a^\mu = \frac{d u^\mu}{d \tau} = \frac{d^2 x^i}{d \tau^2} = 0 
\end{align*} 
and $a^\mu$ is called $4$-acceleration.

\section{Variational Principle}
It's common knowledge that \q{particles in GR move in straight lines, but straight lines are bent by gravity}. We now want to make sense of this concept, by talking about motion along a path that \textit{minimizes} a certain quantity, and introducing the definition of \textit{geodesics}.\\

Recall Snell's Law of Refraction:
\begin{align*}
    \frac{\sin \theta_r}{\sin \theta_i} = \frac{n_i}{n_2}  
\end{align*}
A ray of light passing from a medium with refraction index of $n_1$ at an angle $\theta_i$ wrt the interface to another medium with index of $n_2$ is bent at an angle $\theta_r$.\\

This result has another (elegant) interpretation:
\begin{center}
    Light chooses the path that minimizes the time to go from the source $S$  to the receiver $R$.
\end{center}

This concept generalizes for a free particle in special relativity. Consider a free particle $P$  that moves from $A$ and arriving at $B$. We will prove that $P$ travels along the worldline that minimizes the \textbf{proper time} $\tau_{AB}$. We will note that this statement is completely equivalent to requiring:
\begin{align*}
    \frac{d^2 x^\mu}{d \tau^2} = 0 
\end{align*}  

Recall that, near the minimum of a function, the function \q{does not change}, meaning that the first derivative is zero:
\begin{align*}
    0 = df = \dv{f}{x} \Big|_{x_0 } dx = 0 \text{ if } \dv{f}{x} \Big|_{x_0 } = 0 \text{ if } x_0 \text{ is a minimum}
\end{align*}

Now consider a small \q{variation} from the initial worldline $x^\mu$, represented by $x^\mu+ dx^\mu$. We want to prove that:
\begin{align*}
    \tau[x^\mu + dx^\mu] - \tau[x^\mu] = d \tau = 0
\end{align*}
which is only true if $x^\mu$ minimizes the proper time.\\
Basically: if the correct $x^\mu$ of a free particle minimizes $\tau$, we expect that by deforming a little the trajectory the value of $\tau$ does not change (at the first order).\\

Let's start by computing the proper time between $A$ and $B$:
\begin{align*}
    \tau &= \int_{A}^{B} d \tau = \int_{A}^{B} d \tau \frac{d \tau^2}{d \tau^2} =\\
    &= \int_{A}^{B} d \tau \frac{dt^2 - \frac{1}{c^2} dx^2 - \frac{1}{c^2} dy^2 - \frac{1}{c^2} dz^2}{d \tau^2} =\\
    &= \int_{A}^{B} d \tau \left[\left(\frac{d t}{d \tau} \right)^2 -\frac{1}{c^2} \left(\frac{dx}{d \tau} \right)^2 - \frac{1}{c^2}\left(\frac{dy}{d \tau} \right)^2 -\frac{1}{c^2} \left(\frac{dz}{d \tau} \right)^2   \right]   
\end{align*}
where $x$, $y$, $z$ are the entries in $x^\mu(\tau)$, as functions of $\tau$: $t(\tau)$, $x(\tau)$, $y(\tau)$, $z(\tau)$.\\

To simplify calculations, let's start by considering a specific kind of deformation, that happens along the first axis:
\begin{align*}
    dx^0 &= 0 \\
    dx^1 &= \delta x^1(\tau) \\
    dx^2 &= 0 \\
    dx^3 &= 0 
\end{align*}
Then we have:
\begin{align*}
    \tau_{AB} [x+dx] &= \int_{A}^{B} d \tau \left[\left(\frac{dt}{d \tau} \right)^2 
    -\frac{1}{c^2} \left(\frac{dx}{d \tau} + \frac{d \delta x}{d \tau}  \right)^2
    -\frac{1}{c^2} \left(\frac{dy}{d \tau} \right)^2
    -\frac{1}{c^2} \left(\frac{dz}{d \tau} \right)^2
    \right] =\\
    &= \int_{A}^{B} d \tau \left[\hlc{Yellow}{\left( \frac{dt}{d \tau} \right)^2 
    }\hlc{Yellow}{-\frac{1}{c^2} \left(\frac{dx}{d \tau} \right)^2 }- \frac{2}{c^2}\frac{dx}{d \tau} \frac{d \delta x}{d \tau}
    -\cancel{\frac{1}{c^2}\left(\frac{d \delta x}{d \tau} \right)^2 
    }\hlc{Yellow}{-\frac{1}{c^2} \left(\frac{dy}{d \tau} \right)^2
    -\frac{1}{c^2} \left(\frac{dz}{d \tau} \right)^2}
    \right] = \\
    &\underset{(a)}{=}  \hlc{Yellow}{\tau_{AB}[x]} - \frac{2}{c^2} \int_{A}^{B} d \tau \frac{dx}{d \tau} \frac{d \delta x}{d \tau}    \\
\end{align*}
In (a) we removed the second order infinitesimal, and then recognized the proper time of the unperturbed trajectory.\\
We can now compute:
\begin{align*}
    d \tau_{AB} &= \tau_{AB}[x + dx] - \tau_{AB}[x] = - \frac{2}{c^2} \int_{A}^{B} d \tau \frac{dx}{d \tau} \frac{d \delta x}{d \tau} =\\
    &= -\frac{2}{c^2} \frac{dx}{d \tau} \delta x\Big|_A^B + \frac{2}{c^2} \int_{A}^{B} d \tau \underbrace{\frac{d^2x}{d \tau^2}}_{= 0} \delta x  
\end{align*}
The first term vanishes as the deformation does not alter the starting and ending points ($A$ and $B$) of the original particles, meaning that $\delta x$ is $0$ in $A$ and $B$. Then, the integral in the second term also vanishes, because for a \textbf{free particle} we have:
\begin{align*}
    \frac{d^2 x}{d \tau^2} \equiv 0 
\end{align*}    

So, summarizing, we showed that:
\begin{align*}
    \text{Free motion} \Leftrightarrow \frac{d^2 x^\mu}{d \tau^2} = 0 \Leftrightarrow \tau_{AB}[x] \text{ is minimum} 
\end{align*}

Note that also Newtonian Mechanics can be expressed in a variational form, by defining the \textbf{action} $S$ as:
\begin{align*}
    S = \int dt [\text{Kinetic Energy} - \text{Potential Energy}]
\end{align*} 
and showing that:
\begin{align*}
    \text{Newton's Laws} \Leftrightarrow \delta S = 0 \text{ (Action is minimum)}
\end{align*}

\section{Motion of Light Rays}
Let's try to replicate the previous passages for a light ray:
\begin{align*}
    u^\mu = \frac{dx^\mu}{d \tau}; \qquad d \tau = \frac{\sqrt{-ds^2}}{c} = 0 
\end{align*}
which is a problem, because we cannot divide by zero. In fact, when we defined $u^\mu$ we needed $v < c$, and we assumed a \textit{massive particle}.\\

So, we need a different definition.\\
If a ray of light moves along the $x$-axis, we define:
\begin{align*}
    u^\mu= (1,1,0,0) 
\end{align*}
and then:
\begin{align*}
    x^\mu = \lambda u^\mu
\end{align*}
where $\lambda$ is called an affine coordinate, such that:
\begin{align*}
    dx\, dx = \lambda^2 du\,du = 0
\end{align*} 

For the energy-momentum of light we know that a single quantum of light (photon) has energy:
\begin{align*}
    E = \hbar \omega; \qquad \hbar = \frac{h}{2\pi} 
\end{align*} 
with $h$ being Planck's constant, and:
\begin{align*}
    \omega = \frac{2\pi}{T}  = 2\pi f
\end{align*}
with $T$ the period and $f$ the frequency of the considered ray of light.\\
Then the momentum is:
\begin{align*}
    \vec{p} = \frac{\hbar }{c}  \vec{k}
\end{align*}  
with $\vec{k}$ the wave vector.\\

So we arrive at the $4$-vector:
\begin{align*}
    p^\mu = \left(\frac{\text{Energy}}{c}, \text{Momentum} \right) = \left(\frac{\hbar\omega }{c}, \frac{\hbar \vec{k}}{c}  \right) = \frac{\hbar}{c} k^\mu
\end{align*}
where:
\begin{align*}
    k^\mu = (\omega, \vec{k})
\end{align*}
is also a $4$-vector.\\
Note that:
\begin{align*}
    p^2 = 0 \Rightarrow k^\mu k_\mu = 0 \Rightarrow -\omega^2 + \vec{k}^2 = 0 \Rightarrow \omega = |\vec{k}|
\end{align*}

\section{Doppler Effect}
Consider a source $S$ emitting radiation at frequency $f$ (wrt a frame of reference $O'$  where $S$ is at rest). Let $O$ be another inertial frame of reference moving at velocity $v$ wrt $O'$ (further away from $S$).\\
If the radiation is emitted along the $x$ axis, then:
\begin{align*}
    k'^\mu = (\omega, \omega, 0, 0)
\end{align*}
And $O$ observes:
\begin{align*}
    k^\mu = \Lambda^\mu_{\diamond \nu} k^\nu
\end{align*}
Explicitly:
\begin{align*}
    \omega' = k'^0 = \Lambda^0_{\diamond \nu} k^\nu = \Lambda^0_{\diamond 0} k^0 + \Lambda^0_{\diamond 1} k^1 = \gamma \omega + (-\gamma \beta) \omega = (1-\beta) \gamma \omega
\end{align*}
and then:
\begin{align*}
    f = \frac{\omega}{2\pi}; \qquad f' = \frac{\omega'}{2 \pi} = (1- \beta) \gamma f  
\end{align*}
We finally arrive at:
\begin{align*}
    f' = \frac{1-\frac{v}{c} }{\sqrt{1-\frac{v^2}{c^2}} } f
\end{align*}
At small speed, $\gamma \approx 1$:
\begin{align*}
    f' \approx \left(1-\frac{v}{c} \right) f
\end{align*}
Note that $f' < f$ when source \& observer are going away from each other.\\
Also, we get the same result if we consider the receiver at rest, and the source moving in the opposite direction (as the principle of relativity implies: there is no absolute space, only the relative motion between observers matters).\\


If the observer is moving \textit{towards} the source, we will just have to substitute $v \leftrightarrow -v$, obtaining:
\begin{align*}
    f' = \frac{1 + \frac{v}{c} }{\sqrt{1 - \frac{v^2}{c^2}} }f 
\end{align*}  




\end{document}
