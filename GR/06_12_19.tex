%&latex
%
\documentclass[../template.tex]{subfiles}
\begin{document}

\section{Frame Dragging}
\lesson{?}{06/12/19}
Recall the line element of a slowly rotating geometry:
\begin{align*}
    \dd{s}^2 = \eta_{\mu \nu} \dd{x^\mu} \dd{x^\nu} - \frac{4GJ}{r} \sin^2 \theta \dd{t} \dd{\varphi} 
\end{align*}
We consider a gyroscope falling in the direction of the axis of rotation. We then use a cartesian coordinate system (as $\hat{z}$ is singular in spherical coordinates). So, we need to change coordinates in the line element. Starting from the polar coordinates definition:
\begin{align*}
    \begin{cases}
        x = r\sin \theta \cos \varphi\\
        y = r \sin \theta \sin \varphi
    \end{cases} \Rightarrow \varphi = \arctan \frac{y}{x} 
\end{align*}
Differentiating:
\begin{align*}
    \dd{\varphi} = \frac{1}{1+y^2/x^2} \dd{}\left(\frac{y}{x} \right) = \frac{x^2}{x^2 + y^2} \frac{\dd{y} x - y \dd{x}}{x^2} = \frac{x\dd{y} - y \dd{x}}{r^2 \sin^2 \theta} = \dd{\varphi}   
\end{align*}
Substituting in $\dd{s}^2$ leads to:
\begin{align*}
    \dd{s}^2 = \eta_{\mu \nu} \dd{x^\mu}\dd{x^\nu} - \frac{4GJ}{r^2} \frac{x \dd{y} - y\dd{x}}{r} \dd{t}  
\end{align*} 
We then expand in first order of $J$:
\begin{align*}
    g_{\mu \nu} = \underbrace{\eta_{\mu \nu}}_{O(J^0)} + \underbrace{\delta g_{\mu \nu}}_{O(J^1)}  
\end{align*} 
with:
\begin{align*}
    \delta g_{01} &= \delta g_{10} = \frac{2GJy}{(x^2 + y^2 + z^2)^{3/2}} \\
    \delta g_{02} & \delta g_{20} = -\frac{2GJx}{(x^2 + y^2 + z^2)^{3/2}} 
\end{align*}
We can then compute the Christoffel's symbols in a perturbative manner:
\begin{align*}
    \Gamma^\mu_{\alpha \beta} = \frac{1}{2} \eta^{\mu \lambda} \left[\delta g_{\lambda \beta, \alpha} + \delta g_{\alpha \lambda, \beta} - \delta g_{\alpha \beta, \lambda}\right] + O(J^2)
\end{align*}
The $4$-velocity for the falling gyroscope will be only on $\hat{z}$:
\begin{align*}
    u^\alpha = (u^t, 0, 0, u^z)
\end{align*}  
And we choose the initial spin aligned with $\hat{x}$:
\begin{align*}
    S^\alpha_{\mathrm{in} } = (0, S_{\mathrm{in} }^x, 0, 0)
\end{align*} 

We now show that $S^z \equiv 0$ at all times. We start with the spin-equation:
\begin{align*}
    \dv{S^z}{\tau} + \Gamma^3_{\alpha \beta} u^\alpha S^\beta
 = 0
\end{align*} 
Looking at $u^\alpha$, we will have non-zero results only for $\alpha = 0,3$:
\begin{align*}
    \dv{S^z}{\tau} + \Gamma_{0 \beta}^3 u^t S^\beta + \Gamma_{3 \beta}^3 u^z S^\beta = 0
\end{align*}  
Then we compute the required Christoffel symbols:
\begin{align*}
    \Gamma_{0 \beta}^3 = \frac{1}{2} \eta^{33} \left[\cancel{\delta g_{3 \beta,0}} + \cancel{\delta g_{03, \beta} }- \delta g_{0 \beta,3}\right] 
\end{align*}
For the last term we have two options:
\begin{align*}
    \delta g_{01,3} \Big|_{x=y=0} = \pdv{z} \frac{2GJy}{(x^2 +y^2 + z^2)} \Big|_{x=y=0} = 0 
\end{align*}
and the same happens for $\delta g_{02,3}$. Note that they are not \textit{always} null, but they vanish \textit{at the rotation axis}.\\

Then:
\begin{align*}
    \Gamma^3_{3 \beta} = \frac{1}{2} \eta^{33} (\cancel{\delta g_{3 \beta, 3} }+ \bcancel{\delta g_{33, \beta}} - \cancel{\delta g_{3 \beta,3}}) = 0
\end{align*}
and so:
\begin{align*}
    \dv{S^z}{\tau} = 0 \Rightarrow S^z = 0 \text{ at all times}
\end{align*}

The same happens with $S^t$, as we now show:
\begin{align*}
    \dv{S^t}{\tau} + \Gamma^0_{\alpha \beta} u^\alpha S^\beta = 0
\end{align*} 
As $\alpha = 0,3$:
\begin{align*}
    \dv{S^t}{\tau} + \Gamma^0_{0 \beta} u^t S^\beta + \Gamma_{3 \beta}^0 u^z S^\beta = 0
\end{align*} 
The Christoffel symbols:
\begin{align*}
    \Gamma_{0 \beta}^0 &= \frac{1}{2} \eta^{00} \left( \cancel{\delta g_{0 \beta,0} }+ \bcancel{\delta g_{00, \beta}} -\cancel{ \delta g_{0 \beta,0}}\right) = 0\\
    \Gamma_{3 \beta}^0 &= \frac{1}{2} \eta^{00} \left(\hlc{Yellow}{\delta g_{0 \beta,3} }+ \cancel{\delta g_{30, \beta}} - \bcancel{\delta g_{3 \beta ,0}}\right) = 0
\end{align*}
and the highlighted term, as seen before, vanishes on $\hat{z}$. So:
\begin{align*}
    \dv{S^t}{\tau} = 0 \Rightarrow S^t \equiv 0 \text{ at all times}
\end{align*}
Summarizing:
\begin{align*}
    u^\alpha = (u^t, 0,0,u^z) \qquad S^\alpha = (0,S^x(\tau), S^y(\tau),0)
\end{align*}
and $\bm{u} \cdot \bm{S} = 0$ immediately holds.

All that's left is to write the system of differential equations for $S^x$ and $S^y$:
\begin{align*}
    \begin{cases}
        \dv{S^1}{\tau} + \Gamma_{\alpha \beta}^1 u^\alpha S^\beta = 0\\
        \dv{S^2}{\tau} + \Gamma_{\alpha \beta}^2 u^\alpha S^\beta = 0
    \end{cases}
\end{align*}   
Since only $\delta g_{01,10,02,20} \neq 0$, one lower index in $\Gamma^{1,2}_{\alpha \beta}$ must be $0$. However, $\beta = 0$ multiplies $S^t = 0$, and so the $0$ index must be $\alpha$. This leads to:
\begin{align*}
    \begin{cases}
        \dv{S^x}{\tau} + \Gamma^1_{0 \beta} \dv{t}{\tau} S^\beta = 0\\
        \dv{S^y}{\tau} + \Gamma^2_{0 \beta} \dv{t}{\tau} S^\beta = 0
    \end{cases}
\end{align*}      
Changing variables and expanding:
\begin{align*}
    \begin{cases}
        \dv{S^x}{t} + \Gamma^1_{01} S^x + \Gamma^1_{02} S^y = 0\\
        \dv{S^y}{t} + \Gamma_{01}^2 S^x + \Gamma^2_{02} S^y = 0
    \end{cases}
\end{align*}
The Christoffel symbols:
\begin{align*}
    \Gamma^1_{01} &= \frac{1}{2} \eta^{11} (\delta g_{11,0} + \delta g_{01,1} - \delta g_{\vec{01,1}}) = 0\\
    \Gamma^2_{02} &= 0\\
    \Gamma^1_{02} &= \frac{1}{2} \eta^{11} \left[ \cancel{\delta g_{12,0} }+ \delta g_{01,2} - \delta g_{02,1}\right] \\
    \Gamma^2_{01} &= \frac{1}{2} \eta^{22} \left(\cancel{\delta g_{21, 0}} + \delta g_{\vec{02,1}} - \delta g_{01,2}\right) 
\end{align*}
and so $\Gamma_{01}^2 = - \Gamma_{02}^1$ are the only non-vanishing symbols.
\begin{align*}
    \Gamma_{02}^1 &= \frac{1}{2} \left[\pdv{y} \frac{2G J y}{(x^2 +y^2 + z^2)^{3/2}} - \pdv{x} \frac{-2GJx}{(x^2 + y^2 +z^2)^{3/2}}  \right] \Big|_{x=y=0} = \\
    &\underset{(a)}{=} \frac{2GJ}{z^3} 
\end{align*} 
where in (a) we note that, to have non-zero results, the derivatives must \textit{kill} the variables at the numerators. This leads to \textit{two equal terms}.  

We can finally substitute back in the equations:
\begin{align*}
    \begin{cases}
        \dv{S^x}{t} + \frac{2GJ}{z^3} S^y = 0\\
        \dv{S^y}{t} - \frac{2GJ}{z^3} S^x = 0  
    \end{cases}
\end{align*}
If $z$ were constant, this would lead to an \textit{harmonic oscillator} with an \textit{instantaneous angular velocity} of the spin-vector $\Omega_{\mathrm{LT}} = 2 GJ/z^3$ (Lens-Thirring). However, $z$ is a function of time, so this will be valid only for a \textit{very slowly moving object}.

In the case of a gyroscope \textit{not on the rotation axis}, but at a direction $\bm{e}^{\hat{r}}$ we would have (calculations omitted):
\begin{align*}
    \bm{\Omega}_{\mathrm{LT}} = \frac{GJ}{c^2 r^3}\left[3 (\bm{J} \cdot \bm{e}^{\hat{r}}) \bm{e}^{\hat{r}} - \bm{J}\right] 
\end{align*}  
Note that this reduces to the formula we found if $\bm{e}^{\hat{r}} \parallel \bm{J}$, and also has the same pattern as an electric field of an electric dipole. 

\section{Kerr geometry (1963)}
The Kerr metric is a \textit{vacuum solution} ($R_{\mu \nu } = 0$) of a \textit{rotating spherical mass}.
\begin{align*}
    \dd{s^2} &= -\left(1-\frac{2GMr}{\rho^2} \right)\dd{t^2} - \frac{4GM a r\sin^2 \theta}{\rho^2} \dd{t} \dd{\varphi} + \frac{\rho^2}{\Delta} \dd{r^2} + \rho^2 \dd{\theta^2} +\\
    &\quad+ \left(r^2 + a^2 + \frac{2GMr a^2 \sin^2 \theta}{\rho^2} \right)\sin^2 \theta \dd{\varphi^2} \\
    a &\equiv \frac{J}{M} \qquad \rho^2 \equiv r^2 + a^2 \cos^2 \theta \qquad \Delta \equiv r^2 - 2GMr + a^2 
\end{align*}  

\textbf{Some properties} 
\begin{itemize}
    \item Note that in $c=1$ units, velocities are dimensionless. So: $[J] = [\text{Mass}][\text{Length}]$. Note that $a \equiv J/M$, and $[a] = \text{Length}$.    
    \item $O(a^0)$ term is Schwarzschild
    \item $O(a^0) + O(a^1)$ leads to the slowly rotating geometry previously seen.
    \item $r \gg GM$ we have,at first order in $1/r$:
    \begin{align*}
        \dd{s^2} = -\left(1-\frac{2GM}{r} \right) \dd{t^2} -\frac{4 G M a}{r} \sin^2 \theta \dd{t} \dd{\theta}  + \left(1+\frac{2GM}{r} \right) \dd{r^2} + r^2 (\dd{\theta^2} + \sin^2 \theta) \dd{\varphi^2}
    \end{align*}  
    So it is possible to orbit with a gyroscope far away from $M$ and measure $M$ an $J$ using this line element.     
    \item As the metric is stationary and axis-symmetric, there are still two Killing vectors:
    \begin{align*}
        \xi^\alpha = (1,0,0,0) \qquad \xi^\alpha = (0,0,0,1)
    \end{align*}   
    \item Symmetry $\theta \to \pi - \theta$
    \item Real singularity at $\rho = 0$, meaning that $r= 0$ \textbf{and} $\cos \theta = 0 \Rightarrow \theta= \pi/2$. Note that $r=0$ \textit{is not a single point}: the metric \textit{changes} depending on $\rho$, and $r = 0$ does not identify a single value of $\rho$ (there are different properties by approaching $r=0$ from different directions, and in fact the singularities appears only if $\theta = \pi/2$). In a certain sense, the rotation \textit{smears} the single point $r=0$ into \textit{a disk}. Another way to see it is by looking at $\dd{s}^2$ and noting that two points at $r=0$ and different values of $\theta$ are \textit{separated by a non-zero distance}, as $\rho \neq 0$.       
    \item There is a \textbf{coordinate singularity} (horizon) when $\Delta = 0$ (generalization of what happens in Schwarzschild).
    \begin{align*}
        \Delta = 0 = r^2 -2 GM r + a^2 = 0 \Rightarrow r_{\pm} = GM \pm \sqrt{G^2 M^2 - a^2}
    \end{align*} 
    So there are \textit{two horizons}, and we will limit our discussion at the outer one ($r_+$). Note that if $a=0$ we get back $r_+ = 2GM$ (Schwarzschild horizon). \\
    Note that if $a > GM$ \textit{there is no horizon}, producing a \textbf{naked singularity}. We postulate (\textbf{cosmic censorship}) that \textit{naked singularities do not exist in nature}. This principle \textit{is not proven}, but it's suggested by the mechanism of black-hole formation.\\
    The case where $a = GM$ is called \textit{extreme Kerr solution}.        
\end{itemize}

We show now that $r = r_+$ defines a \textit{null surface}, that is a surface separating a region where light can go to $r \to \infty$ from a region where light goes to $r \to 0$. So light entering it is \q{trapped inside} this surface. 

\subsection{Null surfaces}
Consider a light cone in Minkowski spacetime, which is defined by $r=t$ ($3$-surface). Any vector on the light cone is of the form (in $\{t,r,\theta,\varphi\}$ coordinates):
\begin{align*}
    x^\mu = (\alpha, \alpha, \beta, \gamma) = \alpha\underbrace{(1,1,0,0)}_{l^\alpha}  + \beta\underbrace{(0,0,1,0)}_{m^\alpha}  + \gamma\underbrace{(0,0,0,1)}_{n^\alpha} 
\end{align*}   
Note that:
\begin{align*}
    \bm{l}\cdot \bm{l} = l^\mu \eta_{\mu \nu}l^\nu = (l^0)^2 + (l^1)^2 = 0
\end{align*}
and so $l^\alpha$ is a null vector. Also $\bm{m}\cdot \bm{m} >0$ and $\bm{n}\cdot \bm{n} >0$ are \textit{space-like} vectors, and $\bm{l}\cdot \bm{m} = \bm{l} \cdot \bm{n} = \bm{m} \cdot \bm{n} = 0$ (they are orthogonal to each other). So $\{l^\alpha, m^\alpha, n^\alpha\}$ is a basis for all vectors on a light cone (and they are all elements of the tangent space).

Then consider, in the Schwarzschild metric:
\begin{align*}
    l^\alpha = (1,0,0,0) \qquad m^\alpha = (0,0,1,0) \qquad n^\alpha = (0,0,0,1)
\end{align*}
and $m^\alpha g_{\alpha \beta} m^\beta >0$, $\bm{m}\cdot \bm{m}, \bm{n}\cdot \bm{n} > 0$, $\bm{l}\cdot \bm{m} = \bm{l} \cdot \bm{n} = \bm{m} \cdot \bm{n} = 0$. Also:
\begin{align*}
    \bm{l}\cdot \bm{l} = l^\alpha g_{\alpha \beta} l^\beta = g_{00} \Big|_{r=2GM} = 0
\end{align*}
And so $\{l^\alpha, m^\alpha, n^\alpha\}$ are a basis for the null surface at the Schwarzschild horizon.

Similarly, in the Kerr metric at the $r= r_+ = GM + \sqrt{G^2 M^2 - a^2}$ horizon, the following vectors define a null surface:
\begin{align*}
    l^\alpha &= (1,0,0, \Omega_H) \qquad \Omega_H \equiv \frac{a}{2GMr_+} \\
    m^\alpha &= (0,0,1,0)\\
    n^\alpha &= (0,0,0,1)
\end{align*} 
Note that, while in Schwarzschild light \textit{trapped} in the horizon \textit{just moves in time}, for the Kerr geometry light is \textit{orbiting} the blackhole with angular velocity $\Omega_H$.


Let's now look at the Kerr horizon $r=r_+$ at fixed $t$ (the value is not important, as the metric is stationary), so to have a 2D surface ($\theta$, $\varphi$). It can be shown that, on the horizon:
\begin{align*}
    \dd{s^2} \Big|_{r=r_+} = \rho_+^2 \dd{\theta^2} + \left(\frac{2GM r_+}{\rho_+} \right)^2 \sin^2 \theta \dd{\varphi^2}
\end{align*}

Note that it is not spherical $\rho_+^2 (\theta) = r^2 + a^2 \cos^2 \theta$, as the equator is \textit{greater} than the corresponding equator for a sphere. To show this, we compute the \textit{length} of the equator (motion at $\theta= \pi/2  $ and $\varphi \in [0, 2\pi)$) and that of a full meridian (fixed $\varphi$, $\theta\colon 0 \to \pi \to 0$). For a sphere, the equator and a circle going through the poles \textit{have the same length}. However, for the Kerr horizon, we will see that:
\begin{align*}
    L_{\mathrm{equator}} > L_{\mathrm{N-S-N} }
\end{align*}    
(N-S-N stands for the path starting at the North pole, going to the South pole and returning to the North).

\begin{align*}
    L_{\mathrm{equator}} &= \int_0^{2 \pi} \dd{\varphi} \sqrt{g_{\varphi \varphi} } \Big|_{\theta= \pi/2 } = 2GM \int_0^{2\pi} \dd{\varphi} = 4\pi GM\\
    L_{\mathrm{NSN}} &= 2 \int_0^\pi \dd{\theta} \sqrt{g_{\theta\theta}} = 2 \int_0^{\pi} \dd{\theta} \sqrt{r^2_+ + a^2 \cos^2 \theta}
\end{align*}
For the second one we would need an elliptic integral. To simplify things, we expand it in the limit of small $a$: 
\begin{align*}
    L_{\mathrm{NSN}} \approx 2 \int_0^\pi \dd{\theta} \left[2GM + \frac{a^2}{4GM}(-2 + \cos^2 \theta) + O(a^4)\right] 
\end{align*}
As we are integrating $\cos^2 \theta$ over a period, we can substitute it with its average ($1/2$), leading to:
\begin{align*}
    =  \left[4GM - \frac{3a^2}{4 GM} \right] \pi 
\end{align*}  
And so $L_{\mathrm{equator} } > L_{NSN}$ for $a \neq 0$ (they are the same for $a=0$).   
\end{document}
