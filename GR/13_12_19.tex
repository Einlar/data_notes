%&latex
%
\documentclass[../template.tex]{subfiles}
\begin{document}

\section{Concepts}
\lesson{?}{13/12/19}
\subsection{Planck's scale}
The Compton wavelength for an energetic particle is defined as:
\begin{align*}
    \lambda = \frac{\hbar c}{\mathcal{E}} 
\end{align*}
Recall the expression for the Schwarzschild radius:
\begin{align*}
    r_s = \frac{GM}{c^2} = \frac{G \mathcal{E}}{c^4}  
\end{align*}
If $\lambda < r_S$ we are in \textbf{quantum gravity} regime. This happens when:
\begin{align*}
    \frac{\hbar c}{\mathcal{E}} = \frac{G \mathcal{E}}{c^4} \Rightarrow \mathcal{E}_p=\sqrt{\frac{\hbar c^5}{G} } = \SI{1.96e9}{\J}  
\end{align*}
This is called \textbf{Planck energy}. Beyond this energy scale, quantum field theory breaks down, as the particles \textit{collapse} into black holes.

Recall that $\SI{1}{\eV} = \SI{1.6e-19}{\J}$, and so:
\begin{align*}
    \mathcal{E}_p = \SI{1.22e19}{\giga\eV}
\end{align*} 

In $c=1$ units, this energy corresponds also to a mass, denoted as \textbf{Planck mass}.   We also introduce the \textit{reduced planck mass} defined as:
\begin{align*}
    M_p = \frac{\mathcal{E}_p}{\sqrt{8 \pi}} = \SI{2.43e19}{\giga\eV} 
\end{align*} 
\subsection{Natural units}
We start by defining $c=1$ as a \textit{dimensionless quantity}, meaning that \textit{lengths} and \textit{times} have the same dimension.

Also setting $\hbar=1$ means that \textit{angular momentum} is \textit{dimensionless}:
\begin{align*}
    1 = [\mathrm{Angular\ Momentum}] = [\mathrm{Length} ][\mathrm{Velocity} ][\mathrm{Mass}]
\end{align*}   
This means that:
\begin{align*}
    [\mathrm{Length} ] = [\mathrm{Time} ] = \frac{1}{[\mathrm{Mass}] } = \frac{1}{[\mathrm{Energy} ]}  
\end{align*}
We can now rewrite Einstein's equations in natural units. Recall that:
\begin{align*}
    G_{\mu \nu} = 8 \pi G T_{\mu \nu}
\end{align*}
Then from:
\begin{align*}
    M_p = \frac{1}{\sqrt{8 \pi}} \underbrace{\frac{1}{\sqrt{G}}}_{\mathcal{E}_p} \Rightarrow 8 \pi G = \frac{1}{M_p^2}   
\end{align*}
and so:
\begin{align*}
    G_{\mu \nu} = \frac{T_{\mu \nu}}{M_p^2} 
\end{align*}

\section{Cosmology}
Consider a \textit{homogeneous} and \textit{isotropic} universe, i.e. the Friedman Lemaitre Robertson Walker solution (FLRW).
\begin{itemize}
    \item \textbf{Homogeneous}: any point is \q{like} every other point
    \item \textbf{Isotropic}: any direction is \q{like} every other direction 
\end{itemize}
For example, a system that is \textit{homogeneous but not isotropic} is the volume inside a charged capacitor (the preferred direction is given by the electric field). An example for a system that is \textit{isotropic but not homogeneous} is the field generated by a point-charge, as it is spherically symmetric.

\medskip

There are three possible cases of curvature:
\begin{itemize}
    \item \textbf{Flat}: $0$ curvature, meaning that the angles of any triangle add up to $180^\circ$.
    \item \textbf{Closed}: Positive curvature, where the angles of a triangle add up to $>180^\circ$
    \item \textbf{Open}: negative curvature (like the surface of a saddle), where the angles of a triangle add up to $<180^\circ$   
\end{itemize}
Experimentally, our universe is very close to being \textit{flat}. However, as very large radii $R$ of curvature generate spaces that are \textit{very close to flat}, we cannot know for sure.

\medskip

The FLRW solution supposes a \textit{flat} universe:
\begin{align*}
    \dd{s^2} = -\dd{t^2} + a^2(t) [\dd{x^2} + \dd{y^2} + \dd{z^2}]
\end{align*} 
The only possible non-zero Christoffel's symbols are:
\begin{align*}
    \Gamma_{ij}^0, \qquad \Gamma_{0j}^i
\end{align*}
Expanding:
\begin{align*}
    \Gamma_{ij}^0 = \frac{1}{2} g^{00} (\cancel{g_{0j,i}} + \cancel{g_{i0,j} }- g_{ij,0}) = \frac{1}{2}(-1)(-1) \pdv{t} (a^2 \delta_{ij})  = a\dot{a} \delta_{ij}
\end{align*}
as $g_{00} = 1$, $g_{0i} = 0$ and $g_{ij} = a^2 \delta_{ij}$.
\begin{align*}
    \Gamma_{0j}^i = \frac{1}{2} g^{ik} g_{jk,0} = \frac{1}{2} a^{-2} \delta_{ik} \pdv{t} (a^2 \delta_{kj}) = \frac{\dot{a}}{a} \delta_{ij}   
\end{align*}
Then $\Gamma^0_{i i} = 3 a \dot{a}$ (repeated indices denote a sum), and $\Gamma_{0i}^i = 3 \dot{a}/a$.

We can now compute the Ricci tensors. $R_{0i}$ is immediately $0$, as there are no non-vanishing $g_{0i}$, and also no non-zero spatial derivatives - so there isn't anything that can contribute to indices $0i$. We also know that $R_{ij}$ must be proportional to $\delta_{ij}$ - as all spatial derivatives vanish.

Then, recall the full formula:
\begin{align*}
    R_{\mu \nu} = \partial_\alpha \Gamma^\alpha_{\mu \nu} - \partial_\nu \Gamma_{\mu \alpha}^\alpha + \Gamma_{\mu \nu}^\lambda \Gamma^\alpha_{\lambda \alpha} - \Gamma^\lambda_{\mu \alpha} \Gamma^\alpha_{\nu \lambda}
\end{align*}
and then we can compute:
\begin{align*}
    R_{00} =\cancel{ \partial_\alpha \Gamma^\alpha_{00}} - \partial_0 \Gamma_{0 \alpha}^\alpha + \cancel{\Gamma_{00}^\lambda \Gamma_{\lambda \alpha}^\alpha }- \Gamma_{0 \alpha}^\lambda \Gamma_{0 \lambda}^\alpha
\end{align*}
as $\Gamma_{00}^\alpha = 0$. Then, for the second term, $\alpha$ can only range over the spatial indices, and the same for the last term:
\begin{align*}
    R_{00} &= -\partial_0 \Gamma_{0i}^i - \Gamma_{0j}^i \Gamma_{0i}^j = - \partial_0 \left(\frac{3 \dot{a}}{a} \right) - \frac{\dot{a}}{a} \delta_{ij} \frac{\dot{a}}{a} \delta_{ji} =\\
    &= -\frac{3 \ddot{a}}{a} + \cancel{\frac{3 \dot{a}^2}{a^2}} - \cancel{3 \frac{\dot{a}^2}{a^2}} = - 3\frac{\ddot{a}}{a}    
\end{align*}  
as $\delta_{ij} \delta_{ji} = (\mathbb{I}_3)_{i i} = 3$.

Then:
\begin{align*}
    R_{ij} = \partial_\alpha \Gamma_{ij}^\alpha -\cancel{\partial_j \Gamma_{i \alpha}}^\alpha + \Gamma_{ij}^\alpha \Gamma_{\alpha \lambda}^\lambda - \Gamma_{i \alpha}^\lambda \Gamma_{j \lambda}^\alpha
\end{align*}
where the second term vanishes are there are no non-zero spatial derivatives. Then $\Gamma_{ij}^\alpha \neq 0$ only for $\alpha = 0$, and the same for the second term. For the last, if $\alpha = 0$, $\lambda = k$, or if $\alpha = k$ then $\lambda = 0$:
\begin{align*}
    R_{ij} &= \partial_0 \Gamma^0_{ij} + \Gamma_{ij}^0 \Gamma_{0k}^k - \Gamma_{ik}^0 \Gamma_{j0}^k - \Gamma_{i0}^k \Gamma_{jk}^0 = \\
    &= \partial_0 [a \dot{a} \delta_{ij}] + a \dot{a} \delta_{ij} 3 \frac{\dot{a}}{a} - a \dot{a} \delta_{ik} \frac{\dot{a}}{a} \delta_{kj} - \frac{\dot{a}}{a} \delta_{ki} a \dot{a} \delta_{jk} =\\
    &=(a\ddot{a} + \dot{a}^2) \delta_{ij} + 3 \dot{a}^2 \delta_{ij} - \dot{a}^2 \delta_{ij} - \dot{a}^2 \delta_{ij} = (a\ddot{a} +2\dot{a}^2) \delta_{ij} = \\
    &= \left(\frac{\ddot{a}}{a} + \frac{2 \dot{a}^2}{a^2}  \right) a^2 \delta_{ij}
\end{align*}      
Summarizing, we have:
\begin{align*}
    R_{00} &= -\frac{3 \ddot{a}}{a}\\
    R_{ij} &= a^2 \delta_{ij} \left[\frac{2 \dot{a}^2}{a^2} + \frac{\ddot{a}}{a}  \right] 
\end{align*}  

The scalar curvature $R$ is then:
\begin{align*}
    R = g^{00} R_{00} + g^{ij} R_{ij} = \frac{3\ddot{a}}{a} + \frac{\delta_{ij}}{a^2} a^2 \delta_{ij} \left[\frac{2\dot{a}^2}{a^2} + \frac{\ddot{a}}{a} \right]   = \frac{6\dot{a}^2}{a^2} + \frac{6 \ddot{a}}{a}  
\end{align*} 
We then compute the Einstein tensor:
\begin{align*}
    G_{00} &= R_{00} - \frac{R}{2} g_{00} = -\cancel{\frac{3\ddot{a}}{a} }+ \frac{3\dot{a}^2}{a^2} + \cancel{\frac{3\ddot{a}}{a}}  = \frac{3\dot{a}^2}{a^2}  \\
    G_{ij} &= R_{ij} - \frac{R}{2} g_{ij} = a^2 \delta_{ij} \left[\frac{2 \dot{a}^2}{a^2} + \frac{\ddot{a}}{a}  \right] - a^2 \delta_{ij}\left[\frac{3 \dot{a}^2}{a^2} + \frac{3 \ddot{a}}{a}  \right] = a^2 \delta_{ij}\left[-\frac{\dot{a}^2}{a^2}  - \frac{2\ddot{a}}{a} \right]
\end{align*}

We fix the energy-momentum tensor, considering a universe filled by a \textit{perfect fluid}:
\begin{align*}
    T^{\mu \nu} = (\rho + p) u^\mu u^\nu + p g^{\mu \nu}
\end{align*} 
where $\rho$ is the energy density, $p$ is the pressure  and $u^\nu$ the $4$-velocity. Recall that:
\begin{align*}
    u^\mu = \left(\dv{t}{\tau} , \dv{\bm{x}}{\tau}\right)
\end{align*}   
We know that $u^0 = \dd{t}/\dd{\tau} > 0$, and also that:
\begin{align*}
    0 = G_{0i} = \frac{T_{0i}}{M_p^2} \qquad T_{0i} = (\rho + p)u^0 u^i+ \cancel{p g^{0i}}
\end{align*} 
and so $u^i = 0$, meaning that the \textit{cosmic fluid} is \textit{at rest}. Obviously, this can't happen in \textit{every frame}, meaning that there is a \textbf{special frame of reference}: that where the cosmic fluid is at rest. This means that, while the theory is Lorentz-invariant, the universe \textit{isn't}, because there is a uniquely identifiable special frame of reference (that of an observer stationary with respect to the Cosmic Microwave Background, meaning that he does not observe any dipole effect).

As $u^i = 0$, from $\bm{u}\cdot \bm{u} = g_{00} = -1$ we have $u^0 = 1$. This leads to:
\begin{align*}
    T^{00} &= \rho + p -p = \rho \Rightarrow T_{00} = \rho\\
    T_{ij} &= p g_{ij} = a^2 p \delta_{ij}\\
    T_{0i} &= 0
\end{align*}   

We can finally write the Einstein's equations:
\begin{align*}
    \frac{3\dot{a}^2}{a^2} = \frac{\rho}{M_p^2}\\
    a^2 \delta_{ij} \left[-\frac{2 \ddot{a}}{a}-\frac{\dot{a}^2}{a^2}  \right] = a^2 \delta_{ij} \frac{p}{M_p^2}   
\end{align*}
leading to:
\begin{align*}
    \begin{cases} \displaystyle
        \frac{3 \dot{a}^2}{a^2} = \frac{\rho}{M_p^2} \\
        \displaystyle -\frac{2 \ddot{a}}{a} - \frac{\dot{a}^2}{a^2} = \frac{\rho}{M_p^2}    
    \end{cases}
\end{align*}

Let's now verify Bianchi identity:
\begin{align*}
    \nabla_\mu G^{\mu \nu} = 0
\end{align*}
For $\nu = 0$:
\begin{align*}
    \partial_\mu G^{\mu 0} + \Gamma^\mu_{\mu \lambda} G^{\lambda 0} + \Gamma_{\mu \lambda}^0 G^{\mu \lambda} &= \partial_0 G^{00} + \Gamma^i_{i0} G^{00} + \Gamma_{ij}^0 G^{ij} =\\
    &= \partial_0 \left(\frac{3 \dot{a}^2}{a^2} \right) + \frac{3 \dot{a}}{a} \frac{3\dot{a}^2}{a^2} + a\dot{a} \delta_{ij}\frac{1}{a^2} \delta_{ij}\left[-\frac{2 \ddot{a}}{a} - \frac{\dot{a}^2}{a^2}   \right]  =\\
    &= 6 \frac{\dot{a}}{a} \left[\frac{\ddot{a}}{a} - \frac{\dot{a}^2}{a^2}  \right] + \frac{9\dot{a}^3}{a^3} - 6 \frac{\dot{a}}{a} \frac{\ddot{a}}{a} - \frac{3 \dot{a}^3}{a^3} = 0     
\end{align*} 
And for $\nu = 1$:
\begin{align*}
    \cancel{\partial_\mu G^{\mu i} }+\bcancel{ \Gamma^{\mu}_{\mu \lambda} G^{\lambda i}} + \bcancel{\Gamma^i_{\mu \lambda} G^{\mu \lambda} }= 0
\end{align*}
In the first term, a non-vanishing derivative implies $\mu = 0$, but $G^{\mu i} \neq 0$ for $\mu = i$, and so the term vanishes. A similar reasoning applies to the other two terms.  

So, the Bianchi identity is satisfied.

\medskip

We want now to verify $\nabla_\mu T^{\mu \nu} = 0$, which directly follows from Einstein's equation. This is a \textbf{local conservation law}. 

For $ \nu = 0$:
\begin{align*}
    \partial_\mu T^{\mu 0} + \Gamma_{\mu \lambda}^\mu T^{\lambda 0} + \Gamma_{\mu \lambda}^0 T^{\mu \lambda} &= \partial_0 T^{00} + \Gamma^i_{i0} T^{i0} + \Gamma_{ij}^0 T^{ij} =\\
    &= \partial_0 \rho + 3 \frac{\dot{a}}{a} \rho + \dot{a} a \delta_{ij} \frac{1}{a^2} \delta_{ij} p = \dot{\rho} + 3 \frac{\dot{a}}{a} \rho + 3 \frac{\dot{a}}{a} p  \underset{!}{=}  0    
\end{align*} 
So we get another equation:
\begin{align*}
    \dot{\rho} + 3 \frac{\dot{a}}{a} (\rho + p) = 0 
\end{align*}
As $\nabla_\mu T^{\mu \nu} = 0$ is a consequence of Einstein's equation, this relation we just found can be retrieved by manipulating the two equations we previously got.

First, let's examine quickly the remaining case for $\nu = i$:
\begin{align*}
    \cancel{\partial_\mu T^{\mu i}} + \bcancel{\Gamma^{\mu}_{\mu \lambda} T^{\lambda i} }+ \cancel{\Gamma^{i}_{\mu \lambda} T^{\mu \lambda} }= 0
\end{align*} 
which is trivially satisfied.

Then, taking the $\partial_0$ of the first Einstein's equation we get:
\begin{align*}
    6 \frac{\dot{a}}{a} \left[\frac{\ddot{a}}{a} - \frac{\dot{a}^2}{a^2}  \right] = \frac{\dot{\rho}}{M_p^2} 
\end{align*} 
And multiplying the second one by $3\dot{a}/a$:
\begin{align*}
    -6 \frac{\dot{a}}{a} \frac{\ddot{a}}{a} - 3 \frac{\dot{a}^3}{a^3} = \frac{3 \dot{a}}{a} \frac{p}{M_p^2}     
\end{align*} 
If we now add them, we can remove the $\ddot{a}$ term:
\begin{align*}
    -9 \frac{\dot{a}^3}{a^3} = \frac{\dot{\rho} + \frac{3 \dot{a}}{a} p }{M_p^2}  
\end{align*} 
Taking again the first equation and multiplying it by $3 \dot{a}/a$ leads to:
\begin{align*}
    \frac{9\dot{a}^3}{a^3} = \frac{3 \dot{a}}{a} \frac{\rho}{M_p^2}   
\end{align*} 
And adding these two equations makes $\dot{a}^3$ vanish:
\begin{align*}
    0 = \frac{\dot{\rho} + \frac{3 \dot{a}}{a} p + \frac{3 \dot{a}}{a} \rho  }{M_p^2} \Rightarrow \dot{\rho} + 3\frac{\dot{a}}{a} (\rho + p)  = 0
\end{align*} 
and this is the equation we got from $\nabla_\mu T^{\mu \nu} = 0$, proving that indeed it follows only from the other two.

To solve these equations, as one of them is redundant, we can \textit{consider only two of them at a time}. The easy choice is the first and the third, as they're both first order:
\begin{align*}
    \begin{dcases}
        \dot{\rho} + 3\frac{\dot{a}}{a} (\rho + p) = 0      \\
        \frac{\dot{a}^2}{a^2} = \frac{\rho}{3 M_p^2}   
    \end{dcases}
\end{align*} 

\subsection{Sources}
The \textbf{equation of state} reads:
\begin{align*}
    w \equiv \frac{p}{\rho} 
\end{align*} 
If $w = -1$, then $p = -\rho$ and so $\rho = \mathrm{constant}$. This means that, for an expanding universe, the energy density \textit{does not drop} - meaning that \textit{space itself} has energy, called \textbf{vacuum energy}.

This is coherent with particle physics, as \textit{vacuum} just denotes the \textit{lowest energy state} (e.g. the Higgs potential). However, this creates problems with quantum mechanics. 

\subsection{Einstein cosmological constant}
Einstein's equation can, in principle, be modified in the following way:
\begin{align*}
    G_{\mu \nu} + \textcolor{Red}{\Lambda g_{\mu \nu}} = \frac{1}{M_p^2} T_{\mu \nu} 
\end{align*}
which \textit{satisfies} Bianchi identity if $\Lambda$ is a constant:
\begin{align*}
    \nabla_\mu (\Lambda g^{\mu \nu}) = 0
\end{align*}  
as the metric is covariantly constant.

Rearranging:
\begin{align*}
    G_{\mu \nu} = \frac{1}{M_p^2} \left(T_{\mu \nu} - \Lambda M_p^2 g_{\mu \nu}\right) 
\end{align*}
and so we can interpret the role of $\Lambda$ (cosmological constant) as a \textit{source} of energy. The components of the Einstein tensor become:
\begin{align*}
    G_{00} &= \frac{1}{M_p^2} \left(\rho  + \Lambda M_p^2\right) \\
    G_{ij} &= \frac{1}{M_p^2} \left(a^2 \delta_{ij} p - \Lambda M_p^2 a^2 \delta_{ij} \right) 
\end{align*}   
And defining:
\begin{align*}
    \rho_{\mathrm{tot} } &= \rho + \Lambda M_p^2 \\ 
    p_{\mathrm{tot} } &= p - \Lambda M_p^2
\end{align*}
and so:
\begin{align*}
    \frac{p_{\Lambda }}{\rho_{\Lambda}}  = -1
\end{align*}
So the cosmological constant has the same effect of a \textit{vacuum energy} (that we saw in the previous paragraph). The idea is that, if $\Lambda$ is very small, it will not be measurable inside the solar system, but it will have a significant effect on the evolution of the entire universe.   
\end{document}
