%&latex
%
\documentclass[../template.tex]{subfiles}
\begin{document}

\section{Special Relativity}
\lesson{1}{3/10/2019}

\begin{dfn}
An \textbf{inertial frame} is a set of coordinates for which the Newton's law of motion are valid, that is a \textit{free} body moves with costant velocity $\vec{v}$.
\end{dfn}

The existence of inertial frames is not obvious. For example, Earth is not really an inertial frame: it rotates and revolves around the Sun. The Sun revolves also around the galaxy, and the galaxy is currently accelerating towards an another galaxy.\\
For all practical purposes, however, approximately inertial frames can be used. In fact, the first Newton law of motion simply postulates the existence of inertial frames.\\

Given an inertial frame $O$, another frame of reference $O'$ is inertial if it moves with constant velocity with respect to $O$.\\
It is possible to convert the set of coordinates of $O$ to that of $O'$. This is done with a \textbf{Lorentz boost}.\\
Suppose that $O'$ and $O$ share their axes, and $O'$ moves along $\hat{x}$ with velocity $v$. The coordinate transform is then:
\begin{align*}
t' &= \frac{t-\frac{vx}{c^2}}{\sqrt{1-\frac{v^2}{c^2}}}\\
x' &= \frac{x-vt}{\sqrt{1-\frac{v^2}{c^2}}}\\
y' &= y\\
z' &= z
\end{align*} 
where $c$ is the speed of light in vacuum.\\

The usual Galilean Transformations are obtained in the limit $v\ll c$. This can be obtained by formally taking $c\to \infty$, leading to:
\begin{align*}
t' &= t\\
x' &= x-vt\\
y' &= y\\
z' &= z
\end{align*}
They do feel as \q{making more sense} because they are connected to how human's perceive the world, living in a specific environment where relativistic effects are too small to be seen.\\
In the Galilean world, time and space are not mixed at all. In the relativistic world, time and space are still separate things - but they do change together, and so a bigger framework is needed to incorporate both of them. This is the notion of \textbf{spacetime}.\\

Let's consider now $2$ events $A$ and $B$. In fact, because now time and space are correlated, one cannot speak simply about positions, but needs to consider also the time information.\\
Let $O$ and $O'$ be two inertial frames. $O$ sees $A$ at $(t_A, \vec{x}_A)$, and $B$ at $(t_B, \vec{x}_B)$, while $O'$ sees $A$ at $(t_A', \vec{x}_A')$ and $B$ at $(t_B', \vec{x}_B')$.\\

For $O$ the space separation between the two events is $|\vec{x}_B - \vec{x}_A| = |\Delta \vec{x}|$, and the time separation is $t_B-t_A =\Delta t$.\\
On the other hand, for $O'$ the same quantities are $|\vec{x}_B'-\vec{x}_A'|=\Delta\vec{x}'$ and $t_B'-t_A'=\Delta t'$.\\

Both observers can compute the following quantities:
\begin{align*}
\Delta s^2 &= -c^2 \Delta t^2 + |\Delta \vec{x}|^2\\
\Delta {s'}^2 &= -c^2\Delta t^2 + |\Delta \vec{x}'|^2
\end{align*}
We note that Lorentz Boosts \textit{preserve} this peculiar quantity, that is called the \textbf{invariant four-distance}.
\begin{align*}
\Delta s^2 = \Delta {s'}^2
\end{align*}
(Prove it as homework)\\
This is a \textit{four}-distance, because it is a distance over a $4$-dimensional spacetime, and also \textbf{invariant} because it is preserved in coordinate transforms, and it's a quantity that every observer can agree on.

\begin{itemize}
\item If two events are connected by a ray of light, then $\Delta s^2 = 0$. This is called \textbf{null-separation}.
\item If two events are connected by something slower than light, then $\Delta s^2 < 0$. This is called \textbf{time-like separation}.
\item If $\Delta s^2 > 0$ then the two events have \textbf{space-like} separation.
\end{itemize}

In a \textbf{spacetime diagram}, a $xy$ plane is plotted at different times, over the $ct$ axis. Rays of light are diagonal lines ($45^\circ$ degrees of slope). A particle situated at the origin at time $0$ (defining the $E$ event) can then move only inside a \textbf{light cone}, and only events inside the light cone can influence or be influenced by the event $E$.


\subsection{Proper Time \& Time Dilation}
Consider two inertial frames $O$ and $O'$, with $O'$ moving at velocity $v$ along the shared $x$ axis. Let $A$ and $B$ be the events in which $O'$ starts/stop his stopwatch.\\
From the point of view of $O'$, $A$ and $B$ occur at the same location, as $O'$ is stationary wrt this frame of reference. However, from $O$ point of view, $O'$ is moving, and then $A$ and $B$ do not happen anymore at the same place.\\

We define the \textbf{proper time} as the \textit{time separation} between $2$ events that occur at the \textbf{same location} for an observer.\\
Note that:
\begin{align*}
\Delta s^2 &= -c^2 \Delta t^2 + \Delta x^2 \quad \text{for $O$}\\
\Delta {s'}^2 &= -c^2 \Delta \tau^2 +\cancel{\Delta {x'}^2} \>\>\text{for $O'$}
\end{align*}
As $\Delta x' = 0$, $\Delta \tau$ is a \textit{proper time}.\\
As the four-distance is invariant:
\begin{align*}
-c^2 \Delta t^2 + \Delta x^2 = -c^2 \Delta \tau^2 \Rightarrow \Delta t^2 - \frac{\Delta x^2}{c^2} = \Delta \tau^2
\end{align*}
and so $\Delta \tau < \Delta t$. So, the time measured in a moving reference trame is longer, i.e. it is \textit{dilated}.\\

Let $\Delta x$ be the distance covered by $O'$ in the time $\Delta t$ as measured by $O$. As $O'$ moves at a constant velocity $v$, it holds $\Delta x = v\Delta t$. Substituting in the previous relation:
\begin{align*}
\Delta t^2 -  \frac{v^2 \Delta t^2}{c^2} = \Delta \tau^2 \Rightarrow \Delta t^2 \left(1 - \frac{v^2}{c^2}\right) = \Delta \tau^2 \Rightarrow \underbrace{\Delta t}_{\text{Dilated time (from $O$)}} = \frac{\overbrace{\Delta \tau}^{\text{Proper time}}}{\sqrt{1-\frac{v^2}{c^2}}} \equiv \gamma \Delta\tau
\end{align*}
where:
\begin{align*}
\gamma(v) \equiv \frac{1}{\sqrt{1-\frac{v^2}{c^2}}}
\end{align*}
Note that $\gamma(v) \geq 1$, and $\gamma(v) \to +\infty$ as $v\to c$. So, the proper time is always the smaller one.

\subsection{Lorentz Boost inversion}
Recall the Lorentz Boost relations:
\begin{align*}
t' &= \frac{t-\frac{v x}{c^2}}{\sqrt{1-\frac{v^2}{c^2}}}\\
x' &= \frac{x-vt}{\sqrt{1-\frac{v^2}{c^2}}}\\
y' &= y\\
z' &= z
\end{align*}
If instead we know the coordinates of $O'$ and want to know the ones of $O$, we must invert them:
\begin{align*}
t &= \frac{t'+\frac{vx}{c^2}}{\sqrt{1-\frac{v^2}{c^2}}}\\
x &= \frac{x'+vt'}{\sqrt{1-\frac{v^2}{c^2}}}\\
y &= y'\\
z &= z'
\end{align*}
There are two ways to prove this:
\begin{enumerate}
\item Solve them explicitly
\item Consider that if $O'$ moves at velocity $\vec{v}$ with respect to $O$, then $O$ moves at velocity $-\vec{v}$ wrt $O'$.
\end{enumerate}
So there is not an inertial frame \q{prettier} than another - there is not one at rest, or one in motion \q{in absolute sense}.\\

Consider:
\begin{align*}
t' = \frac{t-\frac{v x}{c^2}}{\sqrt{1-\frac{v^2}{c^2}}}
\end{align*}
Multiplying by $c$:
\begin{align*}
ct' = \frac{ct-\frac{xv}{c}}{\sqrt{1-\frac{v^2}{c^2}}}
\end{align*}
This can be written in a more compact form:
\begin{align*}
ct' = \gamma ct - \beta\gamma x; \quad \gamma = \frac{1}{\sqrt{1-\frac{v^2}{c^2}}}; \quad \beta = \frac{v}{c}
\end{align*}
Note that the difference between the squares of the two coefficients is equal to $1$:
\begin{align*}
\gamma^2 - \beta^2\gamma^2 = \gamma^2-\beta^2\gamma^2 = \gamma^2(1-\beta^2) = \frac{1}{1-\beta^2}(1-\beta^2) = 1
\end{align*}
This relation is really similar to $\sinh^2-\cosh^2 = 1$. In fact, there exists an angle $\theta$ for which the first coefficient is equal to $\cosh \theta$, and the second to $\sinh \theta$. Then:
\begin{align*}
\tanh \theta = \frac{\sinh\theta}{\cosh\theta} = \frac{\beta\gamma}{\gamma} = \beta = \frac{v}{c}
\end{align*}
In this notation, the Lorentz Boost becomes:
\begin{align}
\label{eqn:boost-tanh}
\begin{cases}
ct' = \cosh\theta ct -\sinh\theta x\\
x' = \cosh\theta x - \sinh\theta ct
\end{cases}
\end{align}
In matrix notation:
\begin{align*}
\begin{pmatrix}ct'\\x'\end{pmatrix} =
\begin{pmatrix}\cosh\theta & -\sinh\theta\\
-\sinh\theta & \cosh\theta\end{pmatrix}\begin{pmatrix}
ct\\ x\end{pmatrix}
\end{align*}
which is really similar to a 2D rotation:
\begin{align*}
\begin{pmatrix}x'\\ y'\end{pmatrix} =
\begin{pmatrix}\cos\theta & \sin\theta\\
-\sin\theta & \cos\theta\end{pmatrix} \begin{pmatrix}x\\ y\end{pmatrix}
\end{align*}
So, in a certain sense, a Lorentz Boost is a kind of rotation in an abstract space.\\

Let's now see how to draw axes in a spacetime diagram. Starting by a graph with axes $x$ and $ct$, of observer $O$, we want to draw the axes $x'$ and $ct'$ of observer $O'$.\\
Noting that the $ct'$ axis is the locus of points with $x'=0$, we can solve:
\begin{align*}
0 = -\sinh \theta ct +\cosh\theta x \Rightarrow ct \sinh\theta = x\cosh\theta \Rightarrow ct =\frac{x}{\tanh\theta}
\end{align*}
This is a line, passing through the origin, with slope \textit{greater} than $1$ (as the denominator $\tanh\theta =v/c$ is always $\leq 1$).\\

On the other hand, the $x'$ axis is made of points with $ct'=0$. Using again (\ref{eqn:boost-tanh}) we can solve:
\begin{align*}
\cosh\theta ct - \sinh\theta x = 0 \Rightarrow ct\cosh\theta = x\sinh\theta \Rightarrow ct = x\tanh\theta
\end{align*}
This is a line, passing through the origin, with slope \textit{lower} than $1$.\\
In summary:
\begin{itemize}
\item $ct'$ axis $\displaystyle\equiv x'=0 \equiv ct =\frac{x}{\tanh\theta}$
\item $x'$ axis $\equiv \displaystyle ct'=0 \equiv ct=x\tanh\theta$
\end{itemize}

\end{document}


