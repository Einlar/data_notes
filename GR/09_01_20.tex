%&latex
%
\documentclass[../template.tex]{subfiles}
\begin{document}

\section{Gravitational Waves - part 2}
\lesson{?}{09/01/20}

Recall that we used a \textit{perturbed} metric:
\begin{align*}
    g_{\mu \nu} = \eta_{\mu \nu} + h_{\mu \nu}(x) \qquad h_{\mu \nu} \ll 1
\end{align*}
We are looking for a \textit{vacuum solution}, $R_{\mu \nu} = 0$, and so we are dealing only with the \textit{propagation} of a gravitational wave, and not its production. Expanding $R_{\mu \nu}$ to $O(h)$:
\begin{align*}
    R_{\mu \nu} = - \frac{1}{2} \square h_{\mu \nu} + \frac{1}{2} \partial_\mu (\partial_\lambda h^\lambda_\nu - \frac{1}{2} \partial_\nu h ) + (\mu \leftrightarrow \nu)  
\end{align*}   
Making a \textit{infinitesimal} change of coordinates: 
\begin{align*}
    x^\mu \to x^\mu + \epsilon^\mu(x)
\end{align*}
we have:
\begin{align*}
    h_{\mu \nu} \to \tilde{h}_{\mu \nu} = h_{\mu \nu} - \partial_\mu \epsilon_\nu - \partial_\nu \epsilon_\mu
\end{align*}
$h_{\mu \nu}$ and $\tilde{h}_{\mu \nu}$ are both describing the same wave, but with different expressions. To make them equal, we need to fix the gauge. Usually, this is done in such a way to simplify the equation. For example, consider the \textit{harmonic gauge}:
\begin{align*}
    \partial_\lambda h^\lambda_\nu - \frac{1}{2} \partial_\nu h =0 
\end{align*} 
leading to:
\begin{align*}
    \square h_{\mu \nu} =0
\end{align*}
One solution is the plane wave:
\begin{align*}
    h_{\mu \nu} = C_{\mu \nu} e^{i \bm{k} \cdot \bm{x}}
\end{align*}
with $\bm{k}\cdot \bm{k} = 0$, $k^\mu$ being the $4$-wave vector, and $P^\mu = \hbar k^\mu \Rightarrow P^2 = 0 \equiv m^2$. Note that gravitational waves propagate along \textit{null geodesics} (as does light).

The harmonic gauge \textit{does not completely fix} the gauge, as there is still some freedom left. In fact, suppose that $h_{\mu \nu}$ satisfies the harmonic gauge, and $\epsilon_\mu$ satisfies $\square \epsilon_\mu = 0$, then:
\begin{align*}
    \tilde{h}_{\mu \nu} = h_{\mu \nu} - \partial_\mu \epsilon_\nu - \partial_\nu \epsilon_\mu
\end{align*} 
also satisfies the harmonic gauge.

Let's prove it.
\begin{align*}
    0 \overset{?}{=} \partial_\lambda \tilde{h}^\lambda_\mu - \frac{1}{2} \partial_\mu \tilde{h} = \partial_\lambda [\cancel{h^\lambda_\mu }- \hlc{Yellow}{\partial^\lambda \epsilon_\mu }- \partial_\mu \epsilon^\lambda] - \frac{1}{2} \partial_\mu [\cancel{h} - 2 \partial \cdot \epsilon]  
\end{align*}
with $h = h^\mu_\mu$, $\partial = \partial^\mu_\mu$, and where we applied the harmonic gauge condition to cancel two terms. 
Then:
\begin{align*}
    = -\hlc{Yellow}{ \square \epsilon_\mu} - \partial_\mu \partial \cdot \epsilon + \partial_\mu \partial \cdot \epsilon
\end{align*}

We want not to impose $4$ additional conditions that remove the residual freedom $x^\mu \to x^\mu + \epsilon^\mu$ with $\square \epsilon^\mu = 0$

\begin{align*}
    \square h_\mu \nu = 0 & \to h_{\mu \nu} = C_{\mu \nu} e^{i \bm{k}\cdot \bm{x}} \qquad k^2 = 0, \quad C_{\mu \nu} = \text{Const.}\\
    \square \epsilon_{\mu \nu} = 0 & \to \epsilon_\mu = \gamma_\mu e^{i \bm{k}\cdot \bm{x}}, \qquad k^2 = 0 \quad \gamma_\mu = \text{const}
\end{align*}
Applying the coordinate change to the plane wave $h_{\mu \nu} \to \tilde{h}_{\mu \nu} - \partial_\mu \epsilon_\nu - \partial_\nu \epsilon_\mu$ we get:
\begin{align*}
    \tilde{C}_{\mu \nu} e^{i \bm{k}\cdot \bm{x}} = C_{\mu \nu} e^{ikx} - \partial_\mu [\gamma_\nu e^{ikx}] - \partial_\nu [\gamma_\mu e^{ikx}]
\end{align*}
Computing the derivatives:
\begin{align*}
    \tilde{C}_{\mu \nu} e^{ikx} = C_{\mu \nu} e^{ikx} - ik_\mu \gamma_\nu e^{ikx} - ik_\nu \gamma_\mu e^{ikx}  
\end{align*}
Removing the phases $e^{ikx}$ we have found that, under a change of coordinates $x^\mu \to x^\mu + \epsilon^\mu$ the constants transform as:
\begin{align*}
    C_{\mu \nu} \to \tilde{C}_{\mu \nu} - ik_\mu \gamma_\nu - ik_\nu \gamma_\mu
\end{align*}
So the plane wave remains a plane wave:
\begin{align*}
    h_{\mu \nu} = C_{\mu \nu} e^{ikx} \to \tilde{h}_{\mu \nu} = \tilde{C}_{\mu \nu} e^{ikx}
\end{align*}
but the coefficients are now different. However, $C_{\mu \nu}$ and $\tilde{C}_{\mu \nu}$ both describe the same physical situations, as $x^\mu \to x^\mu + \epsilon^\mu$ is just a change of coordinates. 

This is because the gauge is not completely fixed - we need additional conditions. As before, we choose them so that the problem become simpler:
\begin{align*}
    \tilde{C}_{00} = \tilde{C}_{0i} = 0
\end{align*}
These fix completely the residual gauge freedom.

\medskip

We need to verify that we \textit{can} impose these conditions. 
\textbf{Goal}: show that, starting from a generic $C_{\mu \nu}$, we can always find $\gamma_\mu$ (i.e. find $\epsilon_\mu$, that is \textit{an appropriate change of variables}) such that $\tilde{C}_{00} = \tilde{C}_{0i} = 0$ $(i=1,2,3)$.
We start from the transformation rule we found earlier:
\begin{align*}
    C_{\mu \nu} \to \tilde{C}_{\mu \nu} - ik_\mu \gamma_\nu - ik_\nu \gamma_\mu
\end{align*}
Setting $\mu = \nu = 0$:
\begin{align*}
    \tilde{C}_{00} = C_{00} - 2i k_0 \gamma_0 \Rightarrow \gamma_0 = \frac{C_{00} }{2 i k_0} \Rightarrow \tilde{C}_{00} = 0 
\end{align*}
Note that there is \textit{only one possible choice} for the change of coordinates, i.e. only one value for $\gamma_0$.

If $\mu = 0$, $\nu=i$:
\begin{align*}
    \tilde{C}_{0i} &= C_{0i} - i k_0 \gamma_i - ik_i \gamma_0 = C_{0i} - ik_0 \gamma_i - \cancel{i} k_i \frac{C_{00}}{2\cancel{i} k_0} \overset{!}{=}  0
\end{align*}
leading to:
\begin{align*}
    i k_0 \gamma_i = C_{0i} - \frac{k_i}{2 k_0} C_{00} \Rightarrow \gamma_i = \frac{1}{i k_0} \left(C_{0i} - \frac{k_i}{2 k_0} C_{00}  \right)  
\end{align*}
Note $\gamma_\mu$ is completely fixed by the need to get $\tilde{C}_{00} = 0$, $\tilde{C}_{0i} = 0$.
Here we assume $k_\mu \neq 0$. As we will show, this amount on assuming that the gravitational wave has a non-zero frequency. The degenerate case of a $0$ frequency wave - which means to \q{have nothing} - generates diverging terms in this gauge. This is \textit{a gauge artefact}, that can be fixed by just making a different gauge choice. 

\medskip

Summarizing, the gravitational wave equation in the harmonic gauge with the additional constraints to fix completely the gauge is:
\begin{align*}
    \begin{cases}
        \square h_{\mu \nu} = 0\\
        \partial_\lambda  h^\lambda_\mu - \frac{1}{2} \partial_\mu h = 0\\
        h_{00} = h_{0i} = 0 
    \end{cases}
\end{align*}
$h_{\mu \nu}$ has $10$ degrees of freedom (because of symmetry), these ones:
\begin{align*}
    h_{\mu \nu} = \left(\begin{array}{cccc}
    \cdot & \cdot & \cdot & \cdot \\ 
     & \cdot & \cdot & \cdot \\ 
     &  & \cdot & \cdot \\ 
     &  &  & \cdot
    \end{array}\right)
\end{align*}
The harmonic gauge is a set of $4$ differential equations ($4$ constraints), and then we have another $4$ constraints from the residual gauge fixing. So a gravitational wave will have $10-4-4=2$ degrees of freedom (physically, they are $2$ polarizations).

\medskip

Let's see this concretely (assuring that all $8$ constraints are independent) for the plane-wave solution:
\begin{align*}
    h_{\mu \nu} = C_{\mu \nu} e^{i\bm{k}\cdot \bm{x}}
\end{align*}
We start from the harmonic gauge (H.G.) condition for $\mu=0$:
\begin{align*}
    \partial_\lambda h^\lambda_0 - \frac{1}{2} \partial_0 h = 0 
\end{align*}
Recall that $h^\alpha_\beta= \eta^{\alpha \gamma} h_{\gamma \beta}$, and so:
\begin{align*}
    \partial_0 h^0_{0} + \partial_i h^i_0 - \frac{1}{2} \partial_0 h = 0 
\end{align*} 
As $\eta^{00} = -1$ and $\eta^{ij} = \delta_{ij}$ we have, because $h_{00} = h_{0i} = 0$:
\begin{align*}
    -\cancel{\partial_0 h_{00}} +\bcancel{ \partial_i h_{i0}} - \frac{1}{2} \partial_0 h = 0 \Rightarrow \partial_0 h = 0 \Rightarrow \partial_0 (C e^{ikx}) = 0 \Rightarrow ik_0 C e^{ikx} = 0
\end{align*}
This means that:
\begin{align*}
    C \equiv C^\mu_\mu = 0 = C^0_0 + C^i_i =-C_{00} + C_{i i} = 0
\end{align*}
So: $h=0$ and $h_{i i} = 0$. This means that the gravitational wave is \textit{traceless} (this can be proved in the general case of a \textit{non-planar} wave).

\medskip

Now, if $\mu = i$ the harmonic-gauge condition, as $h=0$, reduces to:
\begin{align*}
    \partial_\lambda h^\lambda_i = 0 \Rightarrow -\partial_0 h_{0j} + \partial_i h_{ij} = 0 \Rightarrow \partial_i h_{ij} = 0
\end{align*}
This means, as we will see, that the G.W. is \textit{transverse} (oscillates in the direction perpendicular to its propagation).

\medskip

So, we are left to solve:
\begin{align*}
    \begin{cases}
        \square h_{\mu \nu} = 0 \Rightarrow h_{\mu \nu} = C_{\mu \nu} e^{i\bm{k}\cdot \bm{x}} \> \bm{k}^2 = 0\\
        h_{00} = h_{0i} = 0\\
        h_{i i} =  0\\
        \partial_i h_{ij} = 0
    \end{cases}
\end{align*}
Applying these conditions, we get new expressions for the matrix $C_{\mu \nu}$:
\begin{align*}
    \begin{dcases}
        C_{00} = C_{0i} = 0\\
        C_{i i} = 0\\
        i k_i C_{ij} e^{i\bm{k} \cdot \bm{x}} = 0 \Rightarrow k_i C_{ij} = 0
    \end{dcases}
\end{align*}
Let's fix $\bm{k}$ along the $\hat{z}$ axis: $\bm{k} = (0,0,k)$. Then the third condition becomes:
\begin{align*}
    C_{3j} = 0 \Rightarrow C_{31} = C_{32} = C_{33} = 0
\end{align*}
As $C_{\mu \nu}$ is symmetric, we can rewrite this as:
\begin{align*}
    C_{13} = C_{23} = C_{33} = 0
\end{align*}
As $C_{i i} = 0$:
\begin{align*}
    C_{11} + C_{22} + \cancel{C_{33}} = 0
\end{align*}
Summarizing:
\begin{align*}
    \begin{cases}
        C_{00} = C_{01} = C_{02} = C_{03} = 0\\
        C_{11} + C_{22} = 0\\
        C_{13} = C_{23} = C_{33} = 0
    \end{cases}
\end{align*}
We have two independent solutions:
\begin{align*}
    C_{12} = C_{21} \neq 0 \lor C_{22} = -C_{11} \neq 0
\end{align*}
As the wave equation is linear, also any linear combination of these solutions will be a solution. In particular, it will be a matrix $C_{\mu \nu}$ that is a linear combination of the following two basis elements:
\begin{align*}
    e_{ij,+} \equiv \left(\begin{array}{ccc}
    1 & 0 & 0 \\ 
    0 & -1 & 0 \\ 
    0 & 0 & 0
    \end{array}\right) \qquad e_{ij,\times} \equiv \left(\begin{array}{ccc}
    0 & 1 & 0 \\ 
    1 & 0 & 0 \\ 
    0 & 0 & 0
    \end{array}\right)
\end{align*}
We need to normalize them:
\begin{align*}
    e_{ij,+} e_{ij,+} &= e_{ij,+} e_{ji,+} = \operatorname{trace}(e_+ e_+) = \operatorname{Tr} \left[\left(\begin{array}{ccc}
    1 & 0 & 0 \\ 
    0 & -1 & 0 \\ 
    0 & 0 & 0
    \end{array}\right) \left(\begin{array}{ccc}
    1 & 0 & 0 \\ 
    0 & -1 & 0 \\ 
    0 & 0 & 0
    \end{array}\right)\right]   =\\
    &= \operatorname{Tr} \left(\begin{array}{ccc}
    1 & 0 & 0 \\ 
    0 & 1 & 0 \\ 
    0 & 0 & 0
    \end{array}\right) = 2\\ 
    e_{ij,+} e_{ij,\times} &= \operatorname{Tr} (e_+ e_\times) = \operatorname{Tr}\left[\left(\begin{array}{ccc}
    1 & 0 & 0 \\ 
    0 & -1 & 0 \\ 
    0 & 0 & 0
    \end{array}\right) \left(\begin{array}{ccc}
    0 & 1 & 0 \\ 
    1 & 0 & 0 \\ 
    0 & 0 & 0
    \end{array}\right)\right]  = \operatorname{Tr} \left(\begin{array}{ccc}
    0 & 1 & 0 \\ 
    -1 & 0 & 0 \\ 
    0 & 0 & 0
    \end{array}\right)  = 0\\
    e_{ij,\times}e_{ij,\times} &= \operatorname{Tr} (e_\times e_\times) = \operatorname{Tr}\left[\left(\begin{array}{ccc}
    0 & 1 & 0 \\ 
    1 & 0 & 0 \\ 
    0 & 0 & 0
    \end{array}\right) \left(\begin{array}{ccc}
    0 & 1 & 0 \\ 
    1 & 0 & 0 \\ 
    0 & 0 & 0
    \end{array}\right)\right] = \operatorname{Tr}\left(\begin{array}{ccc}
    1 & 0 & 0 \\ 
    0 & 1 & 0 \\ 
    0 & 0 & 0
    \end{array}\right)   = 2
\end{align*}
And so:
\begin{align*}
    \operatorname{Tr}(e_r e_s) = 2 \delta_{rs} \qquad \{r,s\} \in \{+,x\} 
\end{align*}
A G.W. can be decomposed as:
\begin{align*}
    h_{ij} (x) = (h_+ e_{ij,+} + h_\times e_{ij,\times}) e^{i\bm{k} \cdot \bm{x}} \quad \bm{k}^2 = 0
\end{align*}
where $h_+$ and $h_x$ are the \textit{amplitudes} of the two polarizations. 

\medskip
Let's now try to get a better understanding of this solution. First of all, note that the metric is real $g_{\mu \nu} = \eta_{\mu \nu} + h_{\mu \nu}(x)$, but the solution we have is complex. However, as the differential equation has real coefficients, then if $h_{\mu \nu}$ solves it, also its complex conjugate $h^*_{\mu \nu}$ solves it. Then also their linear combinations will be solutions, and so we can combine them to write a solution that is manifestly real:
\begin{align*}
    C_{ij} e^{ikx} = C_{ij} [\cos(kx) + i\sin(kx)] \> \text{is a solution}\\
    C_{ij} e^{-ikx} = C_{ij} [\cos(kx) - i \sin(kx)] \> \text{is a solution}\\
    \Rightarrow C_{ij}[C_1 \cos(kx) + C_2 \sin(kx)]\> \text{is a solution}
\end{align*}
Absorbing the constants:
\begin{align*}
    C_{ij} \cos(kx + \varphi) \> \text{is a solution}
\end{align*}
where $\varphi$ is an arbitrary phase.

So, we can rewrite a generic plane wave solution as:
\begin{align*}
    h_{ij}(z) = \sum_{r = +,\times} h_r e_{ij,r} \cos(kt - kz + \varphi_r)
\end{align*}
where we used $\bm{k} \cdot \bm{x} + \varphi = -kt + kz + \varphi$ and changed the overall sign and renamed $-\varphi \to \varphi$. Note that it is periodic in time with period $T= 2\pi/k$. We then define the frequency as:
\begin{align*}
    f = \frac{1}{T} = \frac{k}{2 \pi}  
\end{align*}

Let's now focus on getting a geometric interpretation for the polarizations.
\begin{align*}
    e_{ij,+} \equiv \left(\begin{array}{ccc}
    1 & 0 & 0 \\ 
    0 & -1 & 0 \\ 
    0 & 0 & 0
    \end{array}\right)
\end{align*}
Recall that $g_{\mu \nu} = \eta_{\mu \nu} + h_{\mu \nu}$ tells us how to measure \textit{distances}. Suppose we have two objects with a fixed comoving distance:
\begin{align*}
    \int \dd{x} \sqrt{g_{x x}}
\end{align*}
In presence of a gravitational wave, the non-zero $h_{\mu \nu}$ will \textit{vary} their distance. 
For the $+$ polarization, at $t=0$, distances along $\hat{x}$ \textit{rise} and along $\hat{y}$ lower. These two directions exchange after half a period (as the wave will have travelled $\lambda/2$). Note that changes happen only in directions $\perp$ to that of motion (transverse wave).  %See fig (1)

Now, consider a $45^\circ$ rotation:a
\begin{align*}
    \left(\begin{array}{c}
    \tilde{x} \\ 
    \tilde{y} \\ 
    \tilde{z}
    \end{array}\right) = \left(\begin{array}{ccc}
    \cos(45^\circ) & \sin(45^\circ) & 0 \\ 
    -\sin(45^\circ) & \cos(45^\circ) & 0 \\ 
    0 & 0 & 1
    \end{array}\right) \left(\begin{array}{c}
    x \\ 
    y \\ 
    z
    \end{array}\right)
\end{align*}
\begin{align*}
    \left(\begin{array}{c}
        \tilde{x} \\ 
        \tilde{y} \\ 
        \tilde{z}
        \end{array}\right) = \left(\begin{array}{ccc}
        1/\sqrt{2} & 1/\sqrt{2} & 0 \\ 
        -1/\sqrt{2} & 1/\sqrt{2} & 0 \\ 
        0 & 0 & 1
        \end{array}\right) \left(\begin{array}{c}
        x \\ 
        y \\ 
        z
        \end{array}\right)
\end{align*}
For a vector along the diagonal, such as $(1,1,0)$, we get:
\begin{align*}
    \left(\begin{array}{ccc}
        1/\sqrt{2} & 1/\sqrt{2} & 0 \\ 
        -1/\sqrt{2} & 1/\sqrt{2} & 0 \\ 
        0 & 0 & 1
        \end{array}\right) \left(\begin{array}{c}
        1 \\ 
        1 \\ 
        0
        \end{array}\right) = \left(\begin{array}{c}
        \sqrt{2} \\ 
        0 \\ 
        0
        \end{array}\right)
\end{align*}
We can write this relation as $\tilde{x}^i = R^i_j x^j$, or more compactly as $\bm{\tilde{x}} = R \bm{x}$. Recall that $R^T R = \mathbb{I}$ for rotation matrices (as they are orthogonal). The line element does not change:
\begin{align*}
    \dd{l^2} = \dd{x^i} g_{ij} \dd{x^j} = \dd{\tilde{x}^i} \tilde{g}_{ij} \dd{\tilde{x}^j} = \dd{x^T} g \dd{x} = \dd{\tilde{x}^T} \tilde{g} \dd{x} = \dd{x^T} R^T \tilde{g} R \dd{x}
\end{align*}
as $\dd{\tilde{x}} = R \dd{x}$. This means that $R^T \tilde{g} R = g \Rightarrow \tilde{g} = R g R^T$, which is the formal way for writing:
\begin{align*}
    \tilde{g}_{ij} = \pdv{x^m}{\tilde{x}^i} g_{mn} \pdv{x^n}{\tilde{x}^j}
\end{align*}

If we focus on the space coordinates $g_{ij} = \delta_{ij} + h_{ij}$:
\begin{align*}
    \mathbb{I} + \tilde{h} = R(\mathbb{I} + h) R^T = R \mathbb{I}R^T + R h R^T = \mathbb{I} + R h R^T 
\end{align*}
and so: $\tilde{h} = R h R^T$. We can now tackle the $\times$ polarization. 
\begin{align*}
    h = e_\times = \left(\begin{array}{ccc}
    0 & 1 & 0 \\ 
    1 & 0 & 0 \\ 
    0 & 0 & 0
    \end{array}\right)
\end{align*}
Then:
\begin{align*}
    \tilde{h} = \left(\begin{array}{ccc}
        1/\sqrt{2} & 1/\sqrt{2} & 0 \\ 
        -1/\sqrt{2} & 1/\sqrt{2} & 0 \\ 
        0 & 0 & 1
        \end{array}\right) \left(\begin{array}{ccc}
        0 & 1 & 0 \\ 
        1 & 0 & 0 \\ 
        0 & 0 & 0
        \end{array}\right) \left(\begin{array}{ccc}
            1/\sqrt{2} & -1/\sqrt{2} & 0 \\ 
            1/\sqrt{2} & 1/\sqrt{2} & 0 \\ 
            0 & 0 & 1
            \end{array}\right) = \left(\begin{array}{ccc}
            1 & 0 & 0 \\ 
            0 & -1 & 0 \\ 
            0 & 0 & 0
            \end{array}\right) = e_+
\end{align*}
So $e_+$ is just $e_\times$ rotated by $45^\circ$.

A generic case will have the effect of $e_+$ \textit{superimposed} to that of $e_\times$, with possibly different amplitudes:
\begin{align*}
    h_{ij} = \sum_{r=+,\times} h_r e_{ij,r} \cos(kt - kz + \varphi_r)
\end{align*} 

\end{document}
