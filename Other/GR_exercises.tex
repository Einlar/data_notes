%&latex
%
\documentclass[../template.tex]{subfiles}
\begin{document}

\chapter{Homework - Week 1}
\lesson{1}{09/10/19}

\section{Problem 1}
\textit{Show that the inverse of a Lorentz boost amounts in changing the sign of $x$ in the expressions for the boost.}\\
Starting from:
\begin{align}\label{eqn:time}
    ct' &=  \gamma ct- \beta \gamma x \\
    x' &= -\beta \gamma ct + \gamma x \label{eqn:x}
\end{align} 
We solve for $x$ in the second equation:
\begin{align*}
    x = \frac{1}{\gamma} (x' + \beta \gamma c t) 
\end{align*}
and substitute in the first:
\begin{align*}
    ct' &= \gamma c t - \beta \gamma\frac{1}{\gamma} (x' + \beta \gamma c t) = \\
    &=  (\gamma - \beta^2 \gamma ) ct - \beta x' 
\end{align*}
and then solve for $t$:
\begin{align*}
    ct = (ct' + \beta x') \frac{1}{\gamma (1-\beta^2)} 
\end{align*} 
Recall that:
\begin{align*}
    \gamma = \frac{1}{\sqrt{1-\beta^2}} \Rightarrow (1-\beta^2) = \frac{1}{\gamma^2} 
\end{align*}
leading to:
\begin{align*}
    ct = \gamma(ct' + \beta x')
\end{align*}
which is the same as (\ref{eqn:time}) except for $t \leftrightarrow t'$ and $x \leftrightarrow -x'$.\\
Then, simply substituting in the equation fo $x$ we finally get:
\begin{align*}
    x &= \frac{1}{\gamma}(x' + \beta \gamma^2 [ct' + \beta x']) = \\
    &=  \frac{1}{\gamma} \frac{x' - \beta^2 x' + \beta ct' + \beta^2 x'}{1-\beta^2} = \\
    &= \frac{\gamma^2}{\gamma} (x' + \beta ct') = \gamma (x' + \beta ct')  \\
\end{align*}  
which is again the same as (\ref{eqn:x}), except for $t \leftrightarrow t'$ and $x \leftrightarrow -x'$.\\

\textit{Using the explicit form of the Lorentz boosts, show that $ds^2 \equiv (c dt)^2 - dx^2 - dy^2 -dz^2$ is a scalar quantity (namely, it is invariant under Lorentz transformations).}\\
We start from:
\begin{align*}
    ds^2 = (c\,dt)^2 - dx^2 - dy^2 - dz^2
\end{align*}
and simply apply the (inverse) Lorentz transformations:
\begin{align*}
    ct &=  \gamma ct' + \beta \gamma x'  \\
    x &= \beta \gamma ct' + \gamma x'  \\
\end{align*}
to arrive at the prime reference frame:
\begin{align*}
    (ds')^2 &= (\gamma c dt' + \beta \gamma dx')^2 - (\beta \gamma ct' + \gamma dx')^2 - (dy')^2 - (dz')^2 = \\
    &= \gamma^2 c^2 (dt')^2 + \beta^2 \gamma^2 (dx')^2 + \cancel{2\beta \gamma^2 c dt'\, dx' }- \beta^2 \gamma^2 c^2 (dt')^2 - \gamma^2 (dx')^2 \cancel{- 2\beta \gamma^2 c dt'\,dx}' - dy' - dz' =  \\
    &= \gamma^2 c^2 (dt')^2 [1-\beta^2] + \gamma^2 (dx')^2 [\beta^2 - 1] - (dy')^2 - (dz')^2 = \\
    &=  c^2 (dt')^2 - (dx')^2 - (dy')^2 - (dz')^2 \\
\end{align*}

\textit{Using the 4-vector notation, show that this statement is equivalent to requiring that $\Lambda^T \eta \Lambda = \eta$}.\\

Recall that:
\begin{align*}
    ds^2 = \eta_{\alpha \beta} dx^\alpha dx^\beta; \quad x'^\mu = \Lambda^\mu_{\diamond \nu} x^\nu
\end{align*}
Then, transforming to the prime frame of reference:
\begin{align*}
    (ds')^2 &= \eta_{\mu \nu} dx'^\mu dx'^\nu = \eta_{\mu \nu} \Lambda^\mu_{\diamond \alpha} dx^\alpha \Lambda^\nu_{\diamond \beta}dx^\beta \overset{!}{=} \eta_{\alpha \beta} dx^\alpha dx^\beta
\end{align*}
Thus:
\begin{align}
    \eta_{\mu \nu} \Lambda^\mu_{\diamond \alpha}\Lambda^\mu_{\diamond \beta} = \Lambda^\mu_{\diamond \alpha} \eta_{\mu \nu} \Lambda^\nu_{\diamond \beta} = \eta_{\alpha \beta}
    \label{eqn:lambda-not}
\end{align}
Recall that a matrix multiplication in Einstein notation is denoted by:
\begin{align*}
    C^i_{\diamond k} = A^i_{\diamond j} B^j_{\diamond k}
\end{align*}
and that $(C^i_{\diamond k})^T = C^k_{\diamond i}$.\\
Then (\ref{eqn:lambda-not}) is equivalent to $\Lambda^T \eta \Lambda = \eta$ in matrix notation.\\

\textit{Show (by explicit matrix multiplication) that indeed, $\Lambda^T \eta \Lambda = \eta $}\\

Recall that:
\begin{align*}
    \Lambda = \Lambda^T = \left(\begin{array}{cccc}
    \gamma & -\beta \gamma & 0 & 0 \\ 
    -\beta \gamma & \gamma & 0 & 0 \\ 
    0 & 0 & 1 & 0 \\ 
    0 & 0 & 0 & 1
    \end{array}\right); \qquad \eta = \left(\begin{array}{cccc}
    -1 & 0 & 0 & 0 \\ 
    0 & 1 & 0 & 0 \\ 
    0 & 0 & 1 & 0 \\ 
    0 & 0 & 0 & 1
    \end{array}\right)
\end{align*}
So $\Lambda^T \eta$ is merely $\Lambda$ with a sign change on the first column, and then:
\begin{align*}
    \Lambda^T \eta \Lambda = \left(\begin{array}{cccc}
    -\gamma & -\beta \gamma & 0 & 0 \\ 
    +\beta \gamma & \gamma & 0 & 0 \\ 
    0 & 0 & 1 & 0 \\ 
    0 & 0 & 0 & 1
    \end{array}\right)
    \left(\begin{array}{cccc}
    \gamma & -\beta \gamma & 0 & 0 \\ 
    -\beta \gamma & \gamma & 0 & 0 \\ 
    0 & 0 & 1 & 0 \\ 
    0 & 0 & 0 & 1
    \end{array}\right) = \left(\begin{array}{cccc}
    -1 & 0 & 0 & 0 \\ 
    0 & 1 & 0 & 0 \\ 
    0 & 0 & 1 & 0 \\ 
    0 & 0 & 0 & 1
    \end{array}\right)
\end{align*}  
(Recall that $\gamma = (1-\beta)^{-1}$, so that $\gamma^2 (\beta^2 -1) = -1$, and so on)


\section{Problem 2}
\begin{enumerate}
    \item With respect to an observer on Earth, muons travel through $\SI{15}{\kilo\m}$ of atmosphere at a velocity $v = 0.995c$. Thus they arrive at Earth after an interval $\Delta t$:
    \begin{align*}
        \Delta t = \frac{h}{v} \approx \SI{50.286}{\micro\s} 
    \end{align*}  
    Ignoring relativity effects, given their mean lifetime $\tau \approx \SI{2.2}{\micro\s}$, the survival probability after $\Delta t$ is:
    \begin{align*}
        p(\Delta t) = \exp\left(-\frac{\Delta t}{\tau} \right)\approx \num{1.2e-10} \sim 0
    \end{align*}
    So, if we ignore SR, it is very unlikely to observe muons at Earth's surface.
    \item Denote with $O$ the inertial frame of reference of an observer on Earth's surface, stationary wrt the atmosphere, and with $O'$ an observer stationary wrt the muons. If we set the common frame origin at the surface, and take the $x$-axis as vertical, directed towards the sky, then $O'$ is moving downward wrt to $O$, that is with a relative velocity of $v = -0.995c$.\\
    $O$ considers two events:
    \begin{itemize}
        \item Starting point of muons: $x_0 = \SI{15}{\kilo\m}$, at time $t_0 = 0$.
        \item Final point of muons: $x_1 = \SI{0}{\kilo\m}$, $t_1 = \Delta t$.
    \end{itemize}
    The mean lifetime of muons is measured in a reference frame where they are stationary, so we must compute the travel time wrt $O'$, that is $t_1' - t_0'$. Simply by using a Lorentz boost:
    \begin{align*}
        ct'_0 &= \gamma ct_0  - \beta \gamma x_0  \\
        ct'_1 &= \gamma ct_1 - \beta \gamma x_1 
    \end{align*} 
    with:
    \begin{align*}
        \beta = -0.995; \qquad \gamma = \sqrt{\frac{1}{1-\beta^2} } \approx \num{10.0125}
    \end{align*} 
    we get:
    \begin{align*}
        t_0' &\approx \frac{0.995 \cdot 10.0125 \cdot \SI{15}{\kilo\m}}{c}  \approx \SI{498.468}{\micro\s}\\
        t_1' &= \gamma \Delta t \approx \SI{503.49}{\micro\s}
    \end{align*}
    so that the \textbf{proper time} is  $\Delta \tau = t_1' - t_0' \approx \SI{5.022}{\micro\s}$, leading to a survival probability of:
    \begin{align*}
        p(\Delta \tau) = \exp\left(-\frac{\Delta \tau}{\tau} \right) \approx 10.2\%
    \end{align*}
    Notice that the same result can be obtained from the formula of \textbf{time dilation}:
    \begin{align*}
        \Delta t = \Delta \tau \gamma \Rightarrow \Delta \tau = \frac{\Delta t}{\gamma} 
    \end{align*} 
    Recall in fact that proper time is always the smallest one.
    \item We know examine the same problem from the point of view of the muons. Denote with $O$ the reference frame of muons, and with $O'$ that of Earth. As only $O'$ is at rest wrt the atmosphere, only $O'$ can measure directly its proper length - which is $\SI{15}{\kilo\m}$. Recall that length is a difference of distances measured at the same time wrt an observer.\\
    Suppose that $O$ wants to measure the atmosphere's length. Then it will compute the spatial distance of two simultaneous events: one located at the atmosphere's start ($x_0 = 0, t_0 = 0$), and one at the atmosphere's end ($x_1 = L, t_1 = 0$). Notice that $t_0 = t_1$. $L$ is not known at the moment, and can be computed if we use a Lorentz Boost to relate it to the known proper length:   
    \begin{align*}
        x' = -\beta \gamma c t + \gamma x
    \end{align*}
    We can now compute the difference:
    \begin{align*}
        \SI{15}{\kilo\m} = x'_1 - x'_0 = \cancel{- \beta \gamma c t_1 }+ \gamma x_1 + \cancel{\beta \gamma ct_0 }- \gamma x_0 = \gamma(x_1 - x_0)   
    \end{align*}
    as $t_0 = t_1$ for the length's definition. So, $O$ measures a length $L$ of:
    \begin{align*}
        L = \frac{L'}{\gamma} \approx \frac{\SI{15}{\kilo\m}}{10.0125} \approx \SI{1.498}{\kilo\m}  
    \end{align*}  
    And so the muons take only $\Delta \tau$ to cross $L'$:
    \begin{align*}
        \Delta \tau = \frac{L'}{v} \approx \frac{\SI{1.498}{\kilo\m}}{0.995 c} \approx \SI{5.022}{\micro\s}  
    \end{align*}  
    which is the same result computed at the previous point.
\end{enumerate}

\section{Problem 3}
\textit{A source emits radiation at an angle $\theta'$ wrt the $x'$-axis in the source rest frame. The source is moving with constant velocity $v$ toward an inertial observer $O$.}\\
\textit{What is the angle between the direction of the radiation and the $x$-axis in the frame of $O$?}\\

Let's focus on a photon emitted in $O'$, with a velocity $c$ with an angle $\theta'$ wrt the $x'$-axis. Its vector velocity $\vec{V}'$  wrt $O'$ is:
\begin{align*}
    \vec{V}' = (c \cos \theta', c \sin \theta')
\end{align*}
We need to find $\vec{V}$ in the $O$ frame of reference.\\

Let's start by deriving the formula of velocity addition in a general case. Suppose that $O'$ is moving wrt $O$ at velocity $v$ along a shared $x$-axis (which can be in fact defined as the direction of relative motion). Then consider a particle $P$ moving at velocity $\vec{V}$ as seen by $O$, and $\vec{V}'$ as measured by $O'$.\\
By differentiating the Lorentz transformations:
\begin{align*}
    cdt' &= \gamma c dt - \beta \gamma dx \\
    dx' &= -\beta \gamma c dt + \gamma dx \\
    dy' &=  dy \\
    dz' &=  dz
\end{align*}
Then, starting from the definition of velocity:
\begin{align*}
    V_x' = \dv{x'}{t'} = \frac{\gamma \frac{dx}{dt} - \beta \gamma c \frac{dt}{dt}  }{\gamma \frac{dt}{dt} - \frac{\beta \gamma}{c} \frac{dx}{dt}   }  = \frac{V_x - v}{1-\frac{vV_x}{c^2}}   
\end{align*}
where we divided numerator and denominator by $dt$ (to highlight the velocities wrt $O$), and then by $\gamma$.\\
We can repeat the same algebra for the other two axes:
\begin{align*}
    V_y' &=  \dv{y'}{t'} = \frac{dy}{\gamma dt - \frac{\beta \gamma}{c} dx } = \frac{V_y}{\gamma \left(1-\frac{v V_x}{c^2} \right)}    \\
    V_z' &= \dv{z'}{t'} = \frac{dz}{\gamma dt - \frac{\beta \gamma }{c} dx } = \frac{V_z}{\gamma \left(1-\frac{vV_x}{c^2} \right)}   \\
\end{align*}
Notice how $V_x$ affects the measured velocities along the other two axes.\\

The inverse relations can be simply obtained by substituting $v \leftrightarrow -v$. For this problem, we are interested in the two dimensional case:
\begin{align*}
    V_x &= \frac{V_x + v}{1 + \frac{v V_x}{c^2} }; \qquad V_y = \frac{V_y}{\gamma \left(1+\frac{vV_x}{c^2} \right)} 
\end{align*} 

The angle measured in $O$ is then:
\begin{align*}
    \theta = \arctan\left(\frac{V_y}{V_x} \right)
\end{align*} 
And we can compute the ratio using the formulas derived:
\begin{align*}
    \frac{V_y}{V_x} = \frac{V_y'}{\gamma (V_x' + v)} = \frac{c \sin \theta'}{\gamma(c \cos\theta' + v)} = \frac{\sin\theta'}{\gamma (\cos \theta' + \beta)}    
\end{align*}

\textit{Plot how $\theta$ varies as $v$ increases from $v \ll c$ to $v \approx c$.}\\
If $v \ll c$, $\beta = 0$ and $\gamma = 1$, leading to:
\begin{align*}
    \frac{V_y}{V_x} = \tan\theta' \Rightarrow \theta = \theta'
\end{align*}    
So, in the classical limits, the angle does not change.\\
If $v = c$, however, $\beta= 1$ and $\gamma \to \infty$, leading to:
\begin{align*}
    \frac{V_y}{V_x} \approx 0 \Rightarrow \theta \approx 0 
\end{align*}   
So radiation is \q{bent} on the direction of motion.\\
If we compute the derivative of $\theta$ wrt $v$ we find:
\begin{align*}
    \frac{d}{dv} \frac{\sin \theta'}{\gamma(\cos \theta' + \beta)} = -\frac{\sin\theta'}{\gamma^2(\cos\theta' + \beta)^2} \frac{d}{dv} (\gamma[\cos\theta' + \beta])
\end{align*}  
with:
\begin{align*}
    \frac{d}{dv} \gamma [\cos\theta' + \beta] = \frac{\beta}{c}\frac{1}{(1-\beta^2)^{3/2}} (\cos\theta' + \beta) + \frac{\gamma}{c}  > 0 \text{ if $\theta' \in (0, \pi/2)$ }
\end{align*}
and so the total derivative is always negative, meaning that increasing $v$ decreases the angle $\theta$ from $\theta'$ (with $v=0$) to $0$ (with $v \to c$).\\
In fact, further analysis shows that derivative has a greater absolute value as $v$ approaches $c$, meaning that the limit is reached with a vertical tangent.\\

\textit{Show that the radiation speed measured by $O$ is also $c$}.\\

We want to show that:
\begin{align*}
    \left(\dv{x}{t} \right)^2 + \left(\dv{y}{t} \right)^2 = c^2
\end{align*}
Applying the Lorentz transformations:
\begin{align*}
    dt &= \gamma dt' + \frac{\beta \gamma}{c} dx' \\
    dx &= \gamma dx' + \beta \gamma c dt' \\
    dy &= dy'
\end{align*}
we arrive at:
\begin{align*}
    \left(\dv{x}{t} \right)^2 + \left(\dv{y}{t} \right)^2 = &\frac{1}{(dt)^2} \left(\gamma^2 (dx')^2 + v^2 \gamma^2 (dt')^2 + 2v \gamma^2 dx' + (dy')^2\right) =\\
    &= \gamma^2 c^2 \frac{(dt')^2}{(dt)^2} \left(\cos^2\theta' + \beta^2 + 2 \beta \cos \theta' + \frac{1}{\gamma^2} \sin^2 \theta' \right) = \\
    &= \frac{\gamma^2 c^2 (dt')^2}{\gamma^2 (dt')^2 + \beta^2 \gamma^2 (dx')^2/c^2} \left( \cos^2 \theta' + \beta^2 + 2\beta \cos \theta' + \sin^2 \theta'(1-\beta^2)\right) = \\
    &= c^2 \frac{1}{1+\beta^2 \cos^2 \theta' + 2\beta \cos \theta'} (1+ \beta^2 \cos^2 \theta' + 2\beta\cos \theta') = c^2
\end{align*}

\textit{Assume that the source emits isotropically in its own rest frame. Does $O$ see the source emitting isotropically? Sketch very roughly what $O$ observes, for $v \approx c$.}\\

As $O$ observes rays emitted at $\theta'$ wrt $O'$ at another angle $\theta$, which is a non linear function of $\theta'$, the source appears to $O$ as non-isotropic.\\
For example consider a ray emitted at $\theta' = \pi/2$. The corresponding angle observed by $O$ will be:
\begin{align*}
    \theta = \arctan \left(\frac{1}{\gamma \beta} \right) \neq \frac{\pi}{2} 
\end{align*}  
So, rays emitted within an angle $\alpha'$ from the direction of motion (measured in $O'$, where the source is stationary), are observed by $O$ as emitted within an angle $\alpha < \alpha'$. Basically, radiation is \q{focused} along the direction of motion - and this effect is more and more significant as $v \to c$.\\

More precisely, an isotropic source can be seen as a point emitting photons at an angle $\theta'$ with an uniform pdf:
\begin{align*}
    f(\theta') = \frac{1}{2\pi} \qquad \theta' \in [0, 2\pi) 
\end{align*} 
As $\theta(\theta')$ is monotonic, the pdf transforms under a coordinate change as:
\begin{align*}
    f(\theta') |d \theta'| = g(\theta) |d \theta|
\end{align*} 
where $g(\theta)$ is the pdf seen by $O$. We then derive:
\begin{align*}
    g(\theta) = \underbrace{f(\theta'(\theta))}_{=1}  \left|\dv{\theta'}{\theta} (\theta)\right|
\end{align*} 
The derivative can be computed as:
\begin{align*}
    \dv{\theta'}{\theta} = \left(\dv{\theta}{\theta'} (\theta(\theta')) \right)^{-1}
\end{align*}
Unfortunately, it is difficult to invert $\theta' \mapsto \theta$. However, this can be done numerically, or approximately by using the implicit function theorem and expanding in a Taylor series. However, it is clear that $O$ sees the source as emitting isotropically only if $g(\theta)$ is constant - which is of course not the case.  

\chapter{Week 2}
\section{Problem 1}
\textit{Consider a rocket moving along the $x$-axis whose position as a function of time is:}
\begin{align}
    x(t) = \frac{c}{k}(\sqrt{1+k^2 t^2}-1) \qquad k>0; \> [k] = \si{\m\per\s}
    \label{eqn:x-t} 
\end{align} 
\textit{where $c$ is the speed of light, and $k$ a positive constant having the dimensions of an inverse time. Namely the rocket starts at $x=\infty$ at time $t=-\infty$, reaches $x=0$ at time $t=0$ and then goes back to $x=\infty$ at the time $t=\infty$}
\begin{enumerate}
    \item \textit{Find and plot the velocity $v_x = \dd{x}/\dd{t}$ of the rocket as a function of time, and show that the speed (the magnitude of the velocity) of the rocket never exceeds $c$.}\\
    Differentiating $x(t)$ with respect to $t$:
    \begin{align*}
        v_x(t) \equiv \dv{x}{t} = \frac{ckt}{\sqrt{1+k^2 t^2}} = \pm \frac{c}{\sqrt{1+\frac{1}{k^2 t^2} }} 
    \end{align*}   
    Note that $v_x(-\infty) = -c/k$, $v_x(0) = 0$ and $v_x(\infty) = c/k$. In fact, $v_x(t)$ is always increasing:
    \begin{align*}
        \dv{v_x}{t} = \frac{ck}{(1+k^2t^2)^{3/2}} > 0 \> \quad \forall t 
    \end{align*}    
    If we compute the square modulus it is easy to see that $v_x$ is always strictly less than $c$.  
    \begin{align*}
        \left| \dv{x}{t} \right|^2 = \frac{c^2}{1+\frac{1}{k^2 t^2}} < c \qquad \forall t \> \forall k > 0
    \end{align*}
    \item \textit{Calculate the components of the rocket $4$-velocity $u^\mu$}\\
    Recall the expression for the components of $u^\mu$:
    \begin{align*}
        u^\mu = (\gamma c, \gamma \vec{v})
    \end{align*} 
    with:
    \begin{align*}
        \vec{v} = \left(\frac{ckt}{\sqrt{1+k^2 t^2}}, 0, 0\right)
    \end{align*}
    We start by computing $\gamma$:
    \begin{align*}
        \gamma^2 = \frac{1}{1-\frac{v^2}{c^2} }  = 1 + k^2 t^2 \Rightarrow \gamma = \sqrt{1+k^2 t^2}
    \end{align*} 
    leading to:
    \begin{align*}
        u^\mu = (c(1+k^2t^2)^{1/2}, ckt)
    \end{align*}
    \item \textit{Express $x$ and $t$ as a function of proper time $\tau$ along the trajectory of the rocket (integrate the relation between $\dd{t}$ and $\dd{\tau}$ and impose that $\tau = 0$ when $t=0$)}\\
    We start from:
    \begin{align*}
        c^2 \dd{\tau}^2 &= c^2 \dd{t}^2 - \dd{x}^2 = \dd{t}^2 - \left(\dv{x}{t}\right)^2 \dd{t}^2 =\\
        &= c^2 \dd{t}^2 - \frac{c^2 k^2 t^2}{1+k^2 t^2} \dd{t}^2 = \frac{c^2}{1+k^2 t^2} \dd{t}^2 \Rightarrow \dd{\tau} = \frac{1}{\gamma}\dd{t}   
    \end{align*} 
    which is the same result obtained from the \textit{time dilation} formula. Then, by integrating both sides:
    \begin{align*}
        \tau = \int \frac{1}{\sqrt{1+k^2 t^2}}\dd{t} = \frac{1}{k} \tanh^{-1}(kt) + c  
    \end{align*} 
    and setting $t=0$ when $\tau = 0$ leads to $c = 0$.
    \begin{expl}
        Alternatively, note that $u^0 = \dd{t}/\dd{\tau} = \gamma c$, and so $\dd{\tau} = \dd{t}/(\gamma c)$.  
    \end{expl}
    We can now invert the relation to find $t(\tau)$:
    \begin{align*}
        t(\tau) = \frac{1}{k} \sinh(k \tau) 
    \end{align*}
    and substitute in (\ref{eqn:x-t}) to get $x(\tau)$:
    \begin{align*}
        x(\tau) = \frac{c}{k}(\cosh(k \tau)-1) 
    \end{align*} 
    \item \textit{What is the $4$-acceleration $a^\mu = \dd{u^\mu}/\dd{\tau}$ of the rocket?}
    \item \textit{Compute the $4$-velocity $u^\mu$ and the $4$-acceleration $a^\mu$ in the instantaneous rest frame of the rocket. This is the frame in which, at any given time, the rocket is at rest. In practice, you need to take the $4$-velocity and the $4$-acceleration computed above, and do a Lorentz boost with the velocity that the particle has at that time. Recall that, at small speeds, the spatial components of the acceleration reduce to the non-relativistic acceleration. Therefore, the spatial components of the $4$-acceleration in the instantaneous rest-frame give the acceleration experienced by people on the rocket. For this problem, you should find that this acceleration is constant in time, and equal to $ck$.}\\  

\end{enumerate}

\section{Problem 2}
\textit{A proton (particle A) of mass $m$ and an anti-proton (particle B) annihilate in two photons (particle C and D). Energy and momentum are conserved in the process, $p_A^\mu + p_B^\mu = p_C^\mu + p^\mu_D$. Denote with $p$ the spatial component of $p_A^\mu$ along the $x$-axis.}

\begin{enumerate}
    \item \textit{Write the $4$-momenta of the four particles in the center of mass frame.}\\
    In the center of mass frame, $\vec{p}_A \parallel \hat{x}$, and $\vec{p}_C$ forms an angle of $\theta$ with $\hat{x}$. As total momentum wrt center of mass is $0$, $\vec{p}_A = -\vec{p}_B$ and $\vec{p}_C = - \vec{p}_D$.\\
    Recall that, for a massive particle, the $4$-momentum has the following components:
    \begin{align*}
        p^\mu = (m \gamma c, m \gamma \vec{v}) = \left(\frac{\mathcal{E}}{c}, \vec{p}\right)
    \end{align*}           
    And its self-contraction leads to the mass-shell relation:
    \begin{align*}
        p^\mu p_\mu = -\frac{\mathcal{E}^2}{c^2} + \norm{\vec{p}}^2 = -mc^2 \Rightarrow \mathcal{E}^2 = m^2 c^4 + \norm{\vec{p}}^2 c^2 
    \end{align*}
    Then:
    \begin{align*}
        p^0 = \frac{\mathcal{E}}{c} = \frac{\sqrt{m^2 c^4 + p^2 c^2}}{c}  
    \end{align*}
    leading to:
    \begin{align*}
        p_A^\mu &= \left(\frac{\sqrt{m^2 c^4 + p^2 c^2}}{c}, +p, 0, 0 \right)\\
        p_B^\mu &= \left(\frac{\sqrt{m^2 c^4 + p^2 c^2}}{c}, -p, 0, 0 \right)
    \end{align*}
    as $A$ and $B$ collide \q{head on} in the center of mass frame.\\
    Photons $C$ and $D$ will have the same energy $\mathcal{E}$ of the protons, but with a differently directed momentum ($=\mathcal{E}/c$ from mass-shell):
    \begin{align*}
        p_\mu^C &= \left(\frac{\mathcal{E}}{c}, \frac{\mathcal{E}}{c} \cos \theta, \frac{\mathcal{E}}{c} \sin \theta, 0   \right)\\
        p_\mu^D &= \left(\frac{\mathcal{E}}{c}, -\frac{\mathcal{E}}{c} \cos \theta, -\frac{\mathcal{E}}{c} \sin \theta, 0   \right)
    \end{align*}     

    \item \textit{Find the velocity of the particle B in the center of mass frame}.\\
    We start from:
    \begin{align*}
        p^1 = m \gamma v = p \Rightarrow v = \frac{p}{m \gamma}    
    \end{align*}
    \item \textit{Find the $4$-momenta of the two initial particles in the lab frame (it is the rest frame of B). Hint: perform a Lorentz boost with the velocity obtained in (2) of the quantities obtained in (1).}   
\end{enumerate}

 
\end{document}
